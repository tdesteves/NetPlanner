\clearpage

\section{Opaque without Survivability - Reference Network} \label{Reference_Network}

In this case study we focus on the opaque case without survivability for the reference network.
The opaque transport mode performs OEO (optical-electric-optical) conversions on each intermediate node from the source to the destination node.
One advantage of this mode of transport is that it eliminates accumulation of physical impairments, and allows optimum grooming by performing grooming at each node.

\subsection{Physical Network Topology}\label{Reference_Network_Topology}
\begin{tcolorbox}	
\begin{tabular}{p{2.75cm} p{0.2cm} p{10.5cm}} 	
\textbf{Student Name}  &:& Tiago Esteves    (October 03, 2017 - )\\
\end{tabular}
\end{tcolorbox}


In the figure below we ca see that our reference network consists of 6 nodes and 8 Bidirectional links.
The average length of the links was chosen so that the following calculations are more simplistic, for this was created a matrix of distances between the respective nodes.
Finally, ODU's matrices were also created to be able to determine the total traffic used in each scenario.

\begin{figure}[h!]
\centering
\includegraphics[width=\textwidth]{RedeTeste}
\caption{Physical topology of the reference network.}
\end{figure}

The distance matrix is the same for the two scenarios but the ODU's matrices are not.
In this way only the matrices for the case of low traffic are elucidated, being that in the case of a high traffic it is only necessary to multiply these matrices by the value 10.

\[
Dist=
  \begin{bmatrix}
    0 & 500 & 500 & 0 & 0 & 0 \\
    500 & 0 & 400 & 500 & 0 & 0 \\
    500 & 400 & 0 & 0 & 500 & 0 \\
    0 & 500 & 0 & 0 & 600 & 450 \\
    0 & 0 & 500 & 600 & 0 & 550 \\
    0 & 0 & 0 & 450 & 550 & 0
  \end{bmatrix}
\]

\[
ODU0=
  \begin{bmatrix}
    0 & 5 & 1 & 3 & 1 & 3 \\
    5 & 0 & 0 & 1 & 5 & 0 \\
    1 & 0 & 0 & 1 & 4 & 1 \\
    3 & 1 & 1 & 0 & 1 & 0 \\
    1 & 5 & 4 & 1 & 0 & 3 \\
    3 & 0 & 1 & 1 & 3 & 0
  \end{bmatrix}
\quad ODU1=
  \begin{bmatrix}
    0 & 2 & 4 & 2 & 0 & 5 \\
    2 & 0 & 0 & 3 & 1 & 1 \\
    4 & 0 & 0 & 1 & 1 & 0 \\
    3 & 3 & 1 & 0 & 1 & 3 \\
    0 & 1 & 1 & 1 & 0 & 1 \\
    5 & 1 & 0 & 3 & 1 & 0
  \end{bmatrix}
\quad ODU2=
  \begin{bmatrix}
    0 & 1 & 1 & 1 & 0 & 0 \\
    1 & 0 & 0 & 0 & 1 & 0 \\
    1 & 0 & 0 & 1 & 1 & 0 \\
    1 & 0 & 1 & 0 & 1 & 0 \\
    0 & 1 & 1 & 1 & 0 & 1 \\
    0 & 0 & 0 & 0 & 1 & 0
  \end{bmatrix}
\]
\[
ODU3=
  \begin{bmatrix}
    0 & 0 & 0 & 0 & 0 & 0 \\
    0 & 0 & 1 & 0 & 0 & 1 \\
    0 & 1 & 0 & 0 & 1 & 0 \\
    0 & 0 & 0 & 0 & 0 & 0 \\
    0 & 0 & 1 & 0 & 0 & 0 \\
    0 & 1 & 0 & 0 & 0 & 0
  \end{bmatrix}
\qquad ODU4=
  \begin{bmatrix}
    0 & 0 & 0 & 0 & 0 & 0 \\
    0 & 0 & 0 & 0 & 0 & 1 \\
    0 & 0 & 0 & 0 & 0 & 0 \\
    0 & 0 & 0 & 0 & 0 & 0 \\
    0 & 0 & 0 & 0 & 0 & 1 \\
    0 & 1 & 0 & 0 & 1 & 0
  \end{bmatrix}
\]

\vspace{17pt}

The values indicated in the distance matrix, referred to below, are expressed in kilometers (Km) and as it couldn't be otherwise, this matrix is symmetric.
In relation to the traffic matrices each ODU, referred previously, has its respective value being that the ODU0 corresponds to 1.25 Gbits/s, ODU1 to 2.5 Gbits/s, ODU2 to 10 Gbits/s, ODU3 to 40 Gbits/s and finally the ODU4 corresponds to 100 Gbits/s.
As we can see these matrices are bidirectional because they are symmetric matrices and as such the traffic sent in one direction must be the same traffic sent in the opposite direction. \\

Through these ODU's we can calculate total network traffic for the low traffic scenario:\\

$T_1^0$ = 60x1.25 = 75 Gbits/s \qquad
$T_1^1$ = 50x2.5 = 125 Gbits/s \qquad
$T_1^2$ = 16x10 = 160 Gbits/s \\

$T_1^3$ = 6x40 = 240 Gbits/s \quad
$T_1^4$ = 4x100 = 400 Gbits/s \\

$T_{1}$ = 75 + 125 + 160 + 240 + 400 = 1000 Gbits/s \qquad
$T$ = 1000/2 = 0.5 Tbits/s\\

Where the variable $T_1^x$ represents the unidirectional traffic of the ODUx, for example, $T_1^0$ represents the unidirectional traffic of the ODU0 and $T_1^1$ represents the unidirectional traffic of the ODU1. The variable $T_{1}$ represents the total of unidirectional traffic that is injected into the network and finally the variable $T$ represents the total of bidirectional traffic.\\

We can thus conclude that the total traffic for the two scenarios is as follows:
\begin{itemize}
  \item Low Traffic: \textbf{0.5 TBits/s}
  \item High Traffic: \textbf{5 TBits/s}
\end{itemize}

Finally for this project has to take into consideration the table \ref{table_ref_net} because in it we can see the values of the variables associated with this network.
\begin{table}[h!]
\centering
\begin{tabular}{|| c | c | c||}
 \hline
 Constant & Description & Value \\
 \hline\hline
 N & Number of nodes & 6 \\
 L & Number of bidirectional links & 8 \\
 <$\delta$> & Node out-degree & 2.667 \\
 <len> & Mean link length (km) & 500 \\
 <h> & Mean number of hops for working paths & 1.533 \\
 <h'> & Mean number of hops for backup paths & 2.467 \\
 \hline
\end{tabular}
\caption{Table of reference network values}
\label{table_ref_net}
\end{table}


\subsection{Dimensioning using ILP}
\begin{tcolorbox}	
\begin{tabular}{p{2.75cm} p{0.2cm} p{10.5cm}} 	
\textbf{Student Name}  &:& Tiago Esteves    (October 03, 2017 - )\\
\end{tabular}
\end{tcolorbox}

\vspace{11pt}
In this section we will do the dimensioning of the network mentioned in the previous section to calculate the value of your CAPEX, for this we will use the ILP model describe in section \ref{ILP_Opaque_Survivability} and we can get the best possible solution.
In the initial subsection will be described the network cost where all the formulas and calculations necessary to obtain the CAPEX of the network will be mentioned.
Finally, in the last subsection, the results obtained through the described model will be presented.

\subsubsection{Network costs}\label{Net_Costs}

In this phase the results will be presented to calculate the CAPEX of the reference network.
The value of the CAPEX of the network will be calculated based on the costs of the equipment present in the table below.\\

\begin{table}[h!]
\centering
\begin{tabular}{|| c | c||}
 \hline
 Equipment & Cost \\
 \hline\hline
 OLT without transponders & 15000 \euro \\
 Transponder & 5000 \euro/Gb \\
 Optical Amplifier & 4000 \euro \\
 EXC & 10000 \euro \\
 OXC & 20000 \euro \\
 EXC Port & 1000 \euro /Gb/s\\
 OXC Port & 2500 \euro /porto \\
 \hline
\end{tabular}
\caption{Table with costs}
\label{table_cost_opaque}
\end{table}

In addition to the equipment costs we will also use the parameter "span", which in this case will have a value of 100, because this value is used to calculate the number of optical amplifiers required in the network using Equation \ref{amplifiers}.

\begin{equation}
N^R = \sum\limits_{l=1}^L\left(\left\lceil\frac{len_l}{span}\right\rceil-1\right)
\label{amplifiers}
\end{equation}

The other parameters of this equation are:
\begin{itemize}
\item{$N^R$			$\rightarrow$ Total number of regenerators/amplifiers}
\item{$len_l$		$\rightarrow$ Length of link l}
\item{$span$		$\rightarrow$ Distance between amplifiers}	
\end{itemize}	


To know the value of CAPEX it is necessary to know the value of the cost of the links and the cost of the nodes.
To calculate the cost of the nodes, the sum of the costs of the optical and electrical node is made. For this case the value of the optical cost is zero only needing to know the electric cost of the nodes that is given by equation \ref{electricalCostOpaque}.

\begin{equation}
C_{exc} = \left(\gamma_{e0}\times N\right) + \gamma_{e1} \times \left(T_1 + \left(2 \times w^0 \times \tau \right)\right)
\label{electricalCostOpaque}
\end{equation}

\begin{itemize}
\item{$C_{exc}$		$\rightarrow$	Electrical ports cost}
\item{$\gamma_{e0}$	$\rightarrow$	EXC cost in euros}
\item{$N$			$\rightarrow$	Number of nodes}
\item{$\gamma_{e1}$	$\rightarrow$	EXC port cost in euros}
\item{$T_1$         $\rightarrow$   Total unidirectional traffic}
\item{$w^0$			$\rightarrow$	Total number of optical channels}
\item{$\tau$		$\rightarrow$	Traffic per port}
\end{itemize}

\vspace{11pt}

To calculate the cost of the Links we will use the equation \ref{linkCosts}.

\begin{equation}
C_L = \left(2 \times \gamma_0^{OLT} \times L\right) + \left(2 \times \gamma_1^{OLT} \times \tau \times W\right) + \left(N^R \times c^R\right)
\label{linkCosts}
\end{equation}	
	
\begin{itemize}
\item{$C_L$				$\rightarrow$	Links cost}
\item{$\gamma_0^{OLT}$	$\rightarrow$	OLT cost in euros}
\item{$L$				$\rightarrow$	Number of unidirectional links}
\item{$\gamma_1^{OLT}$	$\rightarrow$	Transponder cost in euros}
\item{$W$             $\rightarrow$	    Total number of optical channels}
\item{$N^R$				$\rightarrow$	Total number of optical amplifiers}
\item{$c^R$				$\rightarrow$	Optical amplifiers cost in euros}
\end{itemize}


\subsubsection{ILP Results}

To perform the calculations using the implementation of the models described in section \ref{ILP_Opaque_Survivability} it is necessary to use a mathematical software tool. For this we will use MATLAB which is ideal for dealing with linear programming problems and can call the LPsolve through an external interface. \\

\textbf{Scenario 1: Reference Network Low Traffic} \label{Scenario1_opaque} \\

In this scenario we used the table \ref{table_ref_net}. In the table \ref{result_ILP1_reference} we can see the values calculated through MatLab and using the values indicated in table \ref{table_cost_opaque} we can finally calculate the CAPEX value.\\

\begin{table}[h!]
\centering
\begin{tabular}{|| c | c||}
 \hline
 Number of optical channels & Value \\
 \hline\hline
 in the link (1,2) &  \\
 in the link (1,3) &  \\
 in the link (2,3) &  \\
 in the link (2,4) &  \\
 in the link (3,5) &  \\
 in the link (4,5) &  \\
 in the link (4,6) &  \\
 in the link (5,6) &  \\
 \hline
\end{tabular}
\caption{Table with results}
\label{result_ILP1_reference}
\end{table}

Using equation \ref{linkCosts} : \\
$C_L$ = $($2 * 15 000 * 8$)$ + $($2 * 5 000 * 100 * $)$ + $($24 * 4 000$)$ \\
$C_L$ = \textbf{ \euro} \\


Using equation \ref{electricalCostOpaque} : \\
$C_{exc}$ = $($6 * 10 000$)$ + 1 000 * $($1 000 + $($2 *  * 100$)$ $)$ \\
$C_N$ = $C_{exc}$ = \textbf{ \euro} \\

$CAPEX$ =  +  = \textbf{ \euro}\\


\textbf{Scenario 2: Reference Network High Traffic} \label{Scenario2_opaque} \\

In this scenario we used again the table \ref{table_ref_net}. In the table \ref{result_ILP2_reference} we can see the values calculated through MatLab and using the values indicated in table \ref{table_cost_opaque} we can finally calculate the CAPEX value.
\newpage

\begin{table}[h!]
\centering
\begin{tabular}{|| c | c||}
 \hline
 Number of optical channels & Value \\
 \hline\hline
 in the link (1,2) &  \\
 in the link (1,3) &  \\
 in the link (2,3) &  \\
 in the link (2,4) &  \\
 in the link (3,5) &  \\
 in the link (4,5) &  \\
 in the link (4,6) &  \\
 in the link (5,6) &  \\
 \hline
\end{tabular}
\caption{Table with results}
\label{result_ILP2_reference}
\end{table}

Using equation \ref{linkCosts} : \\
$C_L$ = $($2 * 15 000 * 8$)$ + $($2 * 5 000 * 100 * $)$ + $($24 * 4 000$)$ \\
$C_L$ = \textbf{ \euro} \\

Using equation \ref{electricalCostOpaque} : \\
$C_{exc}$ = $($6 * 10 000$)$ + 1 000 * $($10 000 + $($2 *  * 100$)$ $)$ \\
$C_N$ = $C_{exc}$ = \textbf{ \euro} \\

$CAPEX$ =  +  = \textbf{ \euro}\\

\subsection{Dimensioning using Heuristics}

\subsubsection{Heuristics Results}

\subsection{Comparative Analysis}

\newpage
\section{Opaque without Survivability - European Optical Network} \label{Realistic_Network}

In this case study we focus on the opaque case without survivability for the realistic network.

\subsection{Physical Network Topology}
\begin{tcolorbox}	
\begin{tabular}{p{2.75cm} p{0.2cm} p{10.5cm}} 	
\textbf{Student Name}  &:& Tiago Esteves    (October 03, 2017 - )\\
\end{tabular}
\end{tcolorbox}

The real network chosen for this work is the EON (European Optical Network).

\begin{figure}[h!]
\centering
\includegraphics[width=\textwidth]{EON_Rede_Realista}
\caption{Physical topology of the realistic network.}
\end{figure}

In this real case we have take into consideration the table \ref{table_real_net} because it is through it that we can see the values of the variables associated with this network.

\begin{table}[h!]
\centering
\begin{tabular}{|| c | c | c||}
 \hline
 Constant & Description & Value \\
 \hline\hline
 N & Number of nodes & 19 \\
 L & Number of bidirectional links & 37 \\
 <$\delta$> & Node out-degree & 3.89 \\
 <len> & Mean link length (km) & 753.76 \\
 <h> & Mean number of hops for working paths & 2.3 \\
 <h'> & Mean number of hops for backup paths & 3.2 \\
 \hline
\end{tabular}
\caption{Table of realistic network values}
\label{table_real_net}
\end{table}

\newpage
Through the previous figure we can see how nodes are organized geographically and the distance matrix created on the next page is constructed based on real distances between them.
For a better understanding of the distances matrix the table \ref{city_nodes_realnet} was created to assign to each city a number of a node in the network.


\begin{table}[h!]
\centering
\begin{tabular}{|| c | c ||}
 \hline
 City & Node \\
 \hline\hline
 Oslo & 1 \\
 Stockholm & 2 \\
 Moscow & 3 \\
 Copenhagen & 4 \\
 Berlin & 5 \\
 Prague & 6 \\
 Vienna & 7 \\
 Zagreb & 8 \\
 Athens & 9 \\
 Rome & 10 \\
 Milan & 11 \\
 Zurich & 12 \\
 Brussels & 13 \\
 Amesterdan & 14 \\
 London & 15 \\
 Dublin & 16 \\
 Paris & 17 \\
 Madrid & 18 \\
 Lisbon & 19 \\
 \hline
\end{tabular}
\caption{Table of city and respective node}
\label{city_nodes_realnet}
\end{table}


The values indicated in the distance matrix, referred to below, are expressed in kilometers (Km).
For this network we must also create matrices of ODU's to determine the total traffic used in each scenario but in this case only the matrices for low traffic are elucidated.

\begin{sidewaysfigure}

\[
Dist=
  \begin{pmatrix}
    0 & 417 & 0 & 484 & 0 & 0 & 0 & 0 & 0 & 0 & 0 & 0 & 0 & 0 & 1155 & 0 & 0 & 0 & 0 \\
    417 & 0 & 1228 & 523 & 811 & 0 & 0 & 0 & 0 & 0 & 0 & 0 & 0 & 0 & 0 & 0 & 0 & 0 & 0 \\
    0 & 1228 & 0 & 0 & 1611 & 0 & 0 & 0 & 0 & 0 & 0 & 0 & 0 & 0 & 0 & 0 & 0 & 0 & 0 \\
    484 & 523 & 0 & 0 & 0 & 0 & 0 & 0 & 0 & 0 & 0 & 0 & 0 & 622 & 0 & 0 & 0 & 0 & 0 \\
    0 & 811 & 1611 & 0 & 0 & 281 & 524 & 0 & 0 & 0 & 843 & 0 & 0 & 577 & 933 & 0 & 0 & 0 & 0 \\
    0 & 0 & 0 & 0 & 281 & 0 & 251 & 0 & 0 & 0 & 646 & 527 & 0 & 712 & 0 & 0 & 0 & 0 & 0 \\
    0 & 0 & 0 & 0 & 524 & 251 & 0 & 268 & 0 & 0 & 0 & 0 & 0 & 0 & 0 & 0 & 0 & 0 & 0 \\
    0 & 0 & 0 & 0 & 0 & 0 & 268 & 0 & 1081 & 518 & 530 & 0 & 0 & 0 & 0 & 0 & 0 & 0 & 0 \\
    0 & 0 & 0 & 0 & 0 & 0 & 0 & 1081 & 0 & 1052 & 0 & 0 & 0 & 0 & 0 & 0 & 0 & 0 & 0 \\
    0 & 0 & 0 & 0 & 0 & 0 & 0 & 518 & 1052 & 0 & 477 & 0 & 0 & 0 & 0 & 0 & 0 & 0 & 0 \\
    0 & 0 & 0 & 0 & 843 & 646 & 0 & 530 & 0 & 477 & 0 & 219 & 0 & 0 & 0 & 0 & 640 & 0 & 0 \\
    0 & 0 & 0 & 0 & 0 & 646 & 0 & 0 & 0 & 0 & 219 & 0 & 493 & 0 & 0 & 0 & 488 & 0 & 0 \\
    0 & 0 & 0 & 0 & 0 & 0 & 0 & 0 & 0 & 0 & 0 & 493 & 0 & 173 & 321 & 0 & 264 & 0 & 0 \\
    0 & 0 & 0 & 622 & 577 & 712 & 0 & 0 & 0 & 0 & 0 & 0 & 173 & 0 & 358 & 0 & 0 & 0 & 0 \\
    1155 & 0 & 0 & 0 & 933 & 0 & 0 & 0 & 0 & 0 & 0 & 0 & 321 & 358 & 0 & 464 & 344 & 0 & 1587 \\
    0 & 0 & 0 & 0 & 0 & 0 & 0 & 0 & 0 & 0 & 0 & 0 & 0 & 0 & 464 & 0 & 782 & 0 & 0 \\
    0 & 0 & 0 & 0 & 0 & 0 & 0 & 0 & 0 & 0 & 640 & 488 & 264 & 0 & 344 & 782 & 0 & 1054 & 0 \\
    0 & 0 & 0 & 0 & 0 & 0 & 0 & 0 & 0 & 0 & 0 & 0 & 0 & 0 & 0 & 0 & 1054 & 0 & 503 \\
    0 & 0 & 0 & 0 & 0 & 0 & 0 & 0 & 0 & 0 & 0 & 0 & 0 & 0 & 1587 & 0 & 0 & 503 & 0
  \end{pmatrix}
\]
\end{sidewaysfigure}

\[
ODU0=
  \begin{bmatrix}
    0 & 1 & 1 & 1 & 1 & 1 & 1 & 1 & 1 & 1 & 1 & 1 & 1 & 1 & 1 & 1 & 1 & 1 & 1 \\
    1 & 0 & 1 & 1 & 1 & 1 & 1 & 1 & 1 & 1 & 1 & 1 & 1 & 1 & 1 & 1 & 1 & 1 & 1 \\
    1 & 1 & 0 & 1 & 1 & 1 & 1 & 1 & 1 & 1 & 1 & 1 & 1 & 1 & 1 & 1 & 1 & 1 & 1 \\
    1 & 1 & 1 & 0 & 1 & 1 & 1 & 1 & 1 & 1 & 1 & 1 & 1 & 1 & 1 & 1 & 1 & 1 & 1 \\
    1 & 1 & 1 & 1 & 0 & 1 & 1 & 1 & 1 & 1 & 1 & 1 & 1 & 1 & 1 & 1 & 1 & 1 & 1 \\
    1 & 1 & 1 & 1 & 1 & 0 & 1 & 1 & 1 & 1 & 1 & 1 & 1 & 1 & 1 & 1 & 1 & 1 & 1 \\
    1 & 1 & 1 & 1 & 1 & 1 & 0 & 1 & 1 & 1 & 1 & 1 & 1 & 1 & 1 & 1 & 1 & 1 & 1 \\
    1 & 1 & 1 & 1 & 1 & 1 & 1 & 0 & 1 & 1 & 1 & 1 & 1 & 1 & 1 & 1 & 1 & 1 & 1 \\
    1 & 1 & 1 & 1 & 1 & 1 & 1 & 1 & 0 & 1 & 1 & 1 & 1 & 0 & 0 & 0 & 0 & 0 & 0 \\
    1 & 1 & 1 & 1 & 1 & 1 & 1 & 1 & 1 & 0 & 0 & 0 & 0 & 0 & 0 & 0 & 0 & 0 & 0 \\
    1 & 1 & 1 & 1 & 1 & 1 & 1 & 1 & 1 & 0 & 0 & 0 & 0 & 0 & 0 & 0 & 0 & 0 & 0 \\
    1 & 1 & 1 & 1 & 1 & 1 & 1 & 1 & 1 & 0 & 0 & 0 & 0 & 0 & 0 & 0 & 0 & 0 & 0 \\
    1 & 1 & 1 & 1 & 1 & 1 & 1 & 1 & 1 & 0 & 0 & 0 & 0 & 0 & 0 & 0 & 0 & 0 & 0 \\
    1 & 1 & 1 & 1 & 1 & 1 & 1 & 1 & 0 & 0 & 0 & 0 & 0 & 0 & 0 & 0 & 0 & 0 & 0 \\
    1 & 1 & 1 & 1 & 1 & 1 & 1 & 1 & 0 & 0 & 0 & 0 & 0 & 0 & 0 & 0 & 0 & 0 & 0 \\
    1 & 1 & 1 & 1 & 1 & 1 & 1 & 1 & 0 & 0 & 0 & 0 & 0 & 0 & 0 & 0 & 0 & 0 & 0 \\
    1 & 1 & 1 & 1 & 1 & 1 & 1 & 1 & 0 & 0 & 0 & 0 & 0 & 0 & 0 & 0 & 0 & 0 & 0 \\
    1 & 1 & 1 & 1 & 1 & 1 & 1 & 1 & 0 & 0 & 0 & 0 & 0 & 0 & 0 & 0 & 0 & 0 & 0 \\
    1 & 1 & 1 & 1 & 1 & 1 & 1 & 1 & 0 & 0 & 0 & 0 & 0 & 0 & 0 & 0 & 0 & 0 & 0
  \end{bmatrix}
\]

\vspace{15pt}

\[
ODU1=
  \begin{bmatrix}
    0 & 1 & 0 & 0 & 0 & 0 & 0 & 0 & 0 & 0 & 1 & 1 & 1 & 1 & 1 & 1 & 1 & 0 & 0 \\
    1 & 0 & 0 & 0 & 0 & 0 & 0 & 0 & 0 & 0 & 1 & 1 & 1 & 1 & 1 & 1 & 1 & 1 & 1 \\
    0 & 0 & 0 & 0 & 0 & 1 & 1 & 1 & 1 & 1 & 1 & 1 & 1 & 1 & 1 & 1 & 1 & 1 & 1 \\
    0 & 0 & 0 & 0 & 1 & 1 & 1 & 1 & 1 & 1 & 1 & 1 & 1 & 1 & 1 & 1 & 1 & 1 & 1 \\
    0 & 0 & 0 & 1 & 0 & 1 & 1 & 1 & 1 & 1 & 1 & 1 & 1 & 1 & 1 & 1 & 1 & 1 & 1 \\
    0 & 0 & 1 & 1 & 1 & 0 & 1 & 1 & 1 & 1 & 1 & 1 & 1 & 1 & 1 & 1 & 1 & 1 & 1 \\
    0 & 0 & 1 & 1 & 1 & 1 & 0 & 1 & 1 & 1 & 1 & 1 & 1 & 1 & 1 & 1 & 1 & 1 & 1 \\
    0 & 0 & 1 & 1 & 1 & 1 & 1 & 0 & 1 & 1 & 1 & 1 & 1 & 1 & 1 & 1 & 1 & 1 & 1 \\
    0 & 0 & 1 & 1 & 1 & 1 & 1 & 1 & 0 & 1 & 1 & 1 & 1 & 0 & 0 & 0 & 0 & 0 & 0 \\
    0 & 0 & 1 & 1 & 1 & 1 & 1 & 1 & 1 & 0 & 0 & 0 & 0 & 0 & 0 & 0 & 0 & 0 & 0 \\
    1 & 1 & 1 & 1 & 1 & 1 & 1 & 1 & 1 & 0 & 0 & 0 & 0 & 0 & 0 & 0 & 0 & 0 & 0 \\
    1 & 1 & 1 & 1 & 1 & 1 & 1 & 1 & 1 & 0 & 0 & 0 & 0 & 0 & 0 & 0 & 0 & 0 & 0 \\
    1 & 1 & 1 & 1 & 1 & 1 & 1 & 1 & 1 & 0 & 0 & 0 & 0 & 0 & 0 & 0 & 0 & 0 & 0 \\
    1 & 1 & 1 & 1 & 1 & 1 & 1 & 1 & 0 & 0 & 0 & 0 & 0 & 0 & 0 & 0 & 0 & 0 & 0 \\
    1 & 1 & 1 & 1 & 1 & 1 & 1 & 1 & 0 & 0 & 0 & 0 & 0 & 0 & 0 & 0 & 0 & 0 & 0 \\
    1 & 1 & 1 & 1 & 1 & 1 & 1 & 1 & 0 & 0 & 0 & 0 & 0 & 0 & 0 & 0 & 0 & 0 & 0 \\
    1 & 1 & 1 & 1 & 1 & 1 & 1 & 1 & 0 & 0 & 0 & 0 & 0 & 0 & 0 & 0 & 0 & 0 & 0 \\
    0 & 1 & 1 & 1 & 1 & 1 & 1 & 1 & 0 & 0 & 0 & 0 & 0 & 0 & 0 & 0 & 0 & 0 & 0 \\
    0 & 1 & 1 & 1 & 1 & 1 & 1 & 1 & 0 & 0 & 0 & 0 & 0 & 0 & 0 & 0 & 0 & 0 & 0
  \end{bmatrix}
\]


\[
ODU2=
  \begin{bmatrix}
    0 & 1 & 1 & 2 & 2 & 0 & 0 & 0 & 0 & 0 & 0 & 0 & 0 & 0 & 0 & 0 & 0 & 0 & 0 \\
    1 & 0 & 2 & 2 & 2 & 2 & 0 & 0 & 0 & 0 & 0 & 0 & 0 & 0 & 0 & 0 & 0 & 0 & 0 \\
    1 & 2 & 0 & 2 & 2 & 2 & 0 & 0 & 0 & 0 & 0 & 0 & 0 & 0 & 0 & 0 & 0 & 0 & 0 \\
    2 & 2 & 2 & 0 & 2 & 2 & 2 & 0 & 0 & 0 & 0 & 0 & 0 & 0 & 0 & 0 & 0 & 0 & 0 \\
    2 & 2 & 2 & 2 & 0 & 0 & 0 & 0 & 0 & 0 & 0 & 0 & 0 & 0 & 0 & 0 & 0 & 0 & 0 \\
    0 & 2 & 2 & 2 & 0 & 0 & 2 & 0 & 2 & 2 & 0 & 0 & 0 & 0 & 0 & 0 & 0 & 0 & 0 \\
    0 & 0 & 0 & 2 & 0 & 2 & 0 & 0 & 0 & 0 & 0 & 0 & 0 & 0 & 0 & 0 & 0 & 0 & 0 \\
    0 & 0 & 0 & 0 & 0 & 0 & 0 & 0 & 0 & 0 & 0 & 0 & 0 & 0 & 0 & 0 & 0 & 0 & 0 \\
    0 & 0 & 0 & 0 & 0 & 2 & 0 & 0 & 0 & 0 & 0 & 0 & 0 & 0 & 0 & 0 & 0 & 0 & 0 \\
    0 & 0 & 0 & 0 & 0 & 2 & 0 & 0 & 0 & 0 & 0 & 0 & 0 & 0 & 0 & 0 & 0 & 0 & 0 \\
    0 & 0 & 0 & 0 & 0 & 0 & 0 & 0 & 0 & 0 & 0 & 0 & 0 & 0 & 0 & 0 & 0 & 0 & 0 \\
    0 & 0 & 0 & 0 & 0 & 0 & 0 & 0 & 0 & 0 & 0 & 0 & 0 & 0 & 0 & 0 & 0 & 0 & 0 \\
    0 & 0 & 0 & 0 & 0 & 0 & 0 & 0 & 0 & 0 & 0 & 0 & 0 & 0 & 0 & 0 & 0 & 0 & 0 \\
    0 & 0 & 0 & 0 & 0 & 0 & 0 & 0 & 0 & 0 & 0 & 0 & 0 & 0 & 0 & 0 & 0 & 0 & 0 \\
    0 & 0 & 0 & 0 & 0 & 0 & 0 & 0 & 0 & 0 & 0 & 0 & 0 & 0 & 0 & 0 & 0 & 0 & 0 \\
    0 & 0 & 0 & 0 & 0 & 0 & 0 & 0 & 0 & 0 & 0 & 0 & 0 & 0 & 0 & 0 & 0 & 0 & 0 \\
    0 & 0 & 0 & 0 & 0 & 0 & 0 & 0 & 0 & 0 & 0 & 0 & 0 & 0 & 0 & 0 & 0 & 0 & 0 \\
    0 & 0 & 0 & 0 & 0 & 0 & 0 & 0 & 0 & 0 & 0 & 0 & 0 & 0 & 0 & 0 & 0 & 0 & 0 \\
    0 & 0 & 0 & 0 & 0 & 0 & 0 & 0 & 0 & 0 & 0 & 0 & 0 & 0 & 0 & 0 & 0 & 0 & 0
  \end{bmatrix}
\]

\vspace{15pt}

\[
ODU3=
  \begin{bmatrix}
    0 & 0 & 0 & 0 & 0 & 0 & 0 & 0 & 0 & 0 & 0 & 0 & 0 & 0 & 0 & 0 & 0 & 0 & 0 \\
    0 & 0 & 0 & 0 & 0 & 0 & 0 & 0 & 0 & 0 & 0 & 0 & 0 & 0 & 0 & 0 & 0 & 0 & 0 \\
    0 & 0 & 0 & 0 & 0 & 0 & 0 & 0 & 0 & 0 & 0 & 0 & 0 & 0 & 0 & 0 & 0 & 0 & 0 \\
    0 & 0 & 0 & 0 & 0 & 0 & 0 & 0 & 0 & 0 & 0 & 0 & 0 & 0 & 0 & 0 & 0 & 0 & 0 \\
    0 & 0 & 0 & 0 & 0 & 0 & 0 & 0 & 0 & 0 & 0 & 0 & 0 & 0 & 0 & 0 & 0 & 0 & 0 \\
    0 & 0 & 0 & 0 & 0 & 0 & 0 & 0 & 0 & 0 & 0 & 0 & 0 & 0 & 0 & 0 & 0 & 0 & 0 \\
    0 & 0 & 0 & 0 & 0 & 0 & 0 & 0 & 0 & 0 & 0 & 0 & 0 & 0 & 0 & 0 & 0 & 0 & 1 \\
    0 & 0 & 0 & 0 & 0 & 0 & 0 & 0 & 0 & 0 & 0 & 0 & 0 & 0 & 0 & 0 & 0 & 0 & 1 \\
    0 & 0 & 0 & 0 & 0 & 0 & 0 & 0 & 0 & 0 & 0 & 0 & 0 & 0 & 0 & 0 & 0 & 0 & 1 \\
    0 & 0 & 0 & 0 & 0 & 0 & 0 & 0 & 0 & 0 & 0 & 0 & 0 & 0 & 0 & 0 & 0 & 0 & 1 \\
    0 & 0 & 0 & 0 & 0 & 0 & 0 & 0 & 0 & 0 & 0 & 0 & 0 & 0 & 0 & 0 & 0 & 0 & 1 \\
    0 & 0 & 0 & 0 & 0 & 0 & 0 & 0 & 0 & 0 & 0 & 0 & 0 & 0 & 0 & 0 & 0 & 0 & 1 \\
    0 & 0 & 0 & 0 & 0 & 0 & 0 & 0 & 0 & 0 & 0 & 0 & 0 & 0 & 0 & 0 & 0 & 0 & 1 \\
    0 & 0 & 0 & 0 & 0 & 0 & 0 & 0 & 0 & 0 & 0 & 0 & 0 & 0 & 0 & 0 & 0 & 0 & 1 \\
    0 & 0 & 0 & 0 & 0 & 0 & 0 & 0 & 0 & 0 & 0 & 0 & 0 & 0 & 0 & 0 & 0 & 0 & 1 \\
    0 & 0 & 0 & 0 & 0 & 0 & 0 & 0 & 0 & 0 & 0 & 0 & 0 & 0 & 0 & 0 & 0 & 0 & 1 \\
    0 & 0 & 0 & 0 & 0 & 0 & 0 & 0 & 0 & 0 & 0 & 0 & 0 & 0 & 0 & 0 & 0 & 0 & 1 \\
    0 & 0 & 0 & 0 & 0 & 0 & 0 & 0 & 0 & 0 & 0 & 0 & 0 & 0 & 0 & 0 & 0 & 0 & 1 \\
    0 & 0 & 0 & 0 & 0 & 0 & 1 & 1 & 1 & 1 & 1 & 1 & 1 & 1 & 1 & 1 & 1 & 1 & 0
  \end{bmatrix}
\]


\[
ODU4=
  \begin{bmatrix}
    0 & 1 & 0 & 0 & 0 & 0 & 0 & 0 & 0 & 0 & 0 & 0 & 0 & 0 & 0 & 0 & 0 & 0 & 0 \\
    1 & 0 & 0 & 0 & 0 & 0 & 0 & 0 & 0 & 0 & 0 & 0 & 0 & 0 & 0 & 0 & 0 & 0 & 0 \\
    0 & 0 & 0 & 1 & 0 & 0 & 0 & 0 & 0 & 0 & 0 & 0 & 0 & 0 & 0 & 0 & 0 & 0 & 0 \\
    0 & 0 & 1 & 0 & 0 & 0 & 0 & 0 & 0 & 0 & 0 & 0 & 0 & 0 & 0 & 0 & 0 & 0 & 0 \\
    0 & 0 & 0 & 0 & 0 & 0 & 0 & 1 & 0 & 0 & 0 & 0 & 0 & 0 & 0 & 0 & 0 & 0 & 0 \\
    0 & 0 & 0 & 0 & 0 & 0 & 1 & 0 & 0 & 0 & 0 & 0 & 0 & 0 & 0 & 0 & 0 & 0 & 0 \\
    0 & 0 & 0 & 0 & 0 & 1 & 0 & 0 & 0 & 0 & 0 & 0 & 0 & 0 & 0 & 0 & 0 & 0 & 0 \\
    0 & 0 & 0 & 0 & 1 & 0 & 0 & 0 & 0 & 0 & 0 & 0 & 0 & 0 & 0 & 0 & 0 & 0 & 0 \\
    0 & 0 & 0 & 0 & 0 & 0 & 0 & 0 & 0 & 1 & 0 & 0 & 0 & 0 & 0 & 0 & 0 & 0 & 0 \\
    0 & 0 & 0 & 0 & 0 & 0 & 0 & 0 & 1 & 0 & 0 & 0 & 0 & 0 & 0 & 0 & 0 & 0 & 0 \\
    0 & 0 & 0 & 0 & 0 & 0 & 0 & 0 & 0 & 0 & 0 & 0 & 0 & 0 & 0 & 0 & 0 & 0 & 0 \\
    0 & 0 & 0 & 0 & 0 & 0 & 0 & 0 & 0 & 0 & 0 & 0 & 1 & 0 & 0 & 0 & 0 & 0 & 0 \\
    0 & 0 & 0 & 0 & 0 & 0 & 0 & 0 & 0 & 0 & 0 & 1 & 0 & 0 & 0 & 0 & 0 & 0 & 0 \\
    0 & 0 & 0 & 0 & 0 & 0 & 0 & 0 & 0 & 0 & 0 & 0 & 0 & 0 & 0 & 0 & 0 & 0 & 0 \\
    0 & 0 & 0 & 0 & 0 & 0 & 0 & 0 & 0 & 0 & 0 & 0 & 0 & 0 & 0 & 1 & 0 & 0 & 0 \\
    0 & 0 & 0 & 0 & 0 & 0 & 0 & 0 & 0 & 0 & 0 & 0 & 0 & 0 & 1 & 0 & 0 & 0 & 0 \\
    0 & 0 & 0 & 0 & 0 & 0 & 0 & 0 & 0 & 0 & 0 & 0 & 0 & 0 & 0 & 0 & 0 & 0 & 0 \\
    0 & 0 & 0 & 0 & 0 & 0 & 0 & 0 & 0 & 0 & 0 & 0 & 0 & 0 & 0 & 0 & 0 & 0 & 1 \\
    0 & 0 & 0 & 0 & 0 & 0 & 0 & 0 & 0 & 0 & 0 & 0 & 0 & 0 & 0 & 0 & 0 & 1 & 0
  \end{bmatrix}
\]

\vspace{11pt}
In the traffic matrices each ODU, referred previously, has its respective value being that the ODU0 corresponds to 1.25 Gbits/s, ODU1 to 2.5 Gbits/s, ODU2 to 10 Gbits/s, ODU3 to 40 Gbits/s and finally the ODU4 corresponds to 100 Gbits/s.
As we can see these matrices are bidirectional because they are symmetric matrices and as such the traffic sent in one direction must be the same traffic sent in the opposite direction.

Through these ODU's we can calculate total network traffic for the low traffic scenario:\\

$T_1^0$ = 240x1.25 = 300 Gbits/s \qquad
$T_1^1$ = 200x2.5 = 500 Gbits/s \qquad
$T_1^2$ = 64x10 = 640 Gbits/s \\

$T_1^3$ = 24x40 = 960 Gbits/s \quad
$T_1^4$ = 16x100 = 1600 Gbits/s \\

$T_{1}$ = 300 + 500 + 640 + 960 + 1600 = 4000 Gbits/s \qquad
$T$ = 4000/2 = 2 Tbits/s\\

Where the variable $T_1^x$ represents the unidirectional traffic of the ODUx, for example, $T_1^3$ represents the unidirectional traffic of the ODU3 and $T_1^4$ represents the unidirectional traffic of the ODU4. The variable $T_{1}$ represents the total of unidirectional traffic that is injected into the network and finally the variable $T$ represents the total of bidirectional traffic.\\

Again, we can thus conclude that the total traffic for the two scenarios is as follows:
\begin{itemize}
  \item Low Traffic: \textbf{2 TBits/s}
  \item High Traffic: \textbf{20 TBits/s}
\end{itemize}


\subsection{Dimensioning using ILP}
\begin{tcolorbox}	
\begin{tabular}{p{2.75cm} p{0.2cm} p{10.5cm}} 	
\textbf{Student Name}  &:& Tiago Esteves    (October 03, 2017 - )\\
\end{tabular}
\end{tcolorbox}

In this section we will do the dimensioning of the network mentioned in the previous section to calculate the value of your CAPEX.
The initial subsection is the same as the subsection of the previous case so in this case it will be omitted presenting only the subsection of the results.\\

\textbf{Scenario 1: Realistic Network Low Traffic} \label{Scenario3_opaque} \\

This real network consists of many nodes and with many links between them as such the lpsolve takes immense time to get an optimal solution. Therefore, in this two cases, the execution time was defined as being two days (48 hours) and after that time presented the best solution. In this scenario we used the table \ref{table_real_net} and in the table \ref{result_ILP3_without} we can see the values calculated through MatLab and using the values indicated in table \ref{table_cost_opaque} we can finally calculate the CAPEX value.\\

Using equation \ref{linkCosts} : \\
$C_L$ = $($2 * 15 000 * 37$)$ + $($2 * 5 000 * 100 * 111 $)$ + $($24 * 4 000$)$ \\
$C_L$ = \textbf{112 206 000\euro} \\

Using equation \ref{electricalCostOpaque} : \\
$C_{exc}$ = $($19 * 10 000$)$ + 1 000 * $($4 000 + $($2 * 111 * 100$)$ $)$ \\
$C_N$ = $C_{exc}$ = \textbf{26 390 000 \euro} \\

$CAPEX$ = 112 206 000 + 26 390 000 = \textbf{138 596 000 \euro}\\

\begin{table}[h!]
\centering
\begin{tabular}{|| c | c||}
 \hline
 Number of optical channels & Value \\
 \hline\hline
in link (1,2) &  \\
in link (1,4) &  \\
in link (1,15) &  \\
in link (2,3) &  \\
in link (2,4) &  \\
in link (2,5) &  \\
in link (3,5) &  \\
in link (4,14) &  \\
in link (5,6) &  \\
in link (5,7) &  \\
in link (5,11) &  \\
in link (5,14) &  \\
in link (5,15) &  \\
in link (6,7) &  \\
in link (6,11) &  \\
in link (6,12) &  \\
in link (6,14) &  \\
in link (7,8) &  \\
in link (8,9) &  \\
in link (8,10) &  \\
in link (8,11) &  \\
in link (9,10) &  \\
in link (10,11) &  \\
in link (11,12) &  \\
in link (11,17) &  \\
in link (12,13) &  \\
in link (12,17) &  \\
in link (13,14) &  \\
in link (13,15) &  \\
in link (13,17) &  \\
in link (14,15) &  \\
in link (15,16) &  \\
in link (15,17) &  \\
in link (15,19) &  \\
in link (16,17) &  \\
in link (17,18) &  \\
in link (18,19) &  \\
\hline
\end{tabular}
\caption{Table with results}
\label{result_ILP3_without}
\end{table}

\textbf{Scenario 2: Realistic Network High Traffic} \label{Scenario4_opaque} \\
In this scenario we used again the table \ref{table_real_net} and in the table \ref{result_ILP4_without} we can see the values calculated through MatLab and using the values indicated in table \ref{table_cost_opaque} we can finally calculate the CAPEX value. \\

\begin{table}[h!]
\centering
\begin{tabular}{|| c | c||}
 \hline
 Number of optical channels & Value \\
 \hline\hline
 in link (1,2) &  \\
in link (1,4) &  \\
in link (1,15) &  \\
in link (2,3) &  \\
in link (2,4) &  \\
in link (2,5) &  \\
in link (3,5) &  \\
in link (4,14) &  \\
in link (5,6) &  \\
in link (5,7) &  \\
in link (5,11) &  \\
in link (5,14) &  \\
in link (5,15) &  \\
in link (6,7) &  \\
in link (6,11) &  \\
in link (6,12) &  \\
in link (6,14) &  \\
in link (7,8) &  \\
in link (8,9) &  \\
in link (8,10) &  \\
in link (8,11) &  \\
in link (9,10) &  \\
in link (10,11) &  \\
in link (11,12) &  \\
in link (11,17) &  \\
in link (12,13) &  \\
in link (12,17) &  \\
in link (13,14) &  \\
in link (13,15) &  \\
in link (13,17) &  \\
in link (14,15) &  \\
in link (15,16) &  \\
in link (15,17) &  \\
in link (15,19) &  \\
in link (16,17) &  \\
in link (17,18) &  \\
in link (18,19) &  \\
\hline
\end{tabular}
\caption{Table with results}
\label{result_ILP4_without}
\end{table}

Using equation \ref{linkCosts} : \\
$C_L$ = $($2 * 15 000 * 37$)$ + $($2 * 5 000 * 100 * $)$ + $($24 * 4 000$)$ \\
$C_L$ = \textbf{ \euro} \\

Using equation \ref{electricalCostOpaque} : \\
$C_{exc}$ = $($19 * 10 000$)$ + 1 000 * $($40 000 + $($2 * 100 * $)$ $)$ \\
$C_N$ = $C_{exc}$ = \textbf{ \euro} \\

$CAPEX$ =  +  = \textbf{ \euro}\\

\subsection{Dimensioning using Heuristics}

\subsubsection{Heuristics Results}

\subsection{Comparative Analysis}
