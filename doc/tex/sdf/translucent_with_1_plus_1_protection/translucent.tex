\clearpage

\begin{tcolorbox}	
\begin{tabular}{p{2.75cm} p{0.2cm} p{10.5cm}} 	
\textbf{Student Name}  &:& Tiago Esteves\\
\textbf{Starting Date} &:& October 03, 2017\\
\textbf{Goal}          &:& Implement the dimensioning of optical networks in the translucent transport mode.
\end{tabular}
\end{tcolorbox}

\section{Translucent with 1+1 Protection}
In this case study we focus on the translucent case with 1 + 1 protection.


\subsection{Physical Network Topology}

\subsubsection{Reference Network}
In the figure below we ca see that our reference network consists of 6 nodes and 8 Bidirectional links.
The average length of the links was chosen so that the following calculations are more simplistic, for this was created a matrix of distances between the respective nodes.
Finally, ODU's matrices were also created to be able to determine the total traffic used in each scenario.

\begin{figure}[h!]
\centering
\includegraphics[width=\textwidth]{RedeTeste}
\caption{Physical Topology of the Reference Network.}
\end{figure}

The distance matrix is the same for the two scenarios but the ODU's matrices are not.
In this way only the matrices for the case of low traffic are elucidated, being that in the case of a high traffic it is only necessary to multiply these matrices by the value 10.

\[
Dist=
  \begin{bmatrix}
    0 & 500 & 500 & 0 & 0 & 0 \\
    500 & 0 & 400 & 500 & 0 & 0 \\
    500 & 400 & 0 & 0 & 500 & 0 \\
    0 & 500 & 0 & 0 & 600 & 450 \\
    0 & 0 & 500 & 600 & 0 & 550 \\
    0 & 0 & 0 & 450 & 550 & 0
  \end{bmatrix}
\]

\[
ODU0=
  \begin{bmatrix}
    0 & 5 & 1 & 3 & 1 & 3 \\
    5 & 0 & 0 & 1 & 5 & 0 \\
    1 & 0 & 0 & 1 & 4 & 1 \\
    3 & 1 & 1 & 0 & 1 & 0 \\
    1 & 5 & 4 & 1 & 0 & 3 \\
    3 & 0 & 1 & 1 & 3 & 0
  \end{bmatrix}
\quad ODU1=
  \begin{bmatrix}
    0 & 2 & 4 & 2 & 0 & 5 \\
    2 & 0 & 0 & 3 & 1 & 1 \\
    4 & 0 & 0 & 1 & 1 & 0 \\
    3 & 3 & 1 & 0 & 1 & 3 \\
    0 & 1 & 1 & 1 & 0 & 1 \\
    5 & 1 & 0 & 3 & 1 & 0
  \end{bmatrix}
\quad ODU2=
  \begin{bmatrix}
    0 & 1 & 1 & 1 & 0 & 0 \\
    1 & 0 & 0 & 0 & 1 & 0 \\
    1 & 0 & 0 & 1 & 1 & 0 \\
    1 & 0 & 1 & 0 & 1 & 0 \\
    0 & 1 & 1 & 1 & 0 & 1 \\
    0 & 0 & 0 & 0 & 1 & 0
  \end{bmatrix}
\]
\[
ODU3=
  \begin{bmatrix}
    0 & 0 & 0 & 0 & 0 & 0 \\
    0 & 0 & 1 & 0 & 0 & 1 \\
    0 & 1 & 0 & 0 & 1 & 0 \\
    0 & 0 & 0 & 0 & 0 & 0 \\
    0 & 0 & 1 & 0 & 0 & 0 \\
    0 & 1 & 0 & 0 & 0 & 0
  \end{bmatrix}
\qquad ODU4=
  \begin{bmatrix}
    0 & 0 & 0 & 0 & 0 & 0 \\
    0 & 0 & 0 & 0 & 0 & 1 \\
    0 & 0 & 0 & 0 & 0 & 0 \\
    0 & 0 & 0 & 0 & 0 & 0 \\
    0 & 0 & 0 & 0 & 0 & 1 \\
    0 & 1 & 0 & 0 & 1 & 0
  \end{bmatrix}
\]

Through these ODU's we can calculate total network traffic for the low traffic scenario:\\
$T_1^0$ = 60x1.25 = 75 Gbits/s \qquad
$T_1^1$ = 50x2.5 = 125 Gbits/s \qquad
$T_1^2$ = 16x10 = 160 Gbits/s \\
$T_1^3$ = 6x40 = 240 Gbits/s \quad
$T_1^4$ = 4x100 = 400 Gbits/s \\
$T_{1total}$ = 75 + 125 + 160 + 240 + 400 = 1000 Gbits/s \qquad
$T_{total}$ = 1000/2 = 0.5 Tbits/s\\

We can thus conclude that the total traffic for the two scenarios is as follows:
\begin{itemize}
  \item Low Traffic: \textbf{0.5 TBits/s}
  \item High Traffic: \textbf{5 TBits/s}
\end{itemize}

Finally for this project has to take into consideration the table \ref{table:5} because in it we can see the values of the variables associated with this network.
\begin{table}[h!]
\centering
\begin{tabular}{|| c | c | c||}
 \hline
 Constant & Description & Value \\
 \hline\hline
 N & Number of nodes & 6 \\
 L & Number of bidirectional links & 8 \\
 <$\delta$> & Node out-degree & 2.667 \\
 <len> & Mean link length (km) & 500 \\
 <h> & Mean number of hops for working paths & 1.533 \\
 <h'> & Mean number of hops for backup paths & 2.467 \\
 \hline
\end{tabular}
\caption{Table of reference network values}
\label{table:5}
\end{table}

\subsubsection{Realistic Network}
The real network chosen for this work is the EON (European Optical Network).
The way the nodes are arranged geographically can be seen from the following figure and the matrix of distances created in the next page is constructed based on real distances.
In this case just ODU's matrices are created to be able to determine the total traffic used in each scenario.

\begin{figure}[h!]
\centering
\includegraphics[width=\textwidth]{EON_Rede_Realista}
\caption{Physical Topology of the Realistic Network.}
\end{figure}


The table \ref{table:6} shows the values of the variables associated with this network.
\begin{table}[h!]
\centering
\begin{tabular}{|| c | c | c||}
 \hline
 Constant & Description & Value \\
 \hline\hline
 N & Number of nodes & 19 \\
 L & Number of bidirectional links & 37 \\
 <$\delta$> & Node out-degree & 3.89 \\
 <len> & Mean link length (km) & 753.76 \\
 <h> & Mean number of hops for working paths & 2.3 \\
 <h'> & Mean number of hops for backup paths & 3.2 \\
 \hline
\end{tabular}
\caption{Table of realistic network values}
\label{table:6}
\end{table}

Again, through the ODU's we can calculate the total traffic for both scenarios being them:
\begin{itemize}
  \item Low Traffic: \textbf{2 TBits/s}
  \item High Traffic: \textbf{20 TBits/s}
\end{itemize}


\[
  \text{Dist} = \kbordermatrix{
    & Oslo & Stockholm & Moscow & Copenhagen & Berlin & Prague & Vienna & Zagreb & Athens & Rome & Milan & Zurich & Brussels & Amesterdan & London & Dublin & Paris & Madrid & Lisbon \\
    Oslo & 0 & 500 & 500 & 0 & 0 & 0 & 0 & 500 & 500 & 0 & 0 & 500 & 500 & 0 & 0 & 0 & 0 & 500 & 500 \\
    Stockholm & 0 & 500 & 500 & 0 & 0 & 0 & 0 & 500 & 500 & 0 & 0 & 500 & 500 & 0 & 0 & 0 & 0 & 500 & 500 \\
    Moscow & 0 & 500 & 500 & 0 & 0 & 0 & 0 & 500 & 500 & 0 & 0 & 500 & 500 & 0 & 0 & 0 & 0 & 500 & 500 \\
    Copenhagen & 0 & 500 & 500 & 0 & 0 & 0 & 0 & 500 & 500 & 0 & 0 & 500 & 500 & 0 & 0 & 0 & 0 & 500 & 500 \\
    Berlin & 0 & 500 & 500 & 0 & 0 & 0 & 0 & 500 & 500 & 0 & 0 & 500 & 500 & 0 & 0 & 0 & 0 & 500 & 500 \\
    Prague & 0 & 500 & 500 & 0 & 0 & 0 & 0 & 500 & 500 & 0 & 0 & 500 & 500 & 0 & 0 & 0 & 0 & 500 & 500 \\
    Vienna & 0 & 500 & 500 & 0 & 0 & 0 & 0 & 500 & 500 & 0 & 0 & 500 & 500 & 0 & 0 & 0 & 0 & 500 & 500 \\
    Zagreb & 0 & 500 & 500 & 0 & 0 & 0 & 0 & 500 & 500 & 0 & 0 & 500 & 500 & 0 & 0 & 0 & 0 & 500 & 500 \\
    Athens & 0 & 500 & 500 & 0 & 0 & 0 & 0 & 500 & 500 & 0 & 0 & 500 & 500 & 0 & 0 & 0 & 0 & 500 & 500 \\
    Rome & 0 & 500 & 500 & 0 & 0 & 0 & 0 & 500 & 500 & 0 & 0 & 500 & 500 & 0 & 0 & 0 & 0 & 500 & 500 \\
    Milan & 0 & 500 & 500 & 0 & 0 & 0 & 0 & 500 & 500 & 0 & 0 & 500 & 500 & 0 & 0 & 0 & 0 & 500 & 500 \\
    Zurich & 0 & 500 & 500 & 0 & 0 & 0 & 0 & 500 & 500 & 0 & 0 & 500 & 500 & 0 & 0 & 0 & 0 & 500 & 500 \\
    Brussels & 0 & 500 & 500 & 0 & 0 & 0 & 0 & 500 & 500 & 0 & 0 & 500 & 500 & 0 & 0 & 0 & 0 & 500 & 500 \\
    Amesterdan & 0 & 500 & 500 & 0 & 0 & 0 & 0 & 500 & 500 & 0 & 0 & 500 & 500 & 0 & 0 & 0 & 0 & 500 & 500 \\
    London & 0 & 500 & 500 & 0 & 0 & 0 & 0 & 500 & 500 & 0 & 0 & 500 & 500 & 0 & 0 & 0 & 0 & 500 & 500 \\
    Dublin & 0 & 500 & 500 & 0 & 0 & 0 & 0 & 500 & 500 & 0 & 0 & 500 & 500 & 0 & 0 & 0 & 0 & 500 & 500 \\
    Paris & 0 & 500 & 500 & 0 & 0 & 0 & 0 & 500 & 500 & 0 & 0 & 500 & 500 & 0 & 0 & 0 & 0 & 500 & 500\\
    Madrid & 0 & 500 & 500 & 0 & 0 & 0 & 0 & 500 & 500 & 0 & 0 & 500 & 500 & 0 & 0 & 0 & 0 & 500 & 500 \\
    Lisbon & 0 & 500 & 500 & 0 & 0 & 0 & 0 & 500 & 500 & 0 & 0 & 500 & 500 & 0 & 0 & 0 & 0 & 500 & 444
  }
\]


\newpage
\subsection{Dimensioning using ILP}

\subsubsection{ILP models} \label{ILP_models_Transluc}

Again, for a better understanding of the functions and variables used in the ILP, a table \ref{description_transluc} will be created with all the variables and their description. \\

\begin{table}[h!]
\centering
\begin{tabular}{ |p{1cm}||p{13cm}|}
 \hline
 \multicolumn{2}{|c|}{Description of notation used in the objective function} \\
 \hline
 \hline
 $i$ & index for start node of a physical link \\
 $j$ & index for end node of a physical link \\
 $o$ & index for node that is origin of a demand \\
 $d$ & index for node that is destination of a demand \\
 $($ i,j $)$ & physical link between the nodes $i$ and $j$ \\
 $($ o,d $)$ & demand between the nodes $o$ and $d$ \\
 $c$ & Client traffic Type $($ 1 to 5 $)$ \\
 $L_{ij}^{od}$ & Number of ODU-o low speed signals from node $o$ to node $d$ employing lightpath ($i$,$j$) \\
 $f_{ij}^{od}$ & Number of 100 Gbit/s optical channels (number of flows) between the link $i$ and $j$ for all demand pairs between $o$ and $d$ \\
 $W_{od}$ & Number of lightpath channels between the nodes $o$ and $d$ \\
 $B$ & Client signals granularities $($1.25, 2.5, 10, 40, 100$)$ \\
 $D_{od}$ & Client traffic demands between the nodes $o$ and $d$ \\
 G & Network topology in form of adjacency matrix \\
 $BD$ & Bandwidth \\
 \hline
\end{tabular}
\caption{Table with description of variables}
\label{description_transluc}
\end{table}

\vspace{20pt}

\begin{equation}
minimize \qquad \qquad \qquad \qquad \qquad  \sum_{(o,d)} W_{od}
\label{ILPTransluc}
\end{equation}

$subject$ $to$

\begin{equation}
\sum_{j \textbackslash \{o\} } L_{ij}^{od} = D_{odc}
\qquad \qquad \qquad \qquad \qquad \qquad \qquad \qquad \qquad \qquad \qquad
\forall (o,d):o<d
\label{ILPTransluc1}
\end{equation}

\begin{equation}
\sum_{j \textbackslash \{o\} } L_{ji}^{od} = \sum_{j \textbackslash \{d\} } L_{ji}^{od}
\qquad \qquad \qquad \qquad \qquad \qquad \qquad
\forall (o,d):o<d , \forall i : i \neq o,d
\label{ILPTransluc2}
\end{equation}

\begin{equation}
\sum_{j \textbackslash \{d\} } L_{ij}^{od} = D_{odc}
\qquad \qquad \qquad \qquad \qquad \qquad \qquad \qquad \qquad \qquad \qquad
\forall (o,d):o<d
\label{ILPTransluc3}
\end{equation}

\begin{equation}
\sum_{(o,d):o<d} \left( B(c) \times L_{ij}^{od}\right) \leq  \sum BD \times W_{ij}
\qquad \qquad \qquad \qquad \qquad
\forall (i,j)
\label{ILPTransluc4}
\end{equation}

\begin{equation}
L_{ij}^{od} \geq 0;
\qquad \qquad \qquad \qquad \qquad \qquad \qquad \qquad \qquad \qquad
\forall (i,j) , \forall (o,d) : o < d
\label{ILPTransluc5}
\end{equation}

\begin{equation}
\sum_{j \textbackslash \{o\} } f_{ji}^{od} = W_{od}
\qquad \qquad \qquad \qquad \qquad \qquad \qquad \qquad \qquad \qquad \qquad
\forall (o,d):o<d
\label{ILPTransluc6}
\end{equation}

\begin{equation}
\sum_{j \textbackslash \{o\} } f_{ji}^{od} = \sum_{j \textbackslash \{d\} } f_{ji}^{od}
\qquad \qquad \qquad \qquad \qquad \qquad \qquad
\forall (o,d):o<d , \forall i : i \neq o,d
\label{ILPTransluc7}
\end{equation}

\begin{equation}
\sum_{j \textbackslash \{d\} } f_{ji}^{od} = W_{od}
\qquad \qquad \qquad \qquad \qquad \qquad \qquad \qquad \qquad \qquad \qquad
\forall (o,d):o<d
\label{ILPTransluc8}
\end{equation}

\begin{equation}
\sum_{(o,d):o<d} \left( f_{ij}^{od} + f_{ji}^{od}\right) \leq 80 G_{ij}
\qquad \qquad \qquad \qquad \qquad \qquad \qquad
\forall (i,j) : i < j
\label{ILPTransluc9}
\end{equation}

\begin{equation}
f_{ij}^{od} \geq 0
\qquad \qquad \qquad \qquad \qquad \qquad \qquad \qquad \qquad \qquad \qquad \qquad
\forall (i,j) \forall (o,d)
\label{ILPTransluc10}
\end{equation}	


\subsubsection{ILP Results}

In this initial phase the results will be presented using ILP to calculate the CAPEX of the reference network.

The value of the CAPEX of the network will be calculated based on the costs of the equipment present in the table below.
\begin{table}[h!]
\centering
\begin{tabular}{|| c | c||}
 \hline
 Equipment & Cost \\
 \hline\hline
 OLT without transponders & 15000 \euro \\
 Transponder & 5000 \euro/Gb \\
 Optical Amplifier & 4000 \euro \\
 EXC & 10000 \euro \\
 OXC & 20000 \euro \\
 EXC Port & 1000 \euro /Gb/s\\
 OXC Port & 2500 \euro /porto \\
 \hline
\end{tabular}
\caption{Table with costs}
\label{table_cost3}
\end{table}

In addition to the equipment costs, we will also use the parameter "span", which in this case will have a value of 100.
Because this value is used to calculate the number of optical amplifiers required in the network using Equation \ref{amplifiersTranslu}.

\begin{equation}
N^R = \sum\limits_{l=1}^L\left(\left\lceil\frac{len_l}{span}\right\rceil-1\right)
\label{amplifiersTranslu}
\end{equation} \\

To know the value of CAPEX it is necessary to know the value of the cost of the links and the cost of the nodes.

To calculate the cost of the nodes, the sum of the costs of the optical and electrical node is made. %For this case the optical cost is given by equation \ref{opticalCost} and the electrical cost by the equation \ref{electricalCostTransp}.

%\begin{equation}
%C_{oxc} = \left(\gamma_{o0} \times N \right) + \gamma_{o1} \times  \left(P_{LINE} + P_{ADD}\right)
%\label{opticalCost}
%\end{equation}	
	
%\begin{itemize}
%\item{$C_{oxc}$		$\rightarrow$	Optical Ports Cost}
%\item{$\gamma_{o0}$	$\rightarrow$	OXC cost in Euros}
%\item{$\gamma_{o1}$	$\rightarrow$	OXC port cost in Euros}
%\item{$P_{TRIB}	$	$\rightarrow$	Number of tributary ports}
%\item{$P_{ADD} $	$\rightarrow$	Number of adding ports}
%\end{itemize}

%\begin{equation}
%C_{exc} = \left(\gamma_{e0}\times N\right) + \gamma_{e1} \times \left(2 \times T_1 \right)		\label{electricalCostTransp}
%\end{equation} \\


To calculate the cost of the Links we will use the equation .

%\begin{equation}
%C_L = \left(\gamma_0^{OLT} \times L\right) + \left(\gamma_1^{OLT} \times \tau \times W\right) + \left(N^R \times %c^R\right)
%\label{linkCostsTransp}
%\end{equation} \\
	

Finally we will calculate the CAPEX values for the various situations mentioned.\\

\textbf{Low Traffic scenario:}\\

$C_L$ = \textbf{\euro}

$C_N$ = \textbf{\euro}

$CAPEX$ = \textbf{\euro}\\

\textbf{High Traffic scenario:}\\

$C_L$ = \textbf{\euro}

$C_N$ = \textbf{ \euro}

$CAPEX$ =  \textbf{ \euro}\\

\subsection{Dimensioning using Analytical solution}

\subsubsection{Analytical Models}

\subsubsection{Analytical Results}

\newpage

\subsection{Dimensioning using Heuristics}

\subsubsection{Heuristics Models}

\subsubsection{Heuristics Results}

\subsection{Comparative Analysis}
