\clearpage

\subsection{Opaque with 1+1 Protection}\label{ILP_Opaque_Protection}
\begin{tcolorbox}	
\begin{tabular}{p{2.75cm} p{0.2cm} p{10.5cm}} 	
\textbf{Student Name}  &:& Tiago Esteves    (October 03, 2017 - )\\
\textbf{Goal}          &:& Implement the ILP model for the opaque transport mode with 1 plus 1 protection.
\end{tabular}
\end{tcolorbox}
\vspace{11pt}

Here, in this case, we must take into account table \ref{description_opaque}, previously mentioned, in order to better understand the objective function.\\

Before carrying out the description of the objective function we must take into account the following particularity of this mode of transport:
\begin{itemize}
  \item $N_{OXC,n}$ = 0, \quad $\forall$ n
  \item $N_{EXC,n}$ = 1, \quad $\forall$ n that process traffic
\end{itemize}


\vspace{11pt}
The objective function of following the ILP is a minimization of the CAPEX through the equation \ref{Capex} where in this case for the cost of nodes we only have in consideration the electric cost \ref{Capex_Node_EXC} because of the particularity previously mentioned.
In this case the value of $P_{exc,c,n}$ is obtained by equation \ref{EXC_pexc1_opaquep} for long-reach and by the equation \ref{EXC_pexc2_opaquep} for short-reach.\\

As previously mentioned, equation \ref{EXC_pexc1_opaquep} refers to the number of long-reach ports of the electrical switch with bit-rate -1 in node n, $P_{exc,-1,n}$, i.e. the number of line ports of node n which can be calculated as

\begin{equation}
P_{exc,-1,n} = \sum_{j=1}^{N} w_{nj}
\label{EXC_pexc1_opaquep}
\end{equation}

\vspace{11pt}
where $w_{nj}$ is the number of optical channels between node $n$ and node $j$.

\vspace{11pt}
As previously mentioned, equation \ref{EXC_pexc2_opaque} refers to the number of sort-reach ports of the electrical switch with bit-rate c in node n, $P_{exc,c,n}$, i.e. the number of tributary ports with bit-rate c in node n which can be calculated as

\begin{equation}
P_{exc,c,n} = \sum_{d=1}^{N} D_{nd,c}
\label{EXC_pexc2_opaquep}
\end{equation}

\vspace{11pt}
where $D_{nd,c}$ are the client demands between nodes $n$ and $d$ with bit rate $c$.

\vspace{11pt}
In this case there is the following particularity:

\begin{itemize}
  \item When $n$=$d$ the value of client demands is always zero, i.e, $D_{nn,c}=0$
\end{itemize}


\vspace{17pt}
The objective function, to be minimized, is the expression \ref{ILPOpaque_CAPEX}.\\

$subject$ $to$
\begin{equation}
\sum_{j\textbackslash \{o\}} f_{ij}^{od} = 2  \qquad \qquad \qquad \qquad \qquad \qquad \qquad \qquad \qquad \qquad
\forall(o,d) : o < d, \forall i: i = o
\label{ILPOpaque1}
\end{equation}

This constraint are equal to the constraint \ref{ILPOpaque1_CAPEX} assuming that Z variable has the value of 2 (work and protection).

\begin{equation}
\sum_{j\textbackslash \{o\}} f_{ij}^{od} = \sum_{j\textbackslash \{d\}} f_{ji}^{od}   \qquad \qquad \qquad \qquad \qquad \qquad \qquad \qquad
\forall(o,d) : o < d, \forall i: i \neq o,d
\label{ILPOpaque2}
\end{equation}

This constraint are equal to the constraint \ref{ILPOpaque2_CAPEX}.

\begin{equation}
\sum_{j\textbackslash \{d\}} f_{ji}^{od} = 2  \qquad \qquad \qquad \qquad \qquad \qquad \qquad \qquad \qquad \qquad
\forall(o,d) : o < d, \forall i: i = d
\label{ILPOpaque3}
\end{equation}

This constraint are equal to the constraint \ref{ILPOpaque3_CAPEX} assuming that Z variable has the value of 2 (work and protection).

\begin{equation}
\sum_{o=1} \sum_{d=o+1} \left(f_{ij}^{od} + f_{ji}^{od}\right) \sum_{c\in C} (B\left(c\right) D_{odc}\leq \tau W_{ij} G_{ij} \qquad \qquad \qquad \qquad
\forall(i,j) : i < j
\label{ILPOpaque4}
\end{equation}

This restriction is considered grooming constraint, so it means the total client traffic flows can not be greater than the capacity of optical transmission system on all links where $\tau$ is always 100.

\begin{equation}
W_{ij} \leq K_{ij} L_{ij} \qquad \qquad \qquad \qquad \qquad \qquad \qquad \qquad \qquad \qquad \qquad \qquad \forall(i,j) : i < j
\label{ILPOpaque5}
\end{equation}

This restriction concerns the capacity of the optical channels which must be less or equal to the maximum number of optical channels. For any situation the maximum number of optical channels supported by each transmission system is 100, i.e., $K_{ij}$ = 100.

\begin{equation}
L_{ij} , f_{ij}^{od} , f_{ji}^{od} \in \{0,1\} \qquad \qquad \qquad \qquad \qquad \qquad \qquad
\forall(i,j) : i < j, \forall(o,d) : o < d
\label{ILPOpaque6}
\end{equation}

The number of flows per demand in this case can be zero if there are no traffic demands or one if considering working or protection traffic, in relation to the use of the link, can be zero if it is not being used or one if is being used.

\begin{equation}
W_{ij} \in \mathbb{N}  \qquad \qquad \qquad \qquad \qquad \qquad \qquad \qquad \qquad \qquad \qquad \qquad \qquad
\forall(i,j) : i < j\label{ILPOpaque7}
\end{equation}

The last constraint is just needed to ensure the number optical of channels is a positive integer values greater than zero.\\


\subsubsection{Result description}

To perform the calculations using the implementation of the models described in previous subsection it is necessary to use the MATLAB once more. \\
We already have all the necessary to obtain the CAPEX value for the reference network \ref{Reference_Network_Topology}. As described in the subsection of network traffic \ref{Reference_Network_Traffic}, we have three values of network traffic (low, medium and high traffic) so we have to obtain three different CAPEX.
The value of the CAPEX of the network will be calculated based on the costs of the equipment present in the table \ref{table_cost_ilp}.\\

\vspace{15pt}
\textbf{Low Traffic Scenario:}\\

In this scenario we have to take into account the traffic calculated in \ref{low_scenario}. In a first phase we will show the various existing topologies of the network. The first are the allowed topologies, physical and optical topology, the second are the logical topology for all ODUs and finally the resulting physical and optical topology.\\

\vspace{11pt}
\begin{figure}[h!]
\centering
\includegraphics[width=12cm]{sdf/ilp/opaque_protection/figures/allowed_physical_topology}
\caption{Allowed physical topology. The allowed physical topology is defined by the duct and sites in the field. It is assumed that each duct supports up to 1 bidirectional transmission system and each site supports up to 1 node.}
\label{allowed_physical_protectionlow}
\end{figure}
\newpage
\begin{figure}[h!]
\centering
\includegraphics[width=11cm]{sdf/ilp/opaque_protection/figures/allowed_optical_topology}
\caption{Allowed optical topology. The allowed optical topology is defined by the transport mode (opaque transport mode in this case). It is assumed that each transmission system supports up to 100 optical channels.}
\label{allowed_optical_protectionlow}
\end{figure}

\begin{figure}[h!]
\centering
\includegraphics[width=11cm]{sdf/ilp/opaque_protection/figures/logical_topology_ODU0_low}
\caption{ODU0 logical topology defined by the ODU0 traffic matrix.}
\label{logical_ODU0_protectionlow}
\end{figure}

\begin{figure}[h!]
\centering
\includegraphics[width=11cm]{sdf/ilp/opaque_protection/figures/logical_topology_ODU1_low}
\caption{ODU1 logical topology defined by the ODU1 traffic matrix.}
\label{logical_ODU1_protectionlow}
\end{figure}
\newpage
\begin{figure}[h!]
\centering
\includegraphics[width=12cm]{sdf/ilp/opaque_protection/figures/logical_topology_ODU2_low}
\caption{ODU2 logical topology defined by the ODU2 traffic matrix.}
\label{logical_ODU2_protectionlow}
\end{figure}

\begin{figure}[h!]
\centering
\includegraphics[width=12cm]{sdf/ilp/opaque_protection/figures/logical_topology_ODU3_low}
\caption{ODU3 logical topology defined by the ODU3 traffic matrix.}
\label{logical_ODU3_protectionlow}
\end{figure}

\begin{figure}[h!]
\centering
\includegraphics[width=12cm]{sdf/ilp/opaque_protection/figures/logical_topology_ODU4_low}
\caption{ODU4 logical topology defined by the ODU4 traffic matrix.}
\label{logical_ODU4_protectionlow}
\end{figure}

\begin{figure}[h!]
\centering
\includegraphics[width=13cm]{sdf/ilp/opaque_protection/figures/physical_topology}
\caption{Physical topology after dimensioning.}
\label{physical_protectionlow}
\end{figure}

\vspace{17pt}

\begin{figure}[h!]
\centering
\includegraphics[width=13cm]{sdf/ilp/opaque_protection/figures/optical_topology_low}
\caption{Optical topology after dimensioning.}
\label{optical_protectionlow}
\end{figure}

\vspace{13pt}
In table \ref{link_opaque_protec_ref_low} we can see the number of optical channels calculated using \ref{Capex_Link} and \ref{ILPOpaque_CAPEX} and the number of amplifiers for each link calculated using \ref{amplifiers}.\\

In table \ref{node_opaque_protec_ref_low} we can see the resulting nodal degree at the physical layer, calculated based on the number of connections that the node in question performs, the number of line ports calculated using \ref{EXC_pexc1_opaquep} and the number of tributary ports calculated using \ref{EXC_pexc2_opaquep} for each node.\\
\newpage
\begin{table}[h!]
\centering
\begin{tabular}{|| c | c | c ||}
 \hline
 \multicolumn{3}{|| c ||}{Information regarding links} \\
 \hline
 \hline
 Bidirectional Link & Optical Channels & Amplifiers\\
 \hline
 Node 1 <-> Node 2 & 2 & 4 \\
 Node 1 <-> Node 3 & 2 & 6 \\
 Node 2 <-> Node 3 & 3 & 0 \\
 Node 2 <-> Node 4 & 3 & 6 \\
 Node 3 <-> Node 5 & 3 & 8 \\
 Node 4 <-> Node 5 & 3 & 1 \\
 Node 4 <-> Node 6 & 3 & 7 \\
 Node 5 <-> Node 6 & 3 & 3 \\
 \hline
\end{tabular}
\caption{Table with information regarding links}
\label{link_opaque_protec_ref_low}
\end{table}


\begin{table}[h!]
\centering
\begin{tabular}{|| c | c | c | c ||}
 \hline
 \multicolumn{4}{|| c ||}{Information regarding nodes} \\
 \hline
 \hline
 Node & Resulting Nodal Degree & Line Ports & Tributary Ports\\
 \hline
 1 & 2 & 4 & 29 \\
 2 & 3 & 8 & 23 \\
 3 & 3 & 8 & 18 \\
 4 & 3 & 9 & 20 \\
 5 & 3 & 9 & 24 \\
 6 & 2 & 6 & 22 \\
\hline
\end{tabular}
\caption{Table with information regarding nodes}
\label{node_opaque_protec_ref_low}
\end{table}

\vspace{13pt}
Through the information obtained previously on the nodes we can now create tables with detailed information about each node. In each table mentioned below we can see how many ports are connected to a given node and its bit rate (in relation to the line ports) and how many ports are assigned to each different bit rate (in relation to the tributary ports).\\

\begin{table}[h!]
\centering
\begin{tabular}{|| c | c | c ||}
 \hline
 \multicolumn{3}{|| c ||}{Detailed description of Node 1} \\
 \hline
 \hline
  & Node<--Optical Channels-->Node & Bit rate \\
 \hline
 \multirow{2}{*}{4 line ports} & 1  <---- 2 ---->  2 & \multirow{2}{*}{100 Gbits/s} \\
 & 1  <---- 2 ---->  3 & \\
 \hline
 \hline
  & Number of tributary ports & Bit rate \\ \hline
\multirow{3}{*}{29 tributary ports} & 13 & ODU0 \\
 & 13 & ODU1 \\
 & 3 & ODU2 \\
\hline
\end{tabular}
\caption{Table with detailed description of node 1}
\end{table}
\newpage
\begin{table}[h!]
\centering
\begin{tabular}{|| c | c | c ||}
 \hline
 \multicolumn{3}{|| c ||}{Detailed description of Node 2} \\
 \hline
 \hline
  & Node<--Optical Channels-->Node & Bit rate \\
 \hline
 \multirow{3}{*}{8 line ports} & 2  <---- 2 ---->  1 & \multirow{3}{*}{100 Gbits/s} \\
 & 2  <---- 3 ---->  3 & \\
 & 2  <---- 3 ---->  4 & \\
 \hline
 \hline
  & Number of tributary ports & Bit rate \\ \hline
\multirow{5}{*}{23 tributary ports} & 11 & ODU0 \\
 & 7 & ODU1 \\
 & 2 & ODU2 \\
 & 2 & ODU3 \\
 & 1 & ODU4 \\
\hline
\end{tabular}
\caption{Table with detailed description of node 2}
\end{table}

\vspace{15pt}
\begin{table}[h!]
\centering
\begin{tabular}{|| c | c | c ||}
 \hline
 \multicolumn{3}{|| c ||}{Detailed description of Node 3} \\
 \hline
 \hline
   & Node<--Optical Channels-->Node & Bit rate \\
 \hline
 \multirow{3}{*}{8 line ports} & 3  <---- 2 ---->  1 & \multirow{3}{*}{100 Gbits/s}\\
 & 3  <---- 3 ---->  2 & \\
 & 3  <---- 3 ---->  5 & \\
 \hline
 \hline
  & Number of tributary ports & Bit rate \\ \hline
\multirow{4}{*}{18 tributary ports} & 7 & ODU0 \\
 & 6 & ODU1\\
 & 3 & ODU2\\
 & 2 & ODU3\\
\hline
\end{tabular}
\caption{Table with detailed description of node 3}
\end{table}

\vspace{15pt}
\begin{table}[h!]
\centering
\begin{tabular}{|| c | c | c ||}
 \hline
 \multicolumn{3}{|| c ||}{Detailed description of Node 4} \\
 \hline
 \hline
   & Node<--Optical Channels-->Node & Bit rate \\
 \hline
 \multirow{3}{*}{9 line ports} & 4  <---- 3 ---->  2 & \multirow{3}{*}{100 Gbits/s}\\
 & 4  <---- 3 ---->  5 & \\
 & 4  <---- 3 ---->  6 & \\
 \hline
 \hline
  & Number of tributary ports & Bit rate \\ \hline
\multirow{3}{*}{20 tributary ports} & 7 & ODU0 \\
 & 10 & ODU1 \\
 & 3 & ODU2 \\
\hline
\end{tabular}
\caption{Table with detailed description of node 4}
\end{table}

\newpage
\begin{table}[h!]
\centering
\begin{tabular}{|| c | c | c ||}
 \hline
 \multicolumn{3}{|| c ||}{Detailed description of Node 5} \\
 \hline
 \hline
   & Node<--Optical Channels-->Node & Bit rate \\
 \hline
 \multirow{3}{*}{9 line ports} & 5  <---- 3 ---->  2 & \multirow{3}{*}{100 Gbits/s}\\
 & 5  <---- 3 ---->  4 & \\
 & 5  <---- 3 ---->  6 & \\
 \hline
 \hline
  & Number of tributary ports & Bit rate \\ \hline
\multirow{5}{*}{24 tributary ports} & 14 & ODU0 \\
 & 4 & ODU1 \\
 & 4 & ODU2 \\
 & 1 & ODU3 \\
 & 1 & ODU4 \\
\hline
\end{tabular}
\caption{Table with detailed description of node 5}
\end{table}

\vspace{17pt}
\begin{table}[h!]
\centering
\begin{tabular}{|| c | c | c ||}
 \hline
 \multicolumn{3}{|| c ||}{Detailed description of Node 6} \\
 \hline
 \hline
   & Node<--Optical Channels-->Node & Bit rate \\
 \hline
 \multirow{2}{*}{6 line ports} & 6  <---- 3 ---->  4 & \multirow{2}{*}{100 Gbits/s}\\
 & 6  <---- 3 ---->  5 & \\
 \hline
 \hline
  & Number of tributary ports & Bit rate \\ \hline
\multirow{5}{*}{22 tributary ports} & 8 & ODU0 \\
 & 10 & ODU1 \\
 & 1 & ODU2 \\
 & 1 & ODU3 \\
 & 2 & ODU4 \\
\hline
\end{tabular}
\caption{Table with detailed description of node 6}
\end{table}

\vspace{17pt}
Now in next page, let's focus on the routing information. These paths are bidirectional so the path from one node to another is the same path in the opposite direction. In table \ref{path_opaque_protec_ref_low} we can see all the routing obtained for all nodes. In the Links column we can see that there are two paths but it is not possible to distinguish them because we do not know which is protection and which is working.\\
\newpage
\begin{table}[h!]
\centering
\begin{tabular}{|| c | c | c | c | c | c | c | c ||}
 \hline
 \multicolumn{8}{|| c ||}{Routing} \\
 \hline
 \hline
 o & d & Links & ODU0 & ODU1 & ODU2 & ODU3 & ODU4\\
 \hline
 \multirow{2}{*}{1} & \multirow{2}{*}{2} & \{(1,2)\} & \multirow{2}{*}{5} & \multirow{2}{*}{2} & \multirow{2}{*}{1} & \multirow{2}{*}{0} & \multirow{2}{*}{0} \\
 & & \{(1,3),(3,2)\} & & & & & \\ \hline
 \multirow{2}{*}{1} & \multirow{2}{*}{3} & \{(1,3)\} & \multirow{2}{*}{1} & \multirow{2}{*}{4} & \multirow{2}{*}{1} & \multirow{2}{*}{0} & \multirow{2}{*}{0}\\
 & & \{(1,2),(2,3)\} & & & & &\\ \hline
 \multirow{2}{*}{1} & \multirow{2}{*}{4} & \{(1,2),(2,4)\} & \multirow{2}{*}{3} & \multirow{2}{*}{2} & \multirow{2}{*}{1} & \multirow{2}{*}{0} & \multirow{2}{*}{0}\\
 & & \{(1,3),(3,5),(5,4)\} & & & & &\\ \hline
 \multirow{2}{*}{1} & \multirow{2}{*}{5} & \{(1,3),(3,5)\} & \multirow{2}{*}{1} & \multirow{2}{*}{0} & \multirow{2}{*}{0} & \multirow{2}{*}{0} & \multirow{2}{*}{0}\\
 & & \{(1,2),(2,4),(4,5)\} & & & & &\\ \hline
 \multirow{2}{*}{1} & \multirow{2}{*}{6} & \{(1,2),(2,4),(4,6)\} & \multirow{2}{*}{3} & \multirow{2}{*}{5} & \multirow{2}{*}{0} & \multirow{2}{*}{0} & \multirow{2}{*}{0}\\
 & & \{(1,3),(3,5),(5,6)\} & & & & &\\ \hline
 \multirow{2}{*}{2} & \multirow{2}{*}{3} & \{(2,3)\} & \multirow{2}{*}{0} & \multirow{2}{*}{0} & \multirow{2}{*}{0} & \multirow{2}{*}{1} & \multirow{2}{*}{0}\\
 & & \{(2,1),(1,3)\} & & & & &\\ \hline
 \multirow{2}{*}{2} & \multirow{2}{*}{4} & \{(2,4)\} & \multirow{2}{*}{1} & \multirow{2}{*}{3} & \multirow{2}{*}{0} & \multirow{2}{*}{0} & \multirow{2}{*}{0}\\
 & & \{(2,3),(3,5),(5,4)\} & & & & &\\ \hline
 \multirow{2}{*}{2} & \multirow{2}{*}{5} & \{(2,3),(3,5)\} & \multirow{2}{*}{5} & \multirow{2}{*}{1} & \multirow{2}{*}{1} & \multirow{2}{*}{0} & \multirow{2}{*}{0}\\
 & & \{(2,4),(4,5)\} & & & & &\\ \hline
 \multirow{2}{*}{2} & \multirow{2}{*}{6} & \{(2,4),(4,6)\} & \multirow{2}{*}{0} & \multirow{2}{*}{1} & \multirow{2}{*}{0} & \multirow{2}{*}{1} & \multirow{2}{*}{1}\\
 & & \{(2,3),(3,5),(5,6)\} & & & & &\\ \hline
 \multirow{2}{*}{3} & \multirow{2}{*}{4} & \{(3,2),(2,4)\} & \multirow{2}{*}{1} & \multirow{2}{*}{1} & \multirow{2}{*}{1} & \multirow{2}{*}{0} & \multirow{2}{*}{0}\\
 & & \{(3,5),(5,4)\} & & & & &\\ \hline
 \multirow{2}{*}{3} & \multirow{2}{*}{5} & \{(3,5)\} & \multirow{2}{*}{4} & \multirow{2}{*}{1} & \multirow{2}{*}{1} & \multirow{2}{*}{1} & \multirow{2}{*}{0}\\
 & & \{(3,1),(1,2),(2,4),(4,5)\} & & & & &\\ \hline
 \multirow{2}{*}{3} & \multirow{2}{*}{6} & \{(3,5),(5,6)\} & \multirow{2}{*}{1} & \multirow{2}{*}{0} & \multirow{2}{*}{0} & \multirow{2}{*}{0} & \multirow{2}{*}{0}\\
 & & \{(3,2),(2,4),(4,6)\} & & & & &\\ \hline
 \multirow{2}{*}{4} & \multirow{2}{*}{5} & \{(4,5)\} & \multirow{2}{*}{1} & \multirow{2}{*}{1} & \multirow{2}{*}{1} & \multirow{2}{*}{0} & \multirow{2}{*}{0}\\
 & & \{(4,6),(6,5)\} & & & & &\\ \hline
 \multirow{2}{*}{4} & \multirow{2}{*}{6} & \{(4,6)\} & \multirow{2}{*}{1} & \multirow{2}{*}{3} & \multirow{2}{*}{0} & \multirow{2}{*}{0} & \multirow{2}{*}{0}\\
 & & \{(4,5),(5,6)\} & & & & &\\ \hline
 \multirow{2}{*}{5} & \multirow{2}{*}{6} & \{(5,6)\} & \multirow{2}{*}{3} & \multirow{2}{*}{1} & \multirow{2}{*}{1} & \multirow{2}{*}{0} & \multirow{2}{*}{1}\\
 & & \{(5,4),(4,6)\} & & & & &\\
 \hline
\end{tabular}
\caption{Table with description of routing. We are assuming that between a pair of nodes all demands follow the same route.}
\label{path_opaque_protec_ref_low}
\end{table}


\vspace{15pt}
Finally and most importantly through table \ref{scriptopaque_protec_ref_low} we can see the CAPEX result for this model. This value is obtained using equation \ref{ILPOpaque_CAPEX} and all of the constraints mentioned above.
In table \ref{formulas_opaque} mentioned in first model we can see how all the values were calculated.\\
\newpage
\begin{table}[h!]
\centering
\begin{tabular}{|| c | c | c | c | c | c | c ||}
 \hline
 \multicolumn{7}{|| c ||}{CAPEX of the Network} \\
 \hline
 \hline
 \multicolumn{3}{|| c |}{ } & Quantity & Unit Price & Cost & Total \\
 \hline
 \multirow{3}{*}{Link Cost} & \multicolumn{2}{ c |}{OLTs} & 16 & 15 000 \euro & 240 000 \euro & \multirow{3}{*}{22 520 000 \euro} \\ \cline{2-6}
 & \multicolumn{2}{ c |}{100 Gbits/s Transceivers} & 44 & 5 000 \euro/Gbit/s & 22 000 000 \euro & \\ \cline{2-6}
 & \multicolumn{2}{ c |}{Amplifiers} & 70 & 4 000 \euro & 280 000 \euro & \\
 \hline
 \multirow{9}{*}{Node Cost} & \multirow{7}{*}{Electrical} & EXCs & 6 & 10 000 \euro & 60 000 \euro & \multirow{9}{*}{4 462 590 \euro} \\ \cline{3-6}
 & & ODU0 Ports & 60 & 10 \euro/port & 600 \euro & \\ \cline{3-6}
 & & ODU1 Ports & 50 & 15 \euro/port & 750 \euro & \\ \cline{3-6}
 & & ODU2 Ports & 16 & 30 \euro/port & 480 \euro & \\ \cline{3-6}
 & & ODU3 Ports & 6 & 60 \euro/port & 360 \euro & \\ \cline{3-6}
 & & ODU4 Ports & 4 & 100 \euro/port & 400 \euro & \\ \cline{3-6}
 & & Line Ports & 44 & 100 000 \euro/port & 4 400 000 \euro & \\ \cline{2-6}
 & \multirow{2}{*}{Optical} & OXCs & 0 & 20 000 \euro & 0 \euro & \\ \cline{3-6}
 & & Ports & 0 & 2 500 \euro/porto & 0 \euro & \\
 \hline
 \multicolumn{6}{|| c |}{Total Network Cost} & 26 982 590 \euro \\
\hline
\end{tabular}
\caption{Table with detailed description of CAPEX}
\label{scriptopaque_protec_ref_low}
\end{table}


\vspace{15pt}
\textbf{Medium Traffic Scenario:}\\

In this scenario we have to take into account the traffic calculated in \ref{medium_traffic_scenario}. In a first phase we will show the various existing topologies of the network. The first are the allowed topologies, physical and optical topology, the second are the logical topology for all ODUs and finally the resulting physical and optical topology.\\

\begin{figure}[h!]
\centering
\includegraphics[width=12cm]{sdf/ilp/opaque_protection/figures/allowed_physical_topology}
\caption{Allowed physical topology. The allowed physical topology is defined by the duct and sites in the field. It is assumed that each duct supports up to 1 bidirectional transmission system and each site supports up to 1 node.}
\label{allowed_physical_protectionmedium}
\end{figure}

\newpage
\begin{figure}[h!]
\centering
\includegraphics[width=11cm]{sdf/ilp/opaque_protection/figures/allowed_optical_topology}
\caption{Allowed optical topology. The allowed optical topology is defined by the transport mode (opaque transport mode in this case). It is assumed that each transmission system supports up to 100 optical channels.}
\label{allowed_optical_protectionmedium}
\end{figure}

\begin{figure}[h!]
\centering
\includegraphics[width=11cm]{sdf/ilp/opaque_protection/figures/logical_topology_ODU0_medium}
\caption{ODU0 logical topology defined by the ODU0 traffic matrix.}
\label{logical_ODU0_protectionmedium}
\end{figure}

\begin{figure}[h!]
\centering
\includegraphics[width=11cm]{sdf/ilp/opaque_protection/figures/logical_topology_ODU1_medium}
\caption{ODU1 logical topology defined by the ODU1 traffic matrix.}
\label{logical_ODU1_protectionmedium}
\end{figure}
\newpage
\begin{figure}[h!]
\centering
\includegraphics[width=12cm]{sdf/ilp/opaque_protection/figures/logical_topology_ODU2_medium}
\caption{ODU2 logical topology defined by the ODU2 traffic matrix.}
\label{logical_ODU2_protectionmedium}
\end{figure}

\begin{figure}[h!]
\centering
\includegraphics[width=12cm]{sdf/ilp/opaque_protection/figures/logical_topology_ODU3_medium}
\caption{ODU3 logical topology defined by the ODU3 traffic matrix.}
\label{logical_ODU3_protectionmedium}
\end{figure}

\begin{figure}[h!]
\centering
\includegraphics[width=12cm]{sdf/ilp/opaque_protection/figures/logical_topology_ODU4_medium}
\caption{ODU4 logical topology defined by the ODU4 traffic matrix.}
\label{logical_ODU4_protectionmedium}
\end{figure}
\newpage
\begin{figure}[h!]
\centering
\includegraphics[width=12cm]{sdf/ilp/opaque_protection/figures/physical_topology}
\caption{Physical topology after dimensioning.}
\label{physical_protectionmedium}
\end{figure}

\vspace{15pt}
\begin{figure}[h!]
\centering
\includegraphics[width=12cm]{sdf/ilp/opaque_protection/figures/optical_topology_medium}
\caption{Optical topology after dimensioning.}
\label{optical_protectionmedium}
\end{figure}

\vspace{15pt}
In table \ref{link_opaque_protec_ref_medium} we can see the number of optical channels calculated using \ref{Capex_Link} and \ref{ILPOpaque_CAPEX} and the number of amplifiers for each link calculated using \ref{Capex_amplifiers}. In the case where there are no optical channels we assume that the number of amplifiers is zero.\\

In table \ref{node_opaque_protec_ref_medium} we can see the resulting nodal degree at the physical layer, calculated based on the number of connections that the node in question performs, the number of line ports calculated using \ref{EXC_pexc1_opaquep} and the number of tributary ports calculated using \ref{EXC_pexc2_opaquep} for each node.\\

\newpage
\begin{table}[h!]
\centering
\begin{tabular}{|| c | c | c ||}
 \hline
 \multicolumn{3}{|| c ||}{Information regarding links} \\
 \hline
 \hline
 Bidirectional Link & Optical Channels & Amplifiers\\
 \hline
 Node 1 <-> Node 2 & 12 & 4 \\
 Node 1 <-> Node 3 & 12 & 6 \\
 Node 2 <-> Node 3 & 33 & 0 \\
 Node 2 <-> Node 4 & 28 & 6 \\
 Node 3 <-> Node 5 & 28 & 8 \\
 Node 4 <-> Node 5 & 26 & 1 \\
 Node 4 <-> Node 6 & 30 & 7 \\
 Node 5 <-> Node 6 & 30 & 3 \\
 \hline
\end{tabular}
\caption{Table with information regarding links}
\label{link_opaque_protec_ref_medium}
\end{table}

\begin{table}[h!]
\centering
\begin{tabular}{|| c | c | c | c ||}
 \hline
 \multicolumn{4}{|| c ||}{Information regarding nodes} \\
 \hline
 \hline
 Node & Connections & Line Ports & Tributary Ports\\
 \hline
 1 & 2 & 24 & 290 \\
 2 & 3 & 73 & 230 \\
 3 & 3 & 73 & 180 \\
 4 & 3 & 84 & 200 \\
 5 & 3 & 84 & 240 \\
 6 & 2 & 60 & 220 \\
\hline
\end{tabular}
\caption{Table with information regarding nodes}
\label{node_opaque_protec_ref_medium}
\end{table}

Through the information obtained previously on the nodes we can now create tables with detailed information about each node.
In each table mentioned below we can see how many ports are connected to a given node and its bit rate (in relation to the line ports) and how many ports are assigned to each different bit rate (in relation to the tributary ports).\\

\begin{table}[h!]
\centering
\begin{tabular}{|| c | c | c ||}
 \hline
 \multicolumn{3}{|| c ||}{Detailed description of Node 1} \\
 \hline
 \hline
  & Node <-- Optical Channels --> Node & bit rate \\ \hline
\multirow{2}{*}{24 line ports} & 1  <---- 12 ---->  2 & \multirow{2}{*}{100 Gbtis/s} \\
 & 1  <---- 12 ---->  3 & \\
 \hline
 \hline
  & Number of tributary ports & bit rate \\ \hline
\multirow{3}{*}{290 tributary ports} & 130 & ODU0 \\
 & 130 & ODU1 \\
 & 30 & ODU2 \\
\hline
\end{tabular}
\caption{Table with detailed description of node 1}
\end{table}

\newpage
\begin{table}[h!]
\centering
\begin{tabular}{|| c | c | c ||}
 \hline
 \multicolumn{3}{|| c ||}{Detailed description of Node 2} \\
 \hline
 \hline
  & Node <-- Optical Channels --> Node & bit rate \\ \hline
 \multirow{3}{*}{73 line ports} & 2  <---- 12 ---->  1 & \multirow{3}{*}{100 Gbtis/s}\\
 & 2  <---- 33 ---->  3 & \\
 & 2  <---- 28 ---->  4 & \\
 \hline
 \hline
  & Number of tributary ports & bit rate \\ \hline
\multirow{5}{*}{230 tributary ports} & 110 & ODU0 \\
 & 70 & ODU1 \\
 & 20 & ODU2 \\
 & 20 & ODU3 \\
 & 10 & ODU4 \\
\hline
\end{tabular}
\caption{Table with detailed description of node 2}
\end{table}

\vspace{13pt}
\begin{table}[h!]
\centering
\begin{tabular}{|| c | c | c ||}
 \hline
 \multicolumn{3}{|| c ||}{Detailed description of Node 3} \\
 \hline
 \hline
  & Node <-- Optical Channels --> Node & bit rate \\ \hline
 \multirow{3}{*}{73 line ports} & 3  <---- 12 ---->  1 & \multirow{3}{*}{100 Gbtis/s}\\
 & 3  <---- 33 ---->  2 & \\
 & 3  <---- 28 ---->  5 & \\
 \hline
 \hline
  & Number of tributary ports & bit rate \\ \hline
\multirow{4}{*}{180 tributary ports} & 70 & ODU0 \\
 & 60 & ODU1\\
 & 30 & ODU2\\
 & 20 & ODU3\\
\hline
\end{tabular}
\caption{Table with detailed description of node 3}
\end{table}

\vspace{13pt}
\begin{table}[h!]
\centering
\begin{tabular}{|| c | c | c ||}
 \hline
 \multicolumn{3}{|| c ||}{Detailed description of Node 4} \\
 \hline
 \hline
  & Node <-- Optical Channels --> Node & bit rate \\ \hline
\multirow{3}{*}{84 line ports} & 4  <---- 28 ---->  2 & \multirow{3}{*}{100 Gbtis/s}\\
 & 4  <---- 26 ---->  5 & \\
 & 4  <---- 30 ---->  6 & \\
 \hline
 \hline
  & Number of tributary ports & bit rate \\ \hline
\multirow{3}{*}{200 tributary ports} & 70 & ODU0 \\
 & 100 & ODU1 \\
 & 30 & ODU2 \\
\hline
\end{tabular}
\caption{Table with detailed description of node 4}
\end{table}

\newpage
\begin{table}[h!]
\centering
\begin{tabular}{|| c | c | c ||}
 \hline
 \multicolumn{3}{|| c ||}{Detailed description of Node 5} \\
 \hline
 \hline
  & Node <-- Optical Channels --> Node & bit rate \\ \hline
 \multirow{3}{*}{84 line ports} & 5  <---- 28 ---->  3 & \multirow{3}{*}{100 Gbtis/s} \\
 & 5  <---- 26 ---->  4 & \\
 & 5  <---- 30 ---->  6 & \\
 \hline
 \hline
  & Number of tributary ports & bit rate \\ \hline
\multirow{5}{*}{240 tributary ports} & 140 & ODU0 \\
 & 40 & ODU1 \\
 & 40 & ODU2 \\
 & 10 & ODU3 \\
 & 10 & ODU4 \\
\hline
\end{tabular}
\caption{Table with detailed description of node 5}
\end{table}

\vspace{17pt}
\begin{table}[h!]
\centering
\begin{tabular}{|| c | c | c ||}
 \hline
 \multicolumn{3}{|| c ||}{Detailed description of Node 6} \\
 \hline
 \hline
  & Node <-- Optical Channels --> Node & bit rate \\ \hline
 \multirow{2}{*}{60 line ports} & 6  <---- 30 ---->  4 & \multirow{2}{*}{100 Gbtis/s} \\
 & 6  <---- 30 ---->  5 & \\
 \hline
 \hline
  & Number of tributary ports & bit rate \\ \hline
\multirow{5}{*}{220 tributary ports} & 80 & ODU0 \\
 & 100 & ODU1 \\
 & 10 & ODU2 \\
 & 10 & ODU3 \\
 & 20 & ODU4 \\
\hline
\end{tabular}
\caption{Table with detailed description of node 6}
\end{table}

\vspace{17pt}
Now in next page, let's focus on the routing information. These paths are bidirectional so the path from one node to another is the same path in the opposite direction. In table \ref{path_opaque_protec_ref_medium} we can see all the routing obtained for all nodes. In the Links column we can see that there are two paths but it is not possible to distinguish them because we do not know which is protection and which is working.\\
\newpage
\begin{table}[h!]
\centering
\begin{tabular}{|| c | c | c | c | c | c | c | c ||}
 \hline
 \multicolumn{8}{|| c ||}{Routing} \\
 \hline
 \hline
 o & d & Links & ODU0 & ODU1 & ODU2 & ODU3 & ODU4\\
 \hline
 \multirow{2}{*}{1} & \multirow{2}{*}{2} & \{(1,2)\} & \multirow{2}{*}{5} & \multirow{2}{*}{2} & \multirow{2}{*}{1} & \multirow{2}{*}{0} & \multirow{2}{*}{0} \\
 & & \{(1,3),(3,2)\} & & & & & \\ \hline
 \multirow{2}{*}{1} & \multirow{2}{*}{3} & \{(1,3)\} & \multirow{2}{*}{1} & \multirow{2}{*}{4} & \multirow{2}{*}{1} & \multirow{2}{*}{0} & \multirow{2}{*}{0}\\
 & & \{(1,2),(2,3)\} & & & & &\\ \hline
 \multirow{2}{*}{1} & \multirow{2}{*}{4} & \{(1,2),(2,4)\} & \multirow{2}{*}{3} & \multirow{2}{*}{2} & \multirow{2}{*}{1} & \multirow{2}{*}{0} & \multirow{2}{*}{0}\\
 & & \{(1,3),(3,5),(5,4)\} & & & & &\\ \hline
 \multirow{2}{*}{1} & \multirow{2}{*}{5} & \{(1,3),(3,5)\} & \multirow{2}{*}{1} & \multirow{2}{*}{0} & \multirow{2}{*}{0} & \multirow{2}{*}{0} & \multirow{2}{*}{0}\\
 & & \{(1,2),(2,4),(4,5)\} & & & & &\\ \hline
 \multirow{2}{*}{1} & \multirow{2}{*}{6} & \{(1,2),(2,4),(4,6)\} & \multirow{2}{*}{3} & \multirow{2}{*}{5} & \multirow{2}{*}{0} & \multirow{2}{*}{0} & \multirow{2}{*}{0}\\
 & & \{(1,3),(3,5),(5,6)\} & & & & &\\ \hline
 \multirow{2}{*}{2} & \multirow{2}{*}{3} & \{(2,3)\} & \multirow{2}{*}{0} & \multirow{2}{*}{0} & \multirow{2}{*}{0} & \multirow{2}{*}{1} & \multirow{2}{*}{0}\\
 & & \{(2,1),(1,3)\} & & & & &\\ \hline
 \multirow{2}{*}{2} & \multirow{2}{*}{4} & \{(2,4)\} & \multirow{2}{*}{1} & \multirow{2}{*}{3} & \multirow{2}{*}{0} & \multirow{2}{*}{0} & \multirow{2}{*}{0}\\
 & & \{(2,3),(3,5),(5,4)\} & & & & &\\ \hline
 \multirow{2}{*}{2} & \multirow{2}{*}{5} & \{(2,4),(4,5)\} & \multirow{2}{*}{5} & \multirow{2}{*}{1} & \multirow{2}{*}{1} & \multirow{2}{*}{0} & \multirow{2}{*}{0}\\
 & & \{(2,3),(3,5)\} & & & & &\\ \hline
 \multirow{2}{*}{2} & \multirow{2}{*}{6} & \{(2,4),(4,6)\} & \multirow{2}{*}{0} & \multirow{2}{*}{1} & \multirow{2}{*}{0} & \multirow{2}{*}{1} & \multirow{2}{*}{1}\\
 & & \{(2,3),(3,5),(5,6)\} & & & & &\\ \hline
 \multirow{2}{*}{3} & \multirow{2}{*}{4} & \{(3,2),(2,4)\} & \multirow{2}{*}{1} & \multirow{2}{*}{1} & \multirow{2}{*}{1} & \multirow{2}{*}{0} & \multirow{2}{*}{0}\\
 & & \{(3,5),(5,4)\} & & & & &\\ \hline
 \multirow{2}{*}{3} & \multirow{2}{*}{5} & \{(3,5)\} & \multirow{2}{*}{4} & \multirow{2}{*}{1} & \multirow{2}{*}{1} & \multirow{2}{*}{1} & \multirow{2}{*}{0}\\
 & & \{(3,2),(2,4),(4,5)\} & & & & &\\ \hline
 \multirow{2}{*}{3} & \multirow{2}{*}{6} & \{(3,5),(5,6)\} & \multirow{2}{*}{1} & \multirow{2}{*}{0} & \multirow{2}{*}{0} & \multirow{2}{*}{0} & \multirow{2}{*}{0}\\
 & & \{(3,2),(2,4),(4,6)\} & & & & &\\ \hline
 \multirow{2}{*}{4} & \multirow{2}{*}{5} & \{(4,5)\} & \multirow{2}{*}{1} & \multirow{2}{*}{1} & \multirow{2}{*}{1} & \multirow{2}{*}{0} & \multirow{2}{*}{0}\\
 & & \{(4,6),(6,5)\} & & & & &\\ \hline
 \multirow{2}{*}{4} & \multirow{2}{*}{6} & \{(4,6)\} & \multirow{2}{*}{1} & \multirow{2}{*}{3} & \multirow{2}{*}{0} & \multirow{2}{*}{0} & \multirow{2}{*}{0}\\
 & & \{(4,5),(5,6)\} & & & & &\\ \hline
 \multirow{2}{*}{5} & \multirow{2}{*}{6} & \{(5,6)\} & \multirow{2}{*}{3} & \multirow{2}{*}{1} & \multirow{2}{*}{1} & \multirow{2}{*}{0} & \multirow{2}{*}{1}\\
 & & \{(5,4),(4,6)\} & & & & &\\
 \hline
\end{tabular}
\caption{Table with description of routing. We are assuming that between a pair of nodes all demands follow the same route.}
\label{path_opaque_protec_ref_medium}
\end{table}


\vspace{15pt}
Finally and most importantly through table \ref{scriptopaque_protec_ref_medium} we can see the CAPEX result for this model. This value is obtained using equation \ref{ILPOpaque_CAPEX} and all of the constraints mentioned above. In table \ref{formulas_opaque} mentioned in first model we can see how all the values were calculated.\\
\newpage
\begin{table}[h!]
\centering
\begin{tabular}{|| c | c | c | c | c | c | c ||}
 \hline
 \multicolumn{7}{|| c ||}{CAPEX of the Network} \\
 \hline
 \hline
 \multicolumn{3}{|| c |}{ } & Quantity & Unit Price & Cost & Total \\
 \hline
 \multirow{3}{*}{Link Cost} & \multicolumn{2}{ c |}{OLTs} & 16 & 15 000 \euro & 240 000 \euro & \multirow{3}{*}{199 520 000 \euro} \\ \cline{2-6}
 & \multicolumn{2}{ c |}{100 Gbits/s Transceivers} & 398 & 5 000 \euro/Gbit/s & 199 000 000 \euro & \\ \cline{2-6}
 & \multicolumn{2}{ c |}{Amplifiers} & 70 & 4 000 \euro & 280 000 \euro & \\
 \hline
 \multirow{9}{*}{Node Cost} & \multirow{7}{*}{Electrical} & EXCs & 6 & 10 000 \euro & 60 000 \euro & \multirow{9}{*}{39 885 900 \euro} \\ \cline{3-6}
 & & ODU0 Ports & 600 & 10 \euro/port & 6 000 \euro & \\ \cline{3-6}
 & & ODU1 Ports & 500 & 15 \euro/port & 7 500 \euro & \\ \cline{3-6}
 & & ODU2 Ports & 160 & 30 \euro/port & 4 800 \euro & \\ \cline{3-6}
 & & ODU3 Ports & 60 & 60 \euro/port & 3 600 \euro & \\ \cline{3-6}
 & & ODU4 Ports & 40 & 100 \euro/port & 4 000 \euro & \\ \cline{3-6}
 & & Line Ports & 398 & 100 000 \euro/port & 39 800 000 \euro & \\ \cline{2-6}
 & \multirow{2}{*}{Optical} & OXCs & 0 & 20 000 \euro & 0 \euro & \\ \cline{3-6}
 & & Ports & 0 & 2 500 \euro/porto & 0 \euro & \\
 \hline
 \multicolumn{6}{|| c |}{Total Network Cost} & 239 405 900 \euro \\
\hline
\end{tabular}
\caption{Table with detailed description of CAPEX}
\label{scriptopaque_protec_ref_medium}
\end{table}


\vspace{15pt}
\textbf{High Traffic Scenario:}\\

In this scenario we have to take into account the traffic calculated in \ref{high_traffic_scenario}. In a first phase we will show the various existing topologies of the network. The first are the allowed topologies, physical and optical topology, the second are the logical topology for all ODUs and finally the resulting physical and optical topology.\\

\begin{figure}[h!]
\centering
\includegraphics[width=12cm]{sdf/ilp/opaque_protection/figures/allowed_physical_topology}
\caption{Allowed physical topology. The allowed physical topology is defined by the duct and sites in the field. It is assumed that each duct supports up to 1 bidirectional transmission system and each site supports up to 1 node.}
\label{allowed_physical_protectionhigh}
\end{figure}
\newpage
\begin{figure}[h!]
\centering
\includegraphics[width=11cm]{sdf/ilp/opaque_protection/figures/allowed_optical_topology}
\caption{Allowed optical topology. The allowed optical topology is defined by the transport mode (opaque transport mode in this case). It is assumed that each transmission system supports up to 100 optical channels.}
\label{allowed_optical_protectionhigh}
\end{figure}

\begin{figure}[h!]
\centering
\includegraphics[width=11cm]{sdf/ilp/opaque_protection/figures/logical_topology_ODU0_high}
\caption{ODU0 logical topology defined by the ODU0 traffic matrix.}
\label{logical_ODU0_protectionhigh}
\end{figure}

\begin{figure}[h!]
\centering
\includegraphics[width=11cm]{sdf/ilp/opaque_protection/figures/logical_topology_ODU1_high}
\caption{ODU1 logical topology defined by the ODU1 traffic matrix.}
\label{logical_ODU1_protectionhigh}
\end{figure}
\newpage
\begin{figure}[h!]
\centering
\includegraphics[width=12cm]{sdf/ilp/opaque_protection/figures/logical_topology_ODU2_high}
\caption{ODU2 logical topology defined by the ODU2 traffic matrix.}
\label{logical_ODU2_protectionhigh}
\end{figure}

\begin{figure}[h!]
\centering
\includegraphics[width=12cm]{sdf/ilp/opaque_protection/figures/logical_topology_ODU3_high}
\caption{ODU3 logical topology defined by the ODU3 traffic matrix.}
\label{logical_ODU3_protectionhigh}
\end{figure}

\begin{figure}[h!]
\centering
\includegraphics[width=12cm]{sdf/ilp/opaque_protection/figures/logical_topology_ODU4_high}
\caption{ODU4 logical topology defined by the ODU4 traffic matrix.}
\label{logical_ODU4_protectionhigh}
\end{figure}
\newpage
\begin{figure}[h!]
\centering
\includegraphics[width=13cm]{sdf/ilp/opaque_protection/figures/physical_topology}
\caption{Physical topology after dimensioning.}
\label{physical_protectionhigh}
\end{figure}

\vspace{25pt}
\begin{figure}[h!]
\centering
\includegraphics[width=13cm]{sdf/ilp/opaque_protection/figures/optical_topology_high}
\caption{Optical topology after dimensioning.}
\label{optical_protectionhigh}
\end{figure}

\vspace{15pt}
In table \ref{link_opaque_protec_ref_high} we can see the number of optical channels calculated using \ref{Capex_Link} and \ref{Capex} and the number of amplifiers for each link calculated using \ref{amplifiers}.\\

In table \ref{node_opaque_protec_ref_high} we can see the resulting nodal degree at the physical layer, calculated based on the number of connections that the node in question performs, the number of line ports calculated using \ref{EXC_pexc1_opaquep} and the number of tributary ports calculated using \ref{EXC_pexc2_opaquep} for each node.\\

\newpage
\begin{table}[h!]
\centering
\begin{tabular}{|| c | c | c ||}
 \hline
 \multicolumn{3}{|| c ||}{Information regarding links} \\
 \hline
 \hline
 Bidirectional Link & Optical Channels & Amplifiers\\
 \hline
 Node 1 <-> Node 2 & 24 & 4 \\
 Node 1 <-> Node 3 & 24 & 6 \\
 Node 2 <-> Node 3 & 65 & 0 \\
 Node 2 <-> Node 4 & 56 & 6 \\
 Node 3 <-> Node 5 & 56 & 8 \\
 Node 4 <-> Node 5 & 52 & 1 \\
 Node 4 <-> Node 6 & 60 & 7 \\
 Node 5 <-> Node 6 & 60 & 3 \\
 \hline
\end{tabular}
\caption{Table with information regarding links}
\label{link_opaque_protec_ref_high}
\end{table}


\begin{table}[h!]
\centering
\begin{tabular}{|| c | c | c | c ||}
 \hline
 \multicolumn{4}{|| c ||}{Information regarding nodes} \\
 \hline
 \hline
 Node & Resulting Nodal Degree & Line Ports & Tributary Ports\\
 \hline
 1 & 2 & 48 & 580 \\
 2 & 3 & 145 & 460 \\
 3 & 3 & 145 & 360 \\
 4 & 3 & 168 & 400 \\
 5 & 3 & 168 & 480 \\
 6 & 2 & 120 & 440 \\
\hline
\end{tabular}
\caption{Table with information regarding nodes}
\label{node_opaque_protec_ref_high}
\end{table}


Through the information obtained previously on the nodes we can now create tables with detailed information about each node. In each table mentioned below we can see how many ports are connected to a given node and its bit rate (in relation to the line ports) and how many ports are assigned to each different bit rate (in relation to the tributary ports).\\

\begin{table}[h!]
\centering
\begin{tabular}{|| c | c | c ||}
 \hline
 \multicolumn{3}{|| c ||}{Detailed description of Node 1} \\
 \hline
 \hline
  & Node <-- Optical Channels --> Node & bit rate \\ \hline
\multirow{2}{*}{48 line ports} & 1  <---- 24 ---->  2 & \multirow{2}{*}{100 Gbtis/s} \\
 & 1  <---- 24 ---->  3 & \\
 \hline
 \hline
  & Number of tributary ports & bit rate \\ \hline
\multirow{3}{*}{580 tributary ports} & 260 & ODU0 \\
 & 260 & ODU1 \\
 & 60 & ODU2 \\
\hline
\end{tabular}
\caption{Table with detailed description of node 1}
\end{table}

\newpage
\begin{table}[h!]
\centering
\begin{tabular}{|| c | c | c ||}
 \hline
 \multicolumn{3}{|| c ||}{Detailed description of Node 2} \\
 \hline
 \hline
  & Node <-- Optical Channels --> Node & bit rate \\ \hline
 \multirow{3}{*}{145 line ports} & 2  <---- 24 ---->  1 & \multirow{3}{*}{100 Gbtis/s}\\
 & 2  <---- 65 ---->  3 & \\
 & 2  <---- 56 ---->  4 & \\
 \hline
 \hline
  & Number of tributary ports & bit rate \\ \hline
\multirow{5}{*}{460 tributary ports} & 220 & ODU0 \\
 & 140 & ODU1 \\
 & 40 & ODU2 \\
 & 40 & ODU3 \\
 & 20 & ODU4 \\
\hline
\end{tabular}
\caption{Table with detailed description of node 2}
\end{table}

\vspace{15pt}
\begin{table}[h!]
\centering
\begin{tabular}{|| c | c | c ||}
 \hline
 \multicolumn{3}{|| c ||}{Detailed description of Node 3} \\
 \hline
 \hline
  & Node <-- Optical Channels --> Node & bit rate \\ \hline
 \multirow{3}{*}{145 line ports} & 3  <---- 24 ---->  1 & \multirow{3}{*}{100 Gbtis/s}\\
 & 3  <---- 65 ---->  2 & \\
 & 3  <---- 56 ---->  5 & \\
 \hline
 \hline
  & Number of tributary ports & bit rate \\ \hline
\multirow{4}{*}{360 tributary ports} & 140 & ODU0 \\
 & 120 & ODU1\\
 & 60 & ODU2\\
 & 40 & ODU3\\
\hline
\end{tabular}
\caption{Table with detailed description of node 3}
\end{table}

\vspace{15pt}
\begin{table}[h!]
\centering
\begin{tabular}{|| c | c | c ||}
 \hline
 \multicolumn{3}{|| c ||}{Detailed description of Node 4} \\
 \hline
 \hline
  & Node <-- Optical Channels --> Node & bit rate \\ \hline
\multirow{3}{*}{168 line ports} & 4  <---- 56 ---->  2 & \multirow{3}{*}{100 Gbtis/s}\\
 & 4  <---- 52 ---->  5 & \\
 & 4  <---- 60 ---->  6 & \\
 \hline
 \hline
  & Number of tributary ports & bit rate \\ \hline
\multirow{3}{*}{400 tributary ports} & 140 & ODU0 \\
 & 200 & ODU1 \\
 & 60 & ODU2 \\
\hline
\end{tabular}
\caption{Table with detailed description of node 4}
\end{table}

\newpage
\begin{table}[h!]
\centering
\begin{tabular}{|| c | c | c ||}
 \hline
 \multicolumn{3}{|| c ||}{Detailed description of Node 5} \\
 \hline
 \hline
  & Node <-- Optical Channels --> Node & bit rate \\ \hline
 \multirow{3}{*}{168 line ports} & 5  <---- 56 ---->  3 & \multirow{3}{*}{100 Gbtis/s} \\
 & 5  <---- 52 ---->  4 & \\
 & 5  <---- 60 ---->  6 & \\
 \hline
 \hline
  & Number of tributary ports & bit rate \\ \hline
\multirow{5}{*}{480 tributary ports} & 280 & ODU0 \\
 & 80 & ODU1 \\
 & 80 & ODU2 \\
 & 20 & ODU3 \\
 & 20 & ODU4 \\
\hline
\end{tabular}
\caption{Table with detailed description of node 5}
\end{table}

\vspace{15pt}
\begin{table}[h!]
\centering
\begin{tabular}{|| c | c | c ||}
 \hline
 \multicolumn{3}{|| c ||}{Detailed description of Node 6} \\
 \hline
 \hline
  & Node <-- Optical Channels --> Node & bit rate \\ \hline
 \multirow{2}{*}{120 line ports} & 6  <---- 60 ---->  4 & \multirow{2}{*}{100 Gbtis/s} \\
 & 6  <---- 60 ---->  5 & \\
 \hline
 \hline
  & Number of tributary ports & bit rate \\ \hline
\multirow{5}{*}{440 tributary ports} & 160 & ODU0 \\
 & 200 & ODU1 \\
 & 20 & ODU2 \\
 & 20 & ODU3 \\
 & 40 & ODU4 \\
\hline
\end{tabular}
\caption{Table with detailed description of node 6}
\end{table}


\vspace{17pt}
Now in next page, let's focus on the routing information. These paths are bidirectional so the path from one node to another is the same path in the opposite direction. In table \ref{path_opaque_protec_ref_high} we can see all the routing obtained for all nodes. In the Links column we can see that there are two paths but it is not possible to distinguish them because we do not know which is protection and which is working.\\
\newpage
\begin{table}[h!]
\centering
\begin{tabular}{|| c | c | c | c | c | c | c | c ||}
 \hline
 \multicolumn{8}{|| c ||}{Routing} \\
 \hline
 \hline
 o & d & Links & ODU0 & ODU1 & ODU2 & ODU3 & ODU4\\
 \hline
 \multirow{2}{*}{1} & \multirow{2}{*}{2} & \{(1,2)\} & \multirow{2}{*}{5} & \multirow{2}{*}{2} & \multirow{2}{*}{1} & \multirow{2}{*}{0} & \multirow{2}{*}{0} \\
 & & \{(1,3),(3,2)\} & & & & & \\ \hline
 \multirow{2}{*}{1} & \multirow{2}{*}{3} & \{(1,3)\} & \multirow{2}{*}{1} & \multirow{2}{*}{4} & \multirow{2}{*}{1} & \multirow{2}{*}{0} & \multirow{2}{*}{0}\\
 & & \{(1,2),(2,3)\} & & & & &\\ \hline
 \multirow{2}{*}{1} & \multirow{2}{*}{4} & \{(1,2),(2,4)\} & \multirow{2}{*}{3} & \multirow{2}{*}{2} & \multirow{2}{*}{1} & \multirow{2}{*}{0} & \multirow{2}{*}{0}\\
 & & \{(1,3),(3,5),(5,4)\} & & & & &\\ \hline
 \multirow{2}{*}{1} & \multirow{2}{*}{5} & \{(1,3),(3,5)\} & \multirow{2}{*}{1} & \multirow{2}{*}{0} & \multirow{2}{*}{0} & \multirow{2}{*}{0} & \multirow{2}{*}{0}\\
 & & \{(1,2),(2,4),(4,5)\} & & & & &\\ \hline
 \multirow{2}{*}{1} & \multirow{2}{*}{6} & \{(1,2),(2,4),(4,6)\} & \multirow{2}{*}{3} & \multirow{2}{*}{5} & \multirow{2}{*}{0} & \multirow{2}{*}{0} & \multirow{2}{*}{0}\\
 & & \{(1,3),(3,5),(5,6)\} & & & & &\\ \hline
 \multirow{2}{*}{2} & \multirow{2}{*}{3} & \{(2,3)\} & \multirow{2}{*}{0} & \multirow{2}{*}{0} & \multirow{2}{*}{0} & \multirow{2}{*}{1} & \multirow{2}{*}{0}\\
 & & \{(2,1),(1,3)\} & & & & &\\ \hline
 \multirow{2}{*}{2} & \multirow{2}{*}{4} & \{(2,4)\} & \multirow{2}{*}{1} & \multirow{2}{*}{3} & \multirow{2}{*}{0} & \multirow{2}{*}{0} & \multirow{2}{*}{0}\\
 & & \{(2,3),(3,5),(5,4)\} & & & & &\\ \hline
 \multirow{2}{*}{2} & \multirow{2}{*}{5} & \{(2,4),(4,5)\} & \multirow{2}{*}{5} & \multirow{2}{*}{1} & \multirow{2}{*}{1} & \multirow{2}{*}{0} & \multirow{2}{*}{0}\\
 & & \{(2,3),(3,5)\} & & & & &\\ \hline
 \multirow{2}{*}{2} & \multirow{2}{*}{6} & \{(2,4),(4,6)\} & \multirow{2}{*}{0} & \multirow{2}{*}{1} & \multirow{2}{*}{0} & \multirow{2}{*}{1} & \multirow{2}{*}{1}\\
 & & \{(2,3),(3,5),(5,6)\} & & & & &\\ \hline
 \multirow{2}{*}{3} & \multirow{2}{*}{4} & \{(3,2),(2,4)\} & \multirow{2}{*}{1} & \multirow{2}{*}{1} & \multirow{2}{*}{1} & \multirow{2}{*}{0} & \multirow{2}{*}{0}\\
 & & \{(3,5),(5,4)\} & & & & &\\ \hline
 \multirow{2}{*}{3} & \multirow{2}{*}{5} & \{(3,5)\} & \multirow{2}{*}{4} & \multirow{2}{*}{1} & \multirow{2}{*}{1} & \multirow{2}{*}{1} & \multirow{2}{*}{0}\\
 & & \{(3,2),(2,4),(4,5)\} & & & & &\\ \hline
 \multirow{2}{*}{3} & \multirow{2}{*}{6} & \{(3,5),(5,6)\} & \multirow{2}{*}{1} & \multirow{2}{*}{0} & \multirow{2}{*}{0} & \multirow{2}{*}{0} & \multirow{2}{*}{0}\\
 & & \{(3,2),(2,4),(4,6)\} & & & & &\\ \hline
 \multirow{2}{*}{4} & \multirow{2}{*}{5} & \{(4,5)\} & \multirow{2}{*}{1} & \multirow{2}{*}{1} & \multirow{2}{*}{1} & \multirow{2}{*}{0} & \multirow{2}{*}{0}\\
 & & \{(4,6),(6,5)\} & & & & &\\ \hline
 \multirow{2}{*}{4} & \multirow{2}{*}{6} & \{(4,6)\} & \multirow{2}{*}{1} & \multirow{2}{*}{3} & \multirow{2}{*}{0} & \multirow{2}{*}{0} & \multirow{2}{*}{0}\\
 & & \{(4,5),(5,6)\} & & & & &\\ \hline
 \multirow{2}{*}{5} & \multirow{2}{*}{6} & \{(5,6)\} & \multirow{2}{*}{3} & \multirow{2}{*}{1} & \multirow{2}{*}{1} & \multirow{2}{*}{0} & \multirow{2}{*}{1}\\
 & & \{(5,4),(4,6)\} & & & & &\\
 \hline
\end{tabular}
\caption{Table with description of routing. We are assuming that between a pair of nodes all demands follow the same route.}
\label{path_opaque_protec_ref_high}
\end{table}


\vspace{15pt}
Finally and most importantly through table \ref{scriptopaque_protec_ref_high} we can see the CAPEX result for this model. This value is obtained using equation \ref{ILPOpaque_CAPEX} and all of the constraints mentioned above. In table \ref{formulas_opaque} mentioned in first model we can see how all the values were calculated.\\
\newpage
\begin{table}[h!]
\centering
\begin{tabular}{|| c | c | c | c | c | c | c ||}
 \hline
 \multicolumn{7}{|| c ||}{CAPEX of the Network} \\
 \hline
 \hline
 \multicolumn{3}{|| c |}{ } & Quantity & Unit Price & Cost & Total \\
 \hline
 \multirow{3}{*}{Link Cost} & \multicolumn{2}{ c |}{OLTs} & 16 & 15 000 \euro & 240 000 \euro & \multirow{3}{*}{397 520 000 \euro} \\ \cline{2-6}
 & \multicolumn{2}{ c |}{100 Gbits/s Transceivers} & 794 & 5 000 \euro/Gbit/s & 397 000 000 \euro & \\ \cline{2-6}
 & \multicolumn{2}{ c |}{Amplifiers} & 70 & 4 000 \euro & 280 000 \euro & \\
 \hline
 \multirow{9}{*}{Node Cost} & \multirow{7}{*}{Electrical} & EXCs & 6 & 10 000 \euro & 60 000 \euro & \multirow{9}{*}{79 511 800 \euro} \\ \cline{3-6}
 & & ODU0 Ports & 1 200 & 10 \euro/port & 12 000 \euro & \\ \cline{3-6}
 & & ODU1 Ports & 1 000 & 15 \euro/port & 15 000 \euro & \\ \cline{3-6}
 & & ODU2 Ports & 320 & 30 \euro/port & 9 600 \euro & \\ \cline{3-6}
 & & ODU3 Ports & 120 & 60 \euro/port & 7 200 \euro & \\ \cline{3-6}
 & & ODU4 Ports & 80 & 100 \euro/port & 8 000 \euro & \\ \cline{3-6}
 & & Line Ports & 794 & 100 000 \euro/port & 79 400 000 \euro & \\ \cline{2-6}
 & \multirow{2}{*}{Optical} & OXCs & 0 & 20 000 \euro & 0 \euro & \\ \cline{3-6}
 & & Ports & 0 & 2 500 \euro/porto & 0 \euro & \\
 \hline
 \multicolumn{6}{|| c |}{Total Network Cost} & 477 031 800 \euro \\
\hline
\end{tabular}
\caption{Table with detailed description of CAPEX}
\label{scriptopaque_protec_ref_high}
\end{table}

\vspace{15pt}
\subsubsection{Conclusions}

Once we have obtained the results for all the scenarios we will now draw some conclusions about these results. For a better analysis of the results will be created the table \ref{table_comparative_opaque_protec} with the number of line ports, tributary ports and transceivers because they are important values for the cost of CAPEX, the cost of links, the cost of nodes and finally the cost of CAPEX.\\

\begin{table}[h!]
\centering
\begin{tabular}{| c | c | c | c |}
 \hline
  & Low Traffic & Medium Traffic  & High Traffic \\
 \hline\hline
 CAPEX survivability & 11 266 590 \euro & 90 605 900 \euro & 178 231 800 \euro \\ \hline
 CAPEX/Gbit/s survivability & 22 533.18 \euro/Gbit/s & 18 121.18 \euro/Gbit/s & 17 823.18 \euro/Gbit/s \\ \hline
 Traffic (Gbit/s) & 500 & 5 000 & 10 000 \\ \hline
 Bidirectional Links used & 8 & 8 & 8 \\ \hline
 Number of Line ports & 44 & 398 & 794 \\ \hline
 Number of Tributary ports & 138 & 1 380 & 2 760 \\ \hline
 Number of Transceivers & 44 & 398 & 794 \\ \hline
 Link Cost & 22 520 000 \euro & 199 520 000 \euro & 397 520 000 \euro \\ \hline
 Node Cost & 4 462 590 \euro & 39 885 900 \euro & 79 511 800 \euro \\ \hline
 CAPEX & \textbf{26 982 590 \euro} & \textbf{239 405 900 \euro} & \textbf{477 031 800 \euro} \\ \hline
 CAPEX/Gbit/s & \textbf{53 965.18 \euro/Gbit/s} & \textbf{47 881.18 \euro/Gbit/s} & \textbf{47 703.18 \euro/Gbit/s}\\ \hline
\end{tabular}
\caption{Table with different value of CAPEX for this case.}
\label{table_comparative_opaque_protec}
\end{table}


Looking at the previous table we can make some comparisons between the several scenarios:

\begin{itemize}
  \item All scenarios uses all available links. This is because in this case regardless of traffic we always need two possible paths.
  \item Comparing the low traffic scenario with the others we can see that despite having an increase of factor ten (medium traffic scenario) and factor twenty (high traffic scenario) the same increase does not occur in the final cost (it is lower). This happens because the number of transceivers is smaller than expected.
  \item Comparing the medium traffic scenario with the high traffic scenario we can see that the increase of the factor is double and in the final cost this factor is very close but still inferior. Again this happens because the number of transceivers is lower but very close to the expected (has two less than expected).
  \item Comparing this cost with survivability cost we can conclude that protection is significantly more expensive. As can be seen in the table this increase is more than double soon with 1+1 protection we have a cost more than twice the cost without protection.
\end{itemize}

\vspace{13pt}
\subsubsection{Opens Issues}

The creation of this model for any scenario, started with some considerations and some open issues being:

\begin{itemize}
  \item Allow blocking.
  \subitem The presented model assume that the solution is possible or impossible, does not support a partial solution where some demands are not routed (are blocked).
  \item Allow multiple transmission system.
  \subitem The presented model for each link only supports one transmission system.
  \item Allowing multi-path routing.
  \subitem The presented model for all demands sharing the same node pairs have to follow the same path.
\end{itemize}

