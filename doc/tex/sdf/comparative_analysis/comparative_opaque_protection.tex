\clearpage

\section{Opaque with 1+1 Protection}\label{comparative_Opaque_Protection}
\begin{tcolorbox}	
\begin{tabular}{p{2.75cm} p{0.2cm} p{10.5cm}} 	
\textbf{Student Name}  &:& Tiago Esteves    (October 03, 2017 - )\\
\textbf{Goal}          &:& Comparative analysis of the results of the models used for the opaque transport mode with 1 plus 1 protection.
\end{tabular}
\end{tcolorbox}
\vspace{11pt}


In this section we will compare the CAPEX values obtained for the three scenarios in the three types of dimensioning. For a better analysis of the results will be created the table \ref{table_comparative_opaque_protec_ref_1} (scenario 1), the table \ref{table_comparative_opaque_protec_ref_2} (scenario 2) and the table \ref{table_comparative_opaque_protec_ref_3} (scenario 3) with the different values obtained.\\

\textbf{Low traffic scenario:}

\begin{table}[h!]
\centering
\begin{tabular}{| c | c | c | c |}
 \hline
   & Analytical & ILP & Heuristic \\
 \hline\hline
 Link Cost & 32 380 000 \euro & 30 452 000 \euro & 29 520 000 \euro \\
 Node Cost & 8 191 200 \euro & 6 062 590 \euro & 5 862 590 \euro \\
 CAPEX & \textbf{40 571 200 \euro} & \textbf{36 514 590 \euro} & \textbf{35 382 590 \euro} \\
 \hline
\end{tabular}
\caption{Table with different value of CAPEX }
\label{table_comparative_opaque_protec_ref_1}
\end{table}

\vspace{11pt}
Looking at the previous table we can make some comparisons between the several different models of dimensioning and finally draw some conclusions.

\begin{itemize}
  \item We can conclude that in this case the dimensioning using ILP is the best (lowest cost).
  \item In comparison with the analytical model we can see that there is a difference considered with a 32\% error, this value is mainly due to the number of optical channels because in the analytical model more optical channels are calculated and used than those required in the ILP model.
  \item In comparison with the heuristic model we can see that there is a small difference with a 11\% error, much smaller than the previous one.
\end{itemize}

\vspace{11pt}
\textbf{Medium traffic scenario:}

\begin{table}[h!]
\centering
\begin{tabular}{| c | c | c | c |}
 \hline
   & Analytical & ILP & Heuristic \\
 \hline\hline
 Link Cost & 308 380 000 \euro & 199 520 000 \euro & 250 520 000 \euro \\
 Node Cost & 81 379 200 \euro & 39 885 900 \euro & 50 062 590 \euro \\
 CAPEX & \textbf{389 759 200 \euro} & \textbf{239 405 900 \euro} & \textbf{300 582 590 \euro} \\
 \hline
\end{tabular}
\caption{Table with different value of CAPEX }
\label{table_comparative_opaque_protec_ref_2}
\end{table}

\vspace{11pt}
Looking at the previous table we can make some comparisons between the several different models of dimensioning and finally draw some conclusions.

\begin{itemize}
  \item We can conclude that in this case the dimensioning using ILP is the best (lowest cost).
  \item In comparison with the analytical model we can see that there is a difference considered with a 50\% error.
  \item In comparison with the heuristic model we can see that there is a smaller difference with a 3\% error.
\end{itemize}


\vspace{11pt}
\textbf{High traffic scenario:}\\

\begin{table}[h!]
\centering
\begin{tabular}{| c | c | c | c |}
 \hline
   & Analytical & ILP & Heuristic \\
 \hline\hline
 Link Cost & 614 380 000 \euro & 397 520 000 \euro & 497 520 000 \euro \\
 Node Cost & 162 700 800 \euro & 79 511 800 \euro & 99 462 590 \euro \\
 CAPEX & \textbf{777 080 800 \euro} & \textbf{477 031 800 \euro} & \textbf{596 982 590 \euro} \\
 \hline
\end{tabular}
\caption{Table with different value of CAPEX }
\label{table_comparative_opaque_protec_ref_3}
\end{table}

\vspace{11pt}
Looking at the previous table we can make some comparisons between the several different models of dimensioning and finally draw some conclusions.

\begin{itemize}
  \item We can conclude that in this case the dimensioning using ILP is the best (lowest cost).
  \item In comparison with the analytical model we can see that there is a difference considered with a 66\% error.
  \item In comparison with the heuristic model we can see that there is a smaller difference with a 2\% error.
\end{itemize}

