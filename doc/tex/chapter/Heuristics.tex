% ------------------------------------------------------------------------
\chapter{Heuristic Models}

Heuristic algorithms are used in this report with an objective to minimize the total CAPEX of the network. It is important to have these comparison results with the linear programming ones, because these formulations are faster than the optimal methods and can generate also a near optimal solution. Another advantage of the heuristic approach is that heuristic solutions leads to good performances in practical network scenarios when we present a sufficiently feasible solution, instead of an optimal solution.

In order to get the total network cost for the reference network, the CAPEX is calculated by routing and grooming heuristic algorithms implemented in Java and it is considered that all network equipment is bidirectional. These algorithms are tested in a network design software called Net2Plan.

This chapter consists in demonstrating how the matrices are created, how the heuristic algorithms work and analyzing the results. It is divided in six subsections and the results differ into three different transport modes: opaque (link-by-link grooming method), transparent (single-hop grooming method) and translucent (multi-hop grooming method). Each one of these transport methods are also distinguished and compared by the possibility of being without survivability or with 1+1 protection and for the cases of low, medium and high traffic in the network.

%Section CAPEX
\clearpage

\section{CAPEX}
\begin{tcolorbox}	
\begin{tabular}{p{2.75cm} p{0.2cm} p{10.5cm}} 	
\textbf{Student Name}  &:& Tiago Esteves    (October 03, 2017 - )\\
\textbf{Goal}          &:& Implement of the heuristic model to obtain the best possible CAPEX of a given network.
\end{tabular}
\end{tcolorbox}
\vspace{11pt}


%Subsection with the different transport mode
\clearpage

\subsection{Opaque without Survivability}\label{heuristic_Opaque_Survivability}
\begin{tcolorbox}	
\begin{tabular}{p{2.75cm} p{0.2cm} p{10.5cm}} 	
\textbf{Student Name}  &:& Pedro Coelho    (March 01, 2018 - )\\
\textbf{Goal}          &:& Implement the Heuristic model for the opaque transport mode without survivability.
\end{tabular}
\end{tcolorbox}

\subsubsection{Model description}

\vspace{11pt}
In the opaque transport mode (link-by-link approach), the lightpath entering any intermediate node is necessarily terminated, i.e., there are performed OEO (optical-electrical-optical) conversions at every intermediate node since the origin to the destination node. These conversions are used for every wavelength at every node.

Contrary to the opaque with dedicated 1+1 protection technique, the opaque without survivability technique does not have a backup path, so if there is a network failure it is more likely to suffer large data losses, which consequently leads to higher network costs. However, the CAPEX will be significantly lower, because that not includes a secondary path that would increase several network elements.

For this model, after the creation of the matrices and the network topology, it is necessary to apply the routing and grooming algorithms created. For the "Logical Topology" algorithm, the user must insert "Opaque" in the "logicalTopology" value and for the "Grooming" algorithm, the user must insert "no" in the parameter value "protection".

At the end, the "Cost Report" algorithm will be applied to obtain the best CAPEX result for the network in question.

Firstly, in the opaque transport mode, the optical node cost is 0 because all the ports in the network are electrical. Consequently, to calculate the nodes' cost in this transport mode it only has to be considered the electrical nodes' cost:

\begin{itemize}
  \item $N_{OXC,n}$ = 0, \quad $\forall$ n
  \item $N_{EXC,n}$ = 1, \quad $\forall$ n that process traffic
\end{itemize}

As previously mentioned, equation \ref{EXC_pexc1_opaque_heuristic} refers to the number of long-reach ports of the electrical switch with bit-rate -1 in node $n$, $P_{exc,-1,n}$, i.e., the number of line ports of node $n$ which can be calculated as

\begin{equation}
P_{exc,-1,n} = \sum_{j=1}^{N} w_{nj}
\label{EXC_pexc1_opaque_heuristic}
\end{equation}

\vspace{11pt}
where $w_{nj}$ is the number of optical channels between node $n$ and node $j$.

\newpage
\vspace{11pt}
As previously mentioned, equation \ref{EXC_pexc2_opaque_heuristic} refers to the number of short-reach ports of the
electrical switch with bit-rate $c$ in node $n$, $P_{exc,c,n}$, i.e., the number of tributary ports with bit-rate $c$ in node $n$ which can be calculated as

\begin{equation}
P_{exc,c,n} = \sum_{d=1}^{N} D_{nd,c}
\label{EXC_pexc2_opaque_heuristic}
\end{equation}

\vspace{11pt}
where $D_{nd,c}$ are the client demands between nodes $n$ and $d$ with bit rate $c$.

\vspace{11pt}
In this case there is the following particularity:

\begin{itemize}
  \item When $n$=$j$, the value of client demands is always zero, i.e, $D_{nn,c}=0$.
\end{itemize}

\subsubsection{Result description}

It is already known all the necessary formulas to obtain the CAPEX value for the reference network \ref{Reference_Network_Topology}. As described in the subsection of the network traffic \ref{Reference_Network_Traffic}, it is necessary to obtain three different values of CAPEX for the low (0.5 Tbit/s), medium (5 Tbit/s) and high (10 Tbit/s) traffic. It is used a network software program called Net2Plan which can design the traffic matrices, create all the network topologies, simulate the algorithms into the network implemented in the programming software called Eclipse and analyze the results obtained.

In this chapter will be demonstrated the results by Vasco's heuristics from 2016. In each of the three traffic scenarios, it will be shown the network topologies followed by the table with the CAPEX value of the network.\\

\textbf{Low Traffic Scenario:}\\

In this scenario we have to take into account the traffic calculated in \ref{low_scenario}. In a first phase we will show the various existing topologies of the network. The first are the allowed topologies, physical and optical topology, the second are the logical topology for all ODUs and finally the resulting physical and optical topology.\\

\begin{figure}[H]
\centering
\includegraphics[width=13cm]{sdf/heuristic/opaque_survivability/low/allowed_physical_low}
\caption{Allowed Physical Topology. The allowed physical topology is defined by the duct and sites in the field. It is assumed that each duct supports up to 1 bidirectional transmission system and each site supports up to 1 node.}
\label{allowed_physical_surv_ref_low_heuristic}
\end{figure}

\begin{figure}[H]
\centering
\includegraphics[width=13cm]{sdf/heuristic/opaque_survivability/low/allowed_optical_low}
\caption{Allowed Optical Topology. The allowed optical topology is defined by the transport mode (opaque transport mode in this case). It is assumed that each transmission system supports up to 100 optical channels.}
\label{allowed_optical_surv_ref_low_heuristic}
\end{figure}

\begin{figure}[H]
\centering
\includegraphics[width=13cm]{sdf/heuristic/opaque_survivability/low/logical_topology_odu0_low}
\caption{ODU0 Logical Topology defined by the ODU0 traffic matrix.}
\label{logical_ODU0_surv_ref_low_heuristic}
\end{figure}

\begin{figure}[H]
\centering
\includegraphics[width=13cm]{sdf/heuristic/opaque_survivability/low/logical_topology_odu1_low}
\caption{ODU1 Logical Topology defined by the ODU1 traffic matrix.}
\label{logical_ODU1_surv_ref_low_heuristic}
\end{figure}

\begin{figure}[H]
\centering
\includegraphics[width=13cm]{sdf/heuristic/opaque_survivability/low/logical_topology_odu2_low}
\caption{ODU2 Logical Topology defined by the ODU2 traffic matrix.}
\label{logical_ODU2_surv_ref_low_heuristic}
\end{figure}

\begin{figure}[H]
\centering
\includegraphics[width=13cm]{sdf/heuristic/opaque_survivability/low/logical_topology_odu3_low}
\caption{ODU3 Logical Topology defined by the ODU3 traffic matrix.}
\label{logical_ODU3_surv_ref_low_heuristic}
\end{figure}

\begin{figure}[H]
\centering
\includegraphics[width=13cm]{sdf/heuristic/opaque_survivability/low/logical_topology_odu4_low}
\caption{ODU4 Logical Topology defined by the ODU4 traffic matrix.}
\label{logical_ODU4_surv_ref_low_heuristic}
\end{figure}

\begin{figure}[H]
\centering
\includegraphics[width=13cm]{sdf/heuristic/opaque_survivability/low/physical_topology_low}
\caption{Physical Topology after dimensioning.}
\label{physical_topology_surv_ref_low_heuristic}
\end{figure}

Following all the steps mentioned in the \ref{net2plan_guide}, applying the routing and grooming heuristic algorithms in the Net2Plan software and using all the data referring to this scenario, the obtained result for the Vasco's heuristics can be consulted in the following table \ref{scriptopaque_surv_ref_low_heuristic}.

\begin{table}[H]
\centering
\begin{tabular}{|| c | c | c | c | c | c | c ||}
 \hline
 \multicolumn{7}{|| c ||}{CAPEX of the Network} \\
 \hline
 \hline
 \multicolumn{3}{|| c |}{ } & Quantity & Unit Price & Cost & Total \\
 \hline
 \multirow{3}{*}{Link Cost} & \multicolumn{2}{ c |}{OLTs} & 16 & 15 000 \euro & 240 000 \euro & \multirow{3}{*}{13 520 000 \euro} \\ \cline{2-6}
 & \multicolumn{2}{ c |}{100 Gb/s Transceivers} & 26 & 5 000 \euro/Gb/s & 13 000 000 \euro & \\ \cline{2-6}
 & \multicolumn{2}{ c |}{Amplifiers} & 70 & 4 000 \euro & 280 000 \euro & \\
 \hline
 \multirow{9}{*}{Node Cost} & \multirow{7}{*}{Electrical} & EXCs & 6 & 10 000 \euro & 60 000 \euro & \multirow{9}{*}{2 662 590 \euro} \\ \cline{3-6}
 & & ODU0 Ports & 60 & 10 \euro & 600 \euro & \\ \cline{3-6}
 & & ODU1 Ports & 50 & 15 \euro & 750 \euro & \\ \cline{3-6}
 & & ODU2 Ports & 16 & 30 \euro & 480 \euro & \\ \cline{3-6}
 & & ODU3 Ports & 6 & 60 \euro & 360 \euro & \\ \cline{3-6}
 & & ODU4 Ports & 4 & 100 \euro & 400 \euro & \\ \cline{3-6}
 & & Line Ports & 26 & 1 000 \euro/Gb/s & 2 600 000 \euro & \\ \cline{2-6}
 & \multirow{2}{*}{Optical} & OXCs & 0 & 20 000 \euro & 0 \euro & \\ \cline{3-6}
 & & Ports & 0 & 2 500 \euro/porto & 0 \euro & \\
 \hline
 \multicolumn{6}{|| c |}{Total Network Cost} & 16 182 590 \euro \\
\hline
\end{tabular}
\caption{Table with detailed description of CAPEX of Vasco's 2016 results.}
\label{scriptopaque_surv_ref_low_heuristic}
\end{table}

Through the formulas mentioned below we can see how all the values of the quantity column were calculated.

\begin{table}[H]
\centering
\begin{tabular}{|| c | c ||}
 \hline
 OLTs & \(\displaystyle 2 \sum_{i=1}^{N}\sum_{j=i+1}^{N} L_{ij} \) \\ \hline
 Transceivers & \(\displaystyle 2 \sum_{i=1}^{N}\sum_{j=i+1}^{N} W_{ij} \) \\ \hline
 Amplifiers & \(\displaystyle \sum_{i=1}^{N}\sum_{j=i+1}^{N} N^R_{ij} L_{ij} \) \\ \hline
 EXCs & \(\displaystyle \sum_{n=1}^N N_{EXC,n} \) \\ \hline
 ODU0 & \(\displaystyle \sum_{o=1}^{N}\sum_{d=1}^{N} D_{od,0} \) \\ \hline
 ODU1 & \(\displaystyle \sum_{o=1}^{N}\sum_{d=1}^{N} D_{od,1} \) \\ \hline
 ODU2 & \(\displaystyle \sum_{o=1}^{N}\sum_{d=1}^{N} D_{od,2} \)\\ \hline
 ODU3 & \(\displaystyle \sum_{o=1}^{N}\sum_{d=1}^{N} D_{od,3} \) \\ \hline
 ODU4 & \(\displaystyle \sum_{o=1}^{N}\sum_{d=1}^{N} D_{od,4} \) \\ \hline
 Line & \(\displaystyle \sum_{i=1}^{N}\sum_{j=1}^{N} W_{ij} \) \\ \hline
 OXCs & For opaque transport mode this parameter is always zero. \\ \hline
 $P_{OXC}$ & Does not exist for this case then it is equal to zero. \\
 \hline
 \end{tabular}
\caption{Table with description of calculation}
\label{formulas_opaque_surv_ref_low_heuristic}
\end{table}

\newpage
\textbf{Medium Traffic Scenario:}\\

In this scenario we have to take into account the traffic calculated in \ref{medium_traffic_scenario}. In a first phase we will show the various existing topologies of the network. The first are the allowed topologies, physical and optical topology, the second are the logical topology for all ODUs and finally the resulting physical and optical topology.\\

\begin{figure}[H]
\centering
\includegraphics[width=13cm]{sdf/heuristic/opaque_survivability/medium/allowed_physical_medium}
\caption{Allowed Physical Topology. The allowed physical topology is defined by the duct and sites in the field. It is assumed that each duct supports up to 1 bidirectional transmission system and each site supports up to 1 node.}
\label{allowed_physical_surv_ref_medium_heuristic}
\end{figure}

\begin{figure}[H]
\centering
\includegraphics[width=13cm]{sdf/heuristic/opaque_survivability/medium/allowed_optical_medium}
\caption{Allowed Optical Topology. The allowed optical topology is defined by the transport mode (opaque transport mode in this case). It is assumed that each transmission system supports up to 100 optical channels.}
\label{allowed_optical_surv_ref_medium_heuristic}
\end{figure}

\begin{figure}[H]
\centering
\includegraphics[width=13cm]{sdf/heuristic/opaque_survivability/medium/logical_topology_odu0_medium}
\caption{ODU0 Logical Topology defined by the ODU0 traffic matrix.}
\label{logical_ODU0_surv_ref_medium_heuristic}
\end{figure}

\begin{figure}[H]
\centering
\includegraphics[width=13cm]{sdf/heuristic/opaque_survivability/medium/logical_topology_odu1_medium}
\caption{ODU1 Logical Topology defined by the ODU1 traffic matrix.}
\label{logical_ODU1_surv_ref_medium_heuristic}
\end{figure}

\begin{figure}[H]
\centering
\includegraphics[width=13cm]{sdf/heuristic/opaque_survivability/medium/logical_topology_odu2_medium}
\caption{ODU2 Logical Topology defined by the ODU2 traffic matrix.}
\label{logical_ODU2_surv_ref_medium_heuristic}
\end{figure}

\begin{figure}[H]
\centering
\includegraphics[width=13cm]{sdf/heuristic/opaque_survivability/medium/logical_topology_odu3_medium}
\caption{ODU3 Logical Topology defined by the ODU3 traffic matrix.}
\label{logical_ODU3_surv_ref_medium_heuristic}
\end{figure}

\begin{figure}[H]
\centering
\includegraphics[width=13cm]{sdf/heuristic/opaque_survivability/medium/logical_topology_odu4_medium}
\caption{ODU4 Logical Topology defined by the ODU4 traffic matrix.}
\label{logical_ODU4_surv_ref_medium_heuristic}
\end{figure}

\begin{figure}[H]
\centering
\includegraphics[width=13cm]{sdf/heuristic/opaque_survivability/medium/physical_topology_medium}
\caption{Physical Topology after dimensioning.}
\label{physical_topology_surv_ref_medium_heuristic}
\end{figure}

Following all the steps mentioned in the \ref{net2plan_guide}, applying the routing and grooming heuristic algorithms in the Net2Plan software and using all the data referring to this scenario, the obtained result for the Vasco's heuristics can be consulted in the following table \ref{scriptopaque_surv_ref_medium_heuristic}.

\begin{table}[H]
\centering
\begin{tabular}{|| c | c | c | c | c | c | c ||}
 \hline
 \multicolumn{7}{|| c ||}{CAPEX of the Network} \\
 \hline
 \hline
 \multicolumn{3}{|| c |}{ } & Quantity & Unit Price & Cost & Total \\
 \hline
 \multirow{3}{*}{Link Cost} & \multicolumn{2}{ c |}{OLTs} & 16 & 15 000 \euro & 240 000 \euro & \multirow{3}{*}{94 520 000 \euro} \\ \cline{2-6}
 & \multicolumn{2}{ c |}{100 Gb/s Transceivers} & 188 & 5 000 \euro/Gb/s & 94 000 000 \euro & \\ \cline{2-6}
 & \multicolumn{2}{ c |}{Amplifiers} & 70 & 4 000 \euro & 280 000 \euro & \\
 \hline
 \multirow{9}{*}{Node Cost} & \multirow{7}{*}{Electrical} & EXCs & 6 & 10 000 \euro & 60 000 \euro & \multirow{9}{*}{18 885 900 \euro} \\ \cline{3-6}
 & & ODU0 Ports & 600 & 10 \euro & 6 000 \euro & \\ \cline{3-6}
 & & ODU1 Ports & 500 & 15 \euro & 7 500 \euro & \\ \cline{3-6}
 & & ODU2 Ports & 160 & 30 \euro & 4 800 \euro & \\ \cline{3-6}
 & & ODU3 Ports & 60 & 60 \euro & 3 600 \euro & \\ \cline{3-6}
 & & ODU4 Ports & 40 & 100 \euro & 4 000 \euro & \\ \cline{3-6}
 & & Line Ports & 188 & 1 000 \euro/Gb/s & 18 800 000 \euro & \\ \cline{2-6}
 & \multirow{2}{*}{Optical} & OXCs & 0 & 20 000 \euro & 0 \euro & \\ \cline{3-6}
 & & Ports & 0 & 2 500 \euro/porto & 0 \euro & \\
 \hline
 \multicolumn{6}{|| c |}{Total Network Cost} & 113 405 900 \euro \\
\hline
\end{tabular}
\caption{Table with detailed description of CAPEX of Vasco's 2016 results.}
\label{scriptopaque_surv_ref_medium_heuristic}
\end{table}

Through the formulas mentioned below we can see how all the values of the quantity column were calculated.

\begin{table}[H]
\centering
\begin{tabular}{|| c | c ||}
 \hline
 OLTs & \(\displaystyle 2 \sum_{i=1}^{N}\sum_{j=i+1}^{N} L_{ij} \) \\ \hline
 Transceivers & \(\displaystyle 2 \sum_{i=1}^{N}\sum_{j=i+1}^{N} W_{ij} \) \\ \hline
 Amplifiers & \(\displaystyle \sum_{i=1}^{N}\sum_{j=i+1}^{N} N^R_{ij} L_{ij} \) \\ \hline
 EXCs & \(\displaystyle \sum_{n=1}^N N_{EXC,n} \) \\ \hline
 ODU0 & \(\displaystyle \sum_{o=1}^{N}\sum_{d=1}^{N} D_{od,0} \) \\ \hline
 ODU1 & \(\displaystyle \sum_{o=1}^{N}\sum_{d=1}^{N} D_{od,1} \) \\ \hline
 ODU2 & \(\displaystyle \sum_{o=1}^{N}\sum_{d=1}^{N} D_{od,2} \)\\ \hline
 ODU3 & \(\displaystyle \sum_{o=1}^{N}\sum_{d=1}^{N} D_{od,3} \) \\ \hline
 ODU4 & \(\displaystyle \sum_{o=1}^{N}\sum_{d=1}^{N} D_{od,4} \) \\ \hline
 Line & \(\displaystyle \sum_{i=1}^{N}\sum_{j=1}^{N} W_{ij} \) \\ \hline
 OXCs & For opaque transport mode this parameter is always zero. \\ \hline
 $P_{OXC}$ & Does not exist for this case then it is equal to zero. \\
 \hline
 \end{tabular}
\caption{Table with description of calculation}
\label{formulas_opaque_surv_ref_medium_heuristic}
\end{table}

\textbf{High Traffic Scenario:}\\

In this scenario we have to take into account the traffic calculated in \ref{high_traffic_scenario}. In a first phase we will show the various existing topologies of the network. The first are the allowed topologies, physical and optical topology, the second are the logical topology for all ODUs and finally the resulting physical and optical topology.\\

\begin{figure}[H]
\centering
\includegraphics[width=13cm]{sdf/heuristic/opaque_survivability/high/allowed_physical_high}
\caption{Allowed Physical Topology. The allowed physical topology is defined by the duct and sites in the field. It is assumed that each duct supports up to 1 bidirectional transmission system and each site supports up to 1 node.}
\label{allowed_physical_surv_ref_high_heuristic}
\end{figure}

\begin{figure}[H]
\centering
\includegraphics[width=13cm]{sdf/heuristic/opaque_survivability/high/allowed_optical_high}
\caption{Allowed Optical Topology. The allowed optical topology is defined by the transport mode (opaque transport mode in this case). It is assumed that each transmission system supports up to 100 optical channels.}
\label{allowed_optical_surv_ref_high_heuristic}
\end{figure}

\begin{figure}[H]
\centering
\includegraphics[width=13cm]{sdf/heuristic/opaque_survivability/high/logical_topology_odu0_high}
\caption{ODU0 Logical Topology defined by the ODU0 traffic matrix.}
\label{logical_ODU0_surv_ref_high_heuristic}
\end{figure}

\begin{figure}[H]
\centering
\includegraphics[width=13cm]{sdf/heuristic/opaque_survivability/high/logical_topology_odu1_high}
\caption{ODU1 Logical Topology defined by the ODU1 traffic matrix.}
\label{logical_ODU1_surv_ref_high_heuristic}
\end{figure}

\begin{figure}[H]
\centering
\includegraphics[width=13cm]{sdf/heuristic/opaque_survivability/high/logical_topology_odu2_high}
\caption{ODU2 Logical Topology defined by the ODU2 traffic matrix.}
\label{logical_ODU2_surv_ref_high_heuristic}
\end{figure}

\begin{figure}[H]
\centering
\includegraphics[width=13cm]{sdf/heuristic/opaque_survivability/high/logical_topology_odu3_high}
\caption{ODU3 Logical Topology defined by the ODU3 traffic matrix.}
\label{logical_ODU3_surv_ref_high_heuristic}
\end{figure}

\begin{figure}[H]
\centering
\includegraphics[width=13cm]{sdf/heuristic/opaque_survivability/high/logical_topology_odu4_high}
\caption{ODU4 Logical Topology defined by the ODU4 traffic matrix.}
\label{logical_ODU4_surv_ref_high_heuristic}
\end{figure}

\begin{figure}[H]
\centering
\includegraphics[width=13cm]{sdf/heuristic/opaque_survivability/high/physical_topology_high}
\caption{Physical Topology after dimensioning.}
\label{physical_topology_surv_ref_high_heuristic}
\end{figure}

Following all the steps mentioned in the \ref{net2plan_guide}, applying the routing and grooming heuristic algorithms in the Net2Plan software and using all the data referring to this scenario, the obtained result for the Vasco's heuristics can be consulted in the following table \ref{scriptopaque_surv_ref_high_heuristic}.

\begin{table}[H]
\centering
\begin{tabular}{|| c | c | c | c | c | c | c ||}
 \hline
 \multicolumn{7}{|| c ||}{CAPEX of the Network} \\
 \hline
 \hline
 \multicolumn{3}{|| c |}{ } & Quantity & Unit Price & Cost & Total \\
 \hline
 \multirow{3}{*}{Link Cost} & \multicolumn{2}{ c |}{OLTs} & 16 & 15 000 \euro & 240 000 \euro & \multirow{3}{*}{186 020 000 \euro} \\ \cline{2-6}
 & \multicolumn{2}{ c |}{100 Gb/s Transceivers} & 371 & 5 000 \euro/Gb/s & 185 500 000 \euro & \\ \cline{2-6}
 & \multicolumn{2}{ c |}{Amplifiers} & 70 & 4 000 \euro & 280 000 \euro & \\
 \hline
 \multirow{9}{*}{Node Cost} & \multirow{7}{*}{Electrical} & EXCs & 6 & 10 000 \euro & 60 000 \euro & \multirow{9}{*}{37 211 800 \euro} \\ \cline{3-6}
 & & ODU0 Ports & 1 200 & 10 \euro & 12 000 \euro & \\ \cline{3-6}
 & & ODU1 Ports & 1 000 & 15 \euro & 15 000 \euro & \\ \cline{3-6}
 & & ODU2 Ports & 320 & 30 \euro & 9 600 \euro & \\ \cline{3-6}
 & & ODU3 Ports & 120 & 60 \euro & 7 200 \euro & \\ \cline{3-6}
 & & ODU4 Ports & 80 & 100 \euro & 8 000 \euro & \\ \cline{3-6}
 & & Line Ports & 371 & 1 000 \euro/Gb/s & 37 100 000 \euro & \\ \cline{2-6}
 & \multirow{2}{*}{Optical} & OXCs & 0 & 20 000 \euro & 0 \euro & \\ \cline{3-6}
 & & Ports & 0 & 2 500 \euro/porto & 0 \euro & \\
 \hline
 \multicolumn{6}{|| c |}{Total Network Cost} & 223 231 800 \euro \\
\hline
\end{tabular}
\caption{Table with detailed description of CAPEX of Vasco's 2016 results.}
\label{scriptopaque_surv_ref_high_heuristic}
\end{table}

Through the formulas mentioned below we can see how all the values of the quantity column were calculated.

\begin{table}[H]
\centering
\begin{tabular}{|| c | c ||}
 \hline
 OLTs & \(\displaystyle 2 \sum_{i=1}^{N}\sum_{j=i+1}^{N} L_{ij} \) \\ \hline
 Transceivers & \(\displaystyle 2 \sum_{i=1}^{N}\sum_{j=i+1}^{N} W_{ij} \) \\ \hline
 Amplifiers & \(\displaystyle \sum_{i=1}^{N}\sum_{j=i+1}^{N} N^R_{ij} L_{ij} \) \\ \hline
 EXCs & \(\displaystyle \sum_{n=1}^N N_{EXC,n} \) \\ \hline
 ODU0 & \(\displaystyle \sum_{o=1}^{N}\sum_{d=1}^{N} D_{od,0} \) \\ \hline
 ODU1 & \(\displaystyle \sum_{o=1}^{N}\sum_{d=1}^{N} D_{od,1} \) \\ \hline
 ODU2 & \(\displaystyle \sum_{o=1}^{N}\sum_{d=1}^{N} D_{od,2} \)\\ \hline
 ODU3 & \(\displaystyle \sum_{o=1}^{N}\sum_{d=1}^{N} D_{od,3} \) \\ \hline
 ODU4 & \(\displaystyle \sum_{o=1}^{N}\sum_{d=1}^{N} D_{od,4} \) \\ \hline
 Line & \(\displaystyle \sum_{i=1}^{N}\sum_{j=1}^{N} W_{ij} \) \\ \hline
 OXCs & For opaque transport mode this parameter is always zero. \\ \hline
 $P_{OXC}$ & Does not exist for this case then it is equal to zero. \\
 \hline
 \end{tabular}
\caption{Table with description of calculation}
\label{formulas_opaque_surv_ref_high_heuristic}
\end{table} 
\clearpage

\subsection{Opaque with 1+1 Protection}\label{heuristic_Opaque_Protection}
\begin{tcolorbox}	
\begin{tabular}{p{2.75cm} p{0.2cm} p{10.5cm}} 	
\textbf{Student Name}  &:& Pedro Coelho    (01/03/2018 - )\\
\textbf{Goal}          &:& Implement the heuristic model for the opaque transport mode with 1 plus 1 protection.
\end{tabular}
\end{tcolorbox}

\vspace{11pt}
The impact of failure in WDM (Wavelength Division Multiplexing) networks is caused by extremely high volume of traffic carried. In a high speed network like the WDM, a failure of a network element may cause failure of various optical channels that leads to large data and revenue losses, which can interrupt communication services.

In this protection scheme, the primary and backup path carry the traffic end-to-end, i.e., there is a need to have a backup path (the unaffected path) in case of a network failure. Then, the receiver will decide which one of the two incoming traffic it is going to pick, if the primary or the backup path.

Although it is the fastest protection scheme, it is also the most expensive, because it normally uses more than the double of the capacity of the primary path. This happens because the backup path is typically longer than the primary.

After the creation of the matrices and the network topology, it is necessary to apply the routing and grooming algorithms created. In the end, a report algorithm will be applied to obtain the best CAPEX result for the network in question.

Firstly, in the opaque transport mode, the optical node cost is 0 because all the ports in the network are electrical. Consequently, to calculate the nodes' cost in this transport mode it only has to be considered the electrical nodes' cost:

\begin{itemize}
  \item $N_{OXC,n}$ = 0, \quad $\forall$ n
  \item $N_{EXC,n}$ = 1, \quad $\forall$ n that process traffic
\end{itemize}

As previously mentioned, equation \ref{EXC_pexc1_opaque_heuristic_protec} refers to the number of long-reach ports of the electrical switch with bit-rate -1 in node $n$, $P_{exc,-1,n}$, i.e., the number of line ports of node $n$ which can be calculated as

\begin{equation}
P_{exc,-1,n} = \sum_{j=1}^{N} w_{nj}
\label{EXC_pexc1_opaque_heuristic_protec}
\end{equation}

\noindent
where $w_{nj}$ is the number of optical channels between node $n$ and node $j$.

\newpage
\vspace{11pt}
As previously mentioned, equation \ref{EXC_pexc2_opaque_heuristic_protec} refers to the number of short-reach ports of the electrical switch with bit-rate $c$ in node $n$, $P_{exc,c,n}$, i.e., the number of tributary ports with bit-rate $c$ in node $n$ which can be calculated as

\begin{equation}
P_{exc,c,n} = \sum_{d=1}^{N} D_{nd,c}
\label{EXC_pexc2_opaque_heuristic_protec}
\end{equation}

\noindent
where $D_{nd,c}$ are the client demands between nodes $n$ and $d$ with bit rate $c$.\\

\noindent
In this case there is the following particularity:

\begin{itemize}
  \item When $n$=$j$, the value of client demands is always zero, i.e, $D_{nn,c}=0$.
\end{itemize}

\vspace{11pt}
To implement this heuristic approach there are used algorithms made in Java in a programming software called Eclipse and they are tested in an open-source network program called Net2Plan. In the Net2Plan guide section \ref{net2plan_guide} there is an explanation on how to use and test them in this network planner.

In the next pages it will be described all the steps performed to obtain the final results in the opaque transport mode with 1+1 protection. In the figure below \ref{fluxogram_opaque_protec} it is shown a fluxogram with the description of this transport mode approach.

\begin{figure}[H]
\centering
\includegraphics[width=16cm]{sdf/heuristic/opaque_protection/figures/fluxogram_opaque_protec}
\caption{Fluxogram with the steps performed in the opaque with 1+1 protection transport mode approach.}
\label{fluxogram_opaque_protec}
\end{figure}

\newpage
\subsubsection{Creation and join the traffic matrices}

\noindent
The first step is to create the traffic matrices based on the reference network \ref{Reference_Network_Topology}. In order to create the 5 traffic matrices in Net2Plan it is necessary the length of all the links and the total traffic used in this network, so later it is needed to define in Net2Plan the length in all end nodes and the total traffic depends on the value of traffic used (low traffic - 0.5 Tbit/s, medium traffic - 5 Tbit/s and high traffic - 10 Tbit/s). As you can see in the figure below, it is defined the path of the 5 ODUs and they will be aggregated in just one single ODU, making it possible to join all the demands in just one file and load it later into the network. This final resulting ODU joins the multiple traffic demands from all the traffic matrices previously created and, of course, the traffic demands will depend on the values used on the creation of the matrices (low, medium and high traffic).

\begin{figure}[H]
\centering
\includegraphics[width=7cm]{sdf/heuristic/opaque_protection/figures/join_matrices_odus}
\caption{Join the 5 ODU traffic matrices into 1 single file "ODUs". The 5 traffic demands from the traffic matrices previously created are joined into 1 file to load it later on Net2Plan.}
\label{join_matrices_odus}
\end{figure}

\begin{figure}[H]
\centering
\includegraphics[width=8cm]{sdf/heuristic/opaque_protection/figures/traffic_matrices}
\caption{Load of the join traffic matrices algorithm for the opaque transport mode on Net2Plan. It is defined the 5 paths to load the 5 ODU traffic matrices and the last path is the one where will be saved the file that joins all 5 the traffic demands.}
\label{traffic_matrices_surv_ref}
\end{figure}

\newpage
\subsubsection{Creation of the physical topology}

\vspace{11pt}
The next step is to create the allowed physical topology of the network in Net2Plan. This network consists in 6 nodes and 8 bidirectional links. It is now also possible to define the length in all links. In the figure below it is shown the allowed physical topology in this transport mode.

\begin{figure}[H]
\centering
\includegraphics[width=12cm]{sdf/heuristic/opaque_protection/figures/allowed_physical}
\caption{Allowed physical topology. The allowed physical topology is defined by the duct and sites in the field. It is assumed that each duct supports up to 1 bidirectional transmission system and each site supports up to 1 node.}
\label{allowed_physical_surv_ref_low_heuristic}
\end{figure}

\subsubsection{Creation of the logical topology}

\vspace{11pt}
It is now time to create the allowed logical topology. A network topology represents how the links and the nodes of the network interconnect with each other and the logical topology algorithm creates the logical topology on another layer. In the opaque transport mode the physical and logical topologies are the same, so it is needed to add a new layer from the lower layer (default layer). The lower layer is the physical layer of the network and it is now created a new upper layer which is the logical layer of the network and represents the logical topology of the opaque transport mode. The allowed physical and optical topologies, the logical topologies for all ODUs and the resulting physical topology is shown in the next section below \ref{result_description_opaque_heuristic_protec} for the three traffic scenarios. It is shown below three figures with the code in Java of the creation of the network logical topology, the load of the logical topology algorithm in Net2Plan and the resulting allowed optical topology for the opaque transport mode with 1+1 protection.

\begin{figure}[H]
\centering
\includegraphics[width=15cm]{sdf/heuristic/opaque_protection/figures/logical_topology_creation_opaque}
\caption{Java code of the logical topology approach for the opaque transport mode. The logical layer is created from the physical layer, as in this transport mode they are the same. The new layer is now the opaque logical topology of the network.}
\label{logical_topology_creation_opaque}
\end{figure}

\begin{figure}[H]
\centering
\includegraphics[width=11cm]{sdf/heuristic/opaque_protection/figures/logical_topology_load_opaque}
\caption{Load of the logical topology algorithm for the opaque transport mode.}
\label{logical_topology_load_opaque_protec}
\end{figure}

\begin{figure}[H]
\centering
\includegraphics[width=10cm]{sdf/heuristic/opaque_protection/figures/allowed_optical}
\caption{Allowed optical topology. It is assumed that each transmission system supports up to 100 optical channels.}
\label{allowed_optical_protec}
\end{figure}

\subsubsection{Creation of routes and aggregation of traffic}

\vspace{11pt}
After a network topology is created, it is now time to set the routing algorithm. In the opaque with 1+1 protection transport mode the routing algorithm starts with going through all the demands, create bidirectional routes (in this case the primary paths) based on the shortest path Dijkstra algorithm and then search the candidate routes for the respective demand. In this report it is used the shortest path type in hops. These routes are ordered sequentially and the shortest one per each demand is the primary path. The demands from the lower layer are removed and then saved in the upper layer. After this step, it is needed to set the traffic demands into the candidate routes that will integrate the network.
As we also have a dedicated 1+1 protection scheme, if the network has this feature active, the algorithm will compare the previous candidate routes that will be saved to a list with the new ones that will be created. If the new routes are different from the previously created ones and if they are the next shortest path routes, then the algorithm will add these routes to the network and they will be the protection segments (backup paths) of the network. The offered traffic demands will be also set into these protection path routes. The final resulting backup path routes are used to prevent network failures. Despite of the fact the network will be much more secure, the network CAPEX will increase more than the double, due to the creation of primary and backup paths.

\begin{figure}[H]
\centering
\includegraphics[width=14cm]{sdf/heuristic/opaque_protection/figures/grooming_opaque_protec1}
\caption{Creation of routes and aggregation of traffic for the opaque with 1+1 protection transport mode. The candidate routes are searched by the shortest path type and the offered traffic demands are set into these routes.}
\label{grooming_opaque_protec1}
\end{figure}

\begin{figure}[H]
\centering
\includegraphics[width=14cm]{sdf/heuristic/opaque_protection/figures/grooming_opaque_protec2}
\caption{Creation of routes and aggregation of traffic for the opaque with 1+1 protection transport mode. The protection segments are added to all the primary paths that were chosen by the shortest path type method.}
\label{grooming_opaque_protec2}
\end{figure}

\begin{table}[H]
\centering
\begin{tabular}{|| c | c ||}
 \hline
 Function & Definition \\
 \hline\hline
 netPlan.getDemands(lowerLayer) & Returns the array of demands for the lower layer. \\
 \hline
 d.getRoutes() & Returns all the routes associated to the demand "d". \\
 \hline
 c.setCarriedTraffic() & \makecell{Sets the route carried traffic and the occupied capacity\\in the links, setting it up to be the same in all links.} \\
 \hline
 d.getOfferedTraffic() & Returns the offered traffic of the demand "d". \\
 \hline
 save.getSeqLinksRealPath() & Returns the links of routes ordered sequentially. \\
 \hline
 save.addProtectionSegment(segment) & Add "segment" \ as a protection path in the route "save". \\
 \hline
\end{tabular}
\caption{Table with the description of the main functions in the creation of routes and aggregation of traffic in the grooming algorithm.}
\label{grooming_table_variables_opaque_protec}
\end{table}

\subsubsection{Calculation of the number of wavelengths per link}

\vspace{11pt}
The final step of the routing and grooming algorithms is to calculate the number of wavelengths per link for the whole network. This is the last and an important step because with the number of wavelengths per link in the network, it is possible to calculate other network components. In the opaque transport mode, as in the figure below shows, the algorithm starts with going through all the links and getting the capacity based on the traffic per demand. The total carried traffic in the link including protection and non-protection segments will be divided by the wavelength capacity and it is now possible to obtain the number of wavelengths per link.

\begin{figure}[H]
\centering
\includegraphics[width=16cm]{sdf/heuristic/opaque_protection/figures/grooming_opaque_protec3}
\caption{Calculation of the number of wavelengths per link for the opaque transport mode. The link capacity is based on the traffic per demand.}
\label{grooming_opaque_surv2}
\end{figure}

\begin{figure}[H]
\centering
\includegraphics[width=11cm]{sdf/heuristic/opaque_protection/figures/grooming_opaque_protec4}
\caption{Load of the grooming algorithm for the opaque with 1+1 protection transport mode. The total number of routes per demand is set to 10, the user can define if the model is with or without protection, the shortest path type is set to "hops" \ and the capacity per wavelength is used 100 optical channels.}
\label{grooming_opaque_surv3}
\end{figure}

\subsubsection{Network cost report}

\vspace{11pt}
In order to obtain the network CAPEX results, the formulas needed to calculate the network elements and that are demonstrated previously in the beginning of this section \ref{heuristic_Opaque_Protection} were "translated" \ into Java code in a cost report algorithm. This algorithm can be loaded in Net2Plan and calculates and shows in tables the network CAPEX and also the per-link and per-node information with more details.
In the opaque transport mode the optical node cost is 0 because all the network ports are electrical, so it only has to be considered the electrical nodes' costs and the electrical links' costs.

\begin{figure}[H]
\centering
\includegraphics[width=10cm]{sdf/heuristic/opaque_protection/figures/cost_report_opaque}
\caption{Load of the cost report algorithm on Net2Plan. The result view is an HTML page with the network optical and electrical components and their costs.}
\label{cost_report_opaque}
\end{figure}

\newpage
\subsubsection{Result description}\label{result_description_opaque_heuristic_protec}

It is already known all the necessary formulas to obtain the CAPEX value for the reference network \ref{Reference_Network_Topology}. As described in the subsection of the network traffic \ref{Reference_Network_Traffic}, it is necessary to obtain three different values of CAPEX for the low (0.5 Tbit/s), medium (5 Tbit/s) and high (10 Tbit/s) traffic. It is used a network software program called Net2Plan which can design the traffic matrices, create all the network topologies, simulate the algorithms into the network implemented in the programming software called Eclipse and analyze the results obtained.\\
In this chapter will be demonstrated the results by Vasco's heuristics from 2016. In each of the three traffic scenarios, it will be shown the network topologies followed by the table with the CAPEX value of the network.\\

\noindent
\textbf{Low Traffic Scenario:}\\

In this scenario we have to take into account the traffic calculated in \ref{low_scenario}. In a first phase we will show the various existing topologies of the network. The first are the allowed topologies, physical and optical topologies, the second are the logical topology for all ODUs and finally the resulting physical topology.\\

\begin{figure}[H]
\centering
\includegraphics[width=13cm]{sdf/heuristic/opaque_protection/figures/allowed_physical}
\caption{Allowed physical topology. The allowed physical topology is defined by the duct and sites in the field. It is assumed that each duct supports up to 1 bidirectional transmission system and each site supports up to 1 node.}
\label{allowed_physical_protec_ref_low_heuristic}
\end{figure}

\begin{figure}[H]
\centering
\includegraphics[width=13cm]{sdf/heuristic/opaque_protection/figures/allowed_optical}
\caption{Allowed optical topology. The allowed optical topology is defined by the transport mode (opaque transport mode in this case). It is assumed that each transmission system supports up to 100 optical channels.}
\label{allowed_optical_protec_ref_low_heuristic}
\end{figure}

\begin{figure}[H]
\centering
\includegraphics[width=13cm]{sdf/heuristic/opaque_protection/figures/logical_topology_odu0_low}
\caption{ODU0 logical topology defined by the ODU0 traffic matrix.}
\label{logical_ODU0_protec_ref_low_heuristic}
\end{figure}

\begin{figure}[H]
\centering
\includegraphics[width=13cm]{sdf/heuristic/opaque_protection/figures/logical_topology_odu1_low}
\caption{ODU1 logical topology defined by the ODU1 traffic matrix.}
\label{logical_ODU1_protec_ref_low_heuristic}
\end{figure}

\begin{figure}[H]
\centering
\includegraphics[width=13cm]{sdf/heuristic/opaque_protection/figures/logical_topology_odu2_low}
\caption{ODU2 logical topology defined by the ODU2 traffic matrix.}
\label{logical_ODU2_protec_ref_low_heuristic}
\end{figure}

\begin{figure}[H]
\centering
\includegraphics[width=13cm]{sdf/heuristic/opaque_protection/figures/logical_topology_odu3_low}
\caption{ODU3 logical topology defined by the ODU3 traffic matrix.}
\label{logical_ODU3_protec_ref_low_heuristic}
\end{figure}

\begin{figure}[H]
\centering
\includegraphics[width=13cm]{sdf/heuristic/opaque_protection/figures/logical_topology_odu4_low}
\caption{ODU4 logical topology defined by the ODU4 traffic matrix.}
\label{logical_ODU4_protec_ref_low_heuristic}
\end{figure}

\begin{figure}[H]
\centering
\includegraphics[width=13cm]{sdf/heuristic/opaque_protection/figures/physical_topology}
\caption{Physical topology after dimensioning.}
\label{physical_topology_protec_ref_low_heuristic}
\end{figure}

\begin{figure}[H]
\centering
\includegraphics[width=13cm]{sdf/heuristic/opaque_protection/figures/optical_topology_low}
\caption{Optical topology after dimensioning.}
\label{optical_topology_protec_ref_low_heuristic}
\end{figure}

Following all the steps mentioned in the \ref{net2plan_guide}, applying the routing and grooming heuristic algorithms in the Net2Plan software and using all the data referring to this scenario, the obtained result for the Vasco's heuristics can be consulted in the following table \ref{scriptopaque_protec_ref_low_heuristic}. In table \ref{formulas_opaque_heuristic} mentioned in previous model we can see how all the values were calculated. \\

\begin{table}[H]
\centering
\begin{tabular}{|| c | c | c | c | c | c | c ||}
 \hline
 \multicolumn{7}{|| c ||}{CAPEX of the Network} \\
 \hline
 \hline
 \multicolumn{3}{|| c |}{ } & Quantity & Unit Price & Cost & Total \\
 \hline
 \multirow{3}{*}{\makecell{Link \\ Cost}} & \multicolumn{2}{ c |}{OLTs} & 16 & 15 000 \euro & 240 000 \euro & \multirow{3}{*}{23 520 000 \euro} \\ \cline{2-6}
 & \multicolumn{2}{ c |}{100 Gbits/s Transceivers} & 46 & 5 000 \euro/Gbit/s & 23 000 000 \euro & \\ \cline{2-6}
 & \multicolumn{2}{ c |}{Amplifiers} & 70 & 4 000 \euro & 280 000 \euro & \\
 \hline
 \multirow{9}{*}{\makecell{Node \\ Cost}} & \multirow{7}{*}{Electrical} & EXCs & 6 & 10 000 \euro & 60 000 \euro & \multirow{9}{*}{4 662 590 \euro} \\ \cline{3-6}
  & & ODU0 Ports & 60 & 10 \euro/port & 600 \euro & \\ \cline{3-6}
 & & ODU1 Ports & 50 & 15 \euro/port & 750 \euro & \\ \cline{3-6}
 & & ODU2 Ports & 16 & 30 \euro/port & 480 \euro & \\ \cline{3-6}
 & & ODU3 Ports & 6 & 60 \euro/port & 360 \euro & \\ \cline{3-6}
 & & ODU4 Ports & 4 & 100 \euro/port & 400 \euro & \\ \cline{3-6}
 & & Line Ports & 46 & 100 000 \euro/port & 4 600 000 \euro & \\ \cline{2-6}
 & \multirow{2}{*}{Optical} & OXCs & 0 & 20 000 \euro & 0 \euro & \\ \cline{3-6}
 & & Ports & 0 & 2 500 \euro/port & 0 \euro & \\
 \hline
 \multicolumn{6}{|| c |}{Total Network Cost} & 28 182 590 \euro \\
\hline
\end{tabular}
\caption{Table with detailed description of CAPEX of Vasco's 2016 results.}
\label{scriptopaque_protec_ref_low_heuristic}
\end{table}

\newpage
\noindent
\textbf{Medium Traffic Scenario:}\\

In this scenario we have to take into account the traffic calculated in \ref{medium_traffic_scenario}. In a first phase we will show the various existing topologies of the network. The first are the allowed topologies, physical and optical topologies, the second are the logical topology for all ODUs and finally the resulting physical topology.\\

\begin{figure}[H]
\centering
\includegraphics[width=13cm]{sdf/heuristic/opaque_protection/figures/allowed_physical}
\caption{Allowed physical topology. The allowed physical topology is defined by the duct and sites in the field. It is assumed that each duct supports up to 1 bidirectional transmission system and each site supports up to 1 node.}
\label{allowed_physical_protec_ref_medium_heuristic}
\end{figure}

\begin{figure}[H]
\centering
\includegraphics[width=13cm]{sdf/heuristic/opaque_protection/figures/allowed_optical}
\caption{Allowed optical topology. The allowed optical topology is defined by the transport mode (opaque transport mode in this case). It is assumed that each transmission system supports up to 100 optical channels.}
\label{allowed_optical_protec_ref_medium_heuristic}
\end{figure}

\begin{figure}[H]
\centering
\includegraphics[width=13cm]{sdf/heuristic/opaque_protection/figures/logical_topology_odu0_medium}
\caption{ODU0 logical topology defined by the ODU0 traffic matrix.}
\label{logical_ODU0_protec_ref_medium_heuristic}
\end{figure}

\begin{figure}[H]
\centering
\includegraphics[width=13cm]{sdf/heuristic/opaque_protection/figures/logical_topology_odu1_medium}
\caption{ODU1 logical topology defined by the ODU1 traffic matrix.}
\label{logical_ODU1_protec_ref_medium_heuristic}
\end{figure}

\begin{figure}[H]
\centering
\includegraphics[width=13cm]{sdf/heuristic/opaque_protection/figures/logical_topology_odu2_medium}
\caption{ODU2 logical topology defined by the ODU2 traffic matrix.}
\label{logical_ODU2_protec_ref_medium_heuristic}
\end{figure}

\begin{figure}[H]
\centering
\includegraphics[width=13cm]{sdf/heuristic/opaque_protection/figures/logical_topology_odu3_medium}
\caption{ODU03 logical topology defined by the ODU3 traffic matrix.}
\label{logical_ODU3_protec_ref_medium_heuristic}
\end{figure}

\begin{figure}[H]
\centering
\includegraphics[width=13cm]{sdf/heuristic/opaque_protection/figures/logical_topology_odu4_medium}
\caption{ODU4 logical topology defined by the ODU4 traffic matrix.}
\label{logical_ODU4_protec_ref_medium_heuristic}
\end{figure}

\begin{figure}[H]
\centering
\includegraphics[width=13cm]{sdf/heuristic/opaque_protection/figures/physical_topology}
\caption{Physical topology after dimensioning.}
\label{physical_topology_protec_ref_medium_heuristic}
\end{figure}

\begin{figure}[H]
\centering
\includegraphics[width=13cm]{sdf/heuristic/opaque_protection/figures/optical_topology_medium}
\caption{Optical topology after dimensioning.}
\label{optical_topology_protec_ref_medium_heuristic}
\end{figure}

Following all the steps mentioned in the \ref{net2plan_guide}, applying the routing and grooming heuristic algorithms in the Net2Plan software and using all the data referring to this scenario, the obtained result for the Vasco's heuristics can be consulted in the following table \ref{scriptopaque_protec_ref_medium_heuristic}. In table \ref{formulas_opaque_heuristic} mentioned in previous model we can see how all the values were calculated. \\

\begin{table}[H]
\centering
\begin{tabular}{|| c | c | c | c | c | c | c ||}
 \hline
 \multicolumn{7}{|| c ||}{CAPEX of the Network} \\
 \hline
 \hline
 \multicolumn{3}{|| c |}{ } & Quantity & Unit Price & Cost & Total \\
 \hline
 \multirow{3}{*}{\makecell{Link \\ Cost}} & \multicolumn{2}{ c |}{OLTs} & 16 & 15 000 \euro & 240 000 \euro & \multirow{3}{*}{199 520 000 \euro} \\ \cline{2-6}
 & \multicolumn{2}{ c |}{100 Gbits/s Transceivers} & 398 & 5 000 \euro/Gbit/s & 199 000 000 \euro & \\ \cline{2-6}
 & \multicolumn{2}{ c |}{Amplifiers} & 70 & 4 000 \euro & 280 000 \euro & \\
 \hline
 \multirow{9}{*}{\makecell{Node \\ Cost}} & \multirow{7}{*}{Electrical} & EXCs & 6 & 10 000 \euro & 60 000 \euro & \multirow{9}{*}{39 885 900 \euro} \\ \cline{3-6}
 & & ODU0 Ports & 600 & 10 \euro/port & 6 000 \euro & \\ \cline{3-6}
 & & ODU1 Ports & 500 & 15 \euro/port & 7 500 \euro & \\ \cline{3-6}
 & & ODU2 Ports & 160 & 30 \euro/port & 4 800 \euro & \\ \cline{3-6}
 & & ODU3 Ports & 60 & 60 \euro/port & 3 600 \euro & \\ \cline{3-6}
 & & ODU4 Ports & 40 & 100 \euro/port & 4 000 \euro & \\ \cline{3-6}
 & & Line Ports & 398 & 100 000 \euro/port & 50 000 000 \euro & \\ \cline{2-6}
 & \multirow{2}{*}{Optical} & OXCs & 0 & 20 000 \euro & 0 \euro & \\ \cline{3-6}
 & & Ports & 0 & 2 500 \euro/port & 0 \euro & \\
 \hline
 \multicolumn{6}{|| c |}{Total Network Cost} & 239 405 900 \euro \\
\hline
\end{tabular}
\caption{Table with detailed description of CAPEX of Vasco's 2016 results.}
\label{scriptopaque_protec_ref_medium_heuristic}
\end{table}

\noindent
\textbf{High Traffic Scenario:}\\

In this scenario we have to take into account the traffic calculated in \ref{high_traffic_scenario}. In a first phase we will show the various existing topologies of the network. The first are the allowed topologies, physical and optical topologies, the second are the logical topology for all ODUs and finally the resulting physical topology.\\

\begin{figure}[H]
\centering
\includegraphics[width=13cm]{sdf/heuristic/opaque_protection/figures/allowed_physical}
\caption{Allowed physical topology. The allowed physical topology is defined by the duct and sites in the field. It is assumed that each duct supports up to 1 bidirectional transmission system and each site supports up to 1 node.}
\label{allowed_physical_protec_ref_high_heuristic}
\end{figure}

\begin{figure}[H]
\centering
\includegraphics[width=13cm]{sdf/heuristic/opaque_protection/figures/allowed_optical}
\caption{Allowed optical topology. The allowed optical topology is defined by the transport mode (opaque transport mode in this case). It is assumed that each transmission system supports up to 100 optical channels.}
\label{allowed_optical_protec_ref_high_heuristic}
\end{figure}

\begin{figure}[H]
\centering
\includegraphics[width=13cm]{sdf/heuristic/opaque_protection/figures/logical_topology_odu0_high}
\caption{ODU0 logical topology defined by the ODU0 traffic matrix.}
\label{logical_ODU0_protec_ref_high_heuristic}
\end{figure}

\begin{figure}[H]
\centering
\includegraphics[width=13cm]{sdf/heuristic/opaque_protection/figures/logical_topology_odu1_high}
\caption{ODU1 logical topology defined by the ODU1 traffic matrix.}
\label{logical_ODU1_protec_ref_high_heuristic}
\end{figure}

\begin{figure}[H]
\centering
\includegraphics[width=13cm]{sdf/heuristic/opaque_protection/figures/logical_topology_odu2_high}
\caption{ODU2 logical topology defined by the ODU2 traffic matrix.}
\label{logical_ODU2_protec_ref_high_heuristic}
\end{figure}

\begin{figure}[H]
\centering
\includegraphics[width=13cm]{sdf/heuristic/opaque_protection/figures/logical_topology_odu3_high}
\caption{ODU3 logical topology defined by the ODU3 traffic matrix.}
\label{logical_ODU3_protec_ref_high_heuristic}
\end{figure}

\begin{figure}[H]
\centering
\includegraphics[width=13cm]{sdf/heuristic/opaque_protection/figures/logical_topology_odu4_high}
\caption{ODU4 logical topology defined by the ODU4 traffic matrix.}
\label{logical_ODU4_protec_ref_high_heuristic}
\end{figure}

\begin{figure}[H]
\centering
\includegraphics[width=13cm]{sdf/heuristic/opaque_protection/figures/physical_topology}
\caption{Physical topology after dimensioning.}
\label{physical_topology_protec_ref_high_heuristic}
\end{figure}

\begin{figure}[H]
\centering
\includegraphics[width=13cm]{sdf/heuristic/opaque_protection/figures/optical_topology_high}
\caption{Optical topology after dimensioning.}
\label{optical_topology_protec_ref_high_heuristic}
\end{figure}

Following all the steps mentioned in the \ref{net2plan_guide}, applying the routing and grooming heuristic algorithms in the Net2Plan software and using all the data referring to this scenario, the obtained result for the Vasco's heuristics can be consulted in the following table \ref{scriptopaque_protec_ref_high_heuristic}. In table \ref{formulas_opaque_heuristic} mentioned in previous model we can see how all the values were calculated. \\

\begin{table}[H]
\centering
\begin{tabular}{|| c | c | c | c | c | c | c ||}
 \hline
 \multicolumn{7}{|| c ||}{CAPEX of the Network} \\
 \hline
 \hline
 \multicolumn{3}{|| c |}{ } & Quantity & Unit Price & Cost & Total \\
 \hline
 \multirow{3}{*}{\makecell{Link \\ Cost}} & \multicolumn{2}{ c |}{OLTs} & 16 & 15 000 \euro & 240 000 \euro & \multirow{3}{*}{397 520 000 \euro} \\ \cline{2-6}
 & \multicolumn{2}{ c |}{100 Gbits/s Transceivers} & 794 & 5 000 \euro/Gbit/s & 397 000 000 \euro & \\ \cline{2-6}
 & \multicolumn{2}{ c |}{Amplifiers} & 70 & 4 000 \euro & 280 000 \euro & \\
 \hline
 \multirow{9}{*}{\makecell{Node \\ Cost}} & \multirow{7}{*}{Electrical} & EXCs & 6 & 10 000 \euro & 60 000 \euro & \multirow{9}{*}{79 514 200 \euro} \\ \cline{3-6}
  & & ODU0 Ports & 1 200 & 10 \euro/port & 12 000 \euro & \\ \cline{3-6}
 & & ODU1 Ports & 1 000 & 15 \euro/port & 15 000 \euro & \\ \cline{3-6}
 & & ODU2 Ports & 320 & 30 \euro/port & 9 600 \euro & \\ \cline{3-6}
 & & ODU3 Ports & 120 & 60 \euro/port & 7 200 \euro & \\ \cline{3-6}
 & & ODU4 Ports & 80 & 100 \euro/port & 8 000 \euro & \\ \cline{3-6}
 & & Line Ports & 794 & 100 000 \euro/port & 99 400 000 \euro & \\ \cline{3-6}
 & \multirow{2}{*}{Optical} & OXCs & 0 & 20 000 \euro & 0 \euro & \\ \cline{3-6}
 & & Ports & 0 & 2 500 \euro/port & 0 \euro & \\
 \hline
 \multicolumn{6}{|| c |}{Total Network Cost} & 477 034 200 \euro \\
\hline
\end{tabular}
\caption{Table with detailed description of CAPEX of Vasco's 2016 results.}
\label{scriptopaque_protec_ref_high_heuristic}
\end{table}

\vspace{13pt}

\subsubsection{Conclusions}

Once we have obtained the results for all the scenarios for opaque without survivability and opaque with 1+1 protection we will now draw some conclusions about these results. For a better analysis of the results will be created the table \ref{table_comparative_opaque_protec_heuristic} with the number of line ports, tributary ports and transceivers because they are important values for the cost of CAPEX, the cost of links, the cost of nodes and finally the cost of CAPEX.\\

\begin{table}[H]
\centering
\begin{tabular}{| c | c | c | c |}
 \hline
 & Low Traffic & Medium Traffic & High Traffic \\
 \hline\hline
 \makecell{CAPEX \\ without survivability} & 14 382 590 \euro & 92 405 900 \euro & 178 834 200 \euro \\ \hline
 \makecell{CAPEX/Gbit/s \\ without survivability} & 28 765 \euro/Gbit/s & 18 481 \euro/Gbit/s & 17 883 \euro/Gbit/s \\ \hline
 Traffic (Gbit/s) & 500 & 5 000 & 10 000 \\ \hline
 Bidirectional Links used & 8 & 8 & 8 \\ \hline
 Number of Line ports & 46 & 398 & 794 \\ \hline
 Number of Tributary ports & 136 & 1 360 & 2 720 \\ \hline
 Number of Transceivers & 46 & 398 & 794 \\ \hline
 Link Cost & 23 520 000 \euro & 199 520 000 \euro & 397 520 000 \euro \\ \hline
 Node Cost & 4 662 590 \euro & 39 885 900 \euro & 79 514 200 \euro \\ \hline
 CAPEX & \textbf{28 182 590 \euro} & \textbf{239 405 900 \euro} & \textbf{477 034 200 \euro} \\ \hline
 CAPEX/Gbit/s & \textbf{56 365 \euro/Gbit/s} & \textbf{47 881 \euro/Gbit/s} & \textbf{47 703 \euro/Gbit/s} \\ \hline
\end{tabular}
\caption{Table with different value of CAPEX for this case.}
\label{table_comparative_opaque_protec_heuristic}
\end{table}

Looking at the previous table we can make some comparisons between the opaque with 1+1 protection scenario:

\begin{itemize}
  \item Comparing the low traffic with the others we can see that despite having an increase of factor ten (medium traffic) and factor twenty (high traffic), the same increase does not occur in the final cost (it is lower);
  \subitem This happens because the number of the transceivers is lower than expected which leads by carrying the traffic with less network components and, consequently, the network CAPEX is lower;
  \item Comparing the medium traffic with the high traffic we can see that the increase of the factor is double and in the final cost this factor is very close but still inferior;
  \subitem This happens because the number of the transceivers is also lower but very close to the expected;
  \item Comparing the CAPEX cost per bit we can see that in the low traffic the cost is higher than the medium and high traffic, which in these two cases the value is very similar;
  \subitem This happens because the lower the traffic, the higher CAPEX/bit will be. We can see that in medium and high traffic the results tend to be one closer value.
\end{itemize}

We can also make some comparisons between the opaque without survivability and opaque with 1+1 protection scenarios:

\begin{itemize}
  \item We can see that in the opaque with 1+1 protection the CAPEX cost for all the three traffic is more than the double;
    \subitem This happens because in the opaque with 1+1 protection there is a need of having a primary and a backup path, in case of a network failure, and the backup path is typically longer and normally uses more than the double of the capacity of the primary;
  \item The number of the network components and the CAPEX cost are directly proportional to the traffic value. The higher the traffic value, the higher the network CAPEX cost;
  \subitem This happens because if the traffic value is higher, the network components have to be in more quantity to carry all the traffic end-to-end, both in the primary and backup paths;
  \item Comparing the CAPEX cost per bit we can see that has a similar case in both of the two scenarios. In the low traffic the cost is higher than the medium and high traffic, which in these two cases the value is very similar;
  \subitem This happens because the lower the traffic, the higher CAPEX/bit will be. We can see that in medium and high traffic the results tend to be one closer value.
\end{itemize}

\vspace{13pt}

\subsubsection{Opens Issues}

The creation of this model for any scenario, started with some considerations and some open issues being:

\begin{itemize}
  \item Allow blocking.
  \subitem The presented model assume that the solution is possible or impossible, does not support a partial solution where some demands are not routed (are blocked);
  \item Allow multiple transmission system.
  \subitem The presented model for each link only supports one transmission system;
  \item Allowing multi-path routing.
  \subitem In the presented model all demands sharing the same end nodes have to follow the same path.
\end{itemize} 
\clearpage

\subsection{Transparent without Survivability}\label{heuristic_Transp_Survivability}
\begin{tcolorbox}	
\begin{tabular}{p{2.75cm} p{0.2cm} p{10.5cm}} 	
\textbf{Student Name}  &:& Pedro Coelho    (01/03/2018 - )\\
\textbf{Goal}          &:& Implement the heuristic model for the transparent transport mode without survivability.
\end{tabular}
\end{tcolorbox}

\subsubsection{Model description}

In the transparent transport mode (single-hop approach), the signals travel through the network in the optical domain between lightpaths. One advantage of this transport mode is that these networks require optical switching. This enables the realization of ongoing optical connections throughout several links without OEO (optical-electrical-optical) conversions. However, there are performed some conversions in some intermediate nodes.\\
Transparent optical connections creates lightpaths which require the assignment of a wavelength that will be used to be exchanged by wavelength converters in order to optimize the network and minimize the total CAPEX.\\
After the creation of the matrices and the network topology, it is necessary to apply the routing and grooming algorithms created. In the end, a report algorithm will be applied to obtain the best CAPEX result for the network in question.\\
We also must take into account the following particularity of this mode of transport:
\begin{itemize}
  \item $N_{OXC,n}$ = 1, \quad $\forall$ n that process traffic
  \item $N_{EXC,n}$ = 1, \quad $\forall$ n that process traffic
\end{itemize}

The minimization of the network CAPEX is made through the equation \ref{Capex_heuristic} where in this case for the cost of nodes we have in consideration the electric cost \ref{Capex_Node_EXC_heuristic} and the optical cost \ref{Capex_Node_OXC_heuristic}.\\

In this case the value of $P_{exc,c,n}$ is obtained by equation \ref{EXC_pexc_transparent_heuristic} for short-reach and by the equation \ref{EXC_pexc2_transparent_heuristic} for long-reach and the value of $P_{oxc,n}$ is obtained by equation \ref{OXC_poxc_transparent_heuristic}.\\

The equation \ref{EXC_pexc_transparent_heuristic} refers to the number of short-reach ports of the electrical switch with bit-rate $c$ in node $n$, $P_{exc,c,n}$, i.e. the number of tributary ports with bit-rate $c$ in node $n$ which can be calculated as

\begin{equation}
P_{exc,c,n} = \sum_{d=1}^{N} D_{nd,c}
\label{EXC_pexc_transparent_heuristic}
\end{equation}

\noindent
where $D_{nd,c}$ are the client demands between nodes $n$ and $d$ with bit rate $c$.\\

\noindent
In this case there is the following particularity:

\begin{itemize}
  \item When $n$=$d$ the value of client demands is always zero, i.e, $D_{nn,c}=0$
\end{itemize}

As previously mentioned, the equation \ref{EXC_pexc2_transparent_heuristic} refers to the number of long-reach ports of the electrical switch with bit-rate -1 in node n, $P_{exc,-1,n}$, i.e. the number of add ports of node n which can be calculated as

\begin{equation}
P_{exc,-1,n} = \sum_{j=1}^{N} f_{nj}^{od}
\label{EXC_pexc2_transparent_heuristic}
\end{equation}

\noindent
where $f_{nj}^{od}$ is the number of optical channels between node $n$ and node $j$ for all demand pairs (od).

\vspace{11pt}
The equation \ref{OXC_poxc_transparent_heuristic} refers to the number of line ports and the number of adding ports of node $n$ which can be calculated as

\begin{equation}
P_{oxc,n} = \sum_{j=1}^{N} 2 f_{nj}^{od} + \sum_{j=1}^{N} \lambda_{nj}
\label{OXC_poxc_transparent_heuristic}
\end{equation}

\noindent
where $f_{nj}^{od}$ refers to the number of line ports for all demand pairs (od) and $\lambda_{nj}$ refers to the number of adding ports.\\

\noindent
The function, to be minimized, is the expression \ref{Minimize_Heuristic_CAPEX}.\\

\vspace{11pt}
This heuristic approach consists in four algorithms made in Java in a programming software called Eclipse and testing them in an open-source network program called Net2Plan. In the Net2Plan guide section \ref{net2plan_guide} there is a brevious explanation of the algorithms and a demonstration on how to use and test them in this network planner. The logical topology and the grooming algorithms are going to be explained below with more details because they have more substantial differences between the three transport modes used in this report.

\subsubsection{Logical topology}
\vspace{11pt}
A network topology represents how the links and the nodes of the network interconnect with each other. These connections are made in the physical (real) and the logical (virtual) topologies and the algorithm creates the logical topology on another layer. The final and resulting physical and optical topologies depend on the method used in each of the transport modes.

In the transparent transport mode each node connects to each other creating direct links between all nodes in the network. Going through all nodes, if a node has a different index from other node, then creates a shortest and direct link between them. These additions of links between nodes are made in the new upper layer of the network. The respective demands are saved in the new upper layer and those demands from the lower layer are then removed. The lower layer is the physical layer of the network and it is now created a new upper layer which is the logical layer of the network and represents the logical topology of the transparent transport mode.

The allowed topologies, physical and optical topologies, the logical topologies for all ODUs and the resulting physical topology is shown in the next section below \ref{result_description_opaque_heuristic_surv} for the three traffic scenarios.

\subsubsection{Grooming}
\vspace{11pt}
After a network topology is created, it is now time to set the grooming algorithm. This algorithm aggregates the traffic into the network, i.e., in the optical channels that are interconnected between end nodes. This aggregation is made by creating bidirectional routes based on the shortest path type (hops or km). In this report it is used the shortest path type in hops.

In the transparent transport mode the grooming algorithm is similar with the one used in opaque transport mode. It starts with going through all the nodes which have different index between them (end nodes), create routes (primary paths) and then set the traffic into those routes. In all direct links between end nodes is reserved a link capacity based on the previous traffic aggregation. One important aspect is that these routes are created based on the shortest path method, comparing all the routes with each other and the final resulting route is the shortest one. As we also have a dedicated 1+1 protection scheme, if the network has this feature, the algorithm will compare the routes again and it will create new routes (backup paths) if they are the next shortest path routes comparing to the previous ones and then set the traffic into those routes. The traffic that is carried into the network will be the sum of all the created paths. Knowing the value of the total traffic and the wavelength capacity of all links, it is possible to calculate the number of wavelengths in each link.

It is shown in the next page below a fluxogram with the description of the algorithms and the steps performed to obtain the final results in the opaque transport mode.

\begin{figure}[H]
\centering
\includegraphics[width=15cm]{sdf/heuristic/transparent_survivability/figures/transparent_fluxogram}
\caption{Fluxogram with transparent transport mode approach.}
\label{transparent_fluxogram}
\end{figure}

\subsubsection{Result description}

It is already known all the necessary formulas to obtain the CAPEX value for the reference network \ref{Reference_Network_Topology}. As described in the subsection of the network traffic \ref{Reference_Network_Traffic}, it is necessary to obtain three different values of CAPEX for the low (0.5 Tbit/s), medium (5 Tbit/s) and high (10 Tbit/s) traffic. It is used a network software program called Net2Plan which can design the traffic matrices, create all the network topologies, simulate the algorithms into the network implemented in the programming software called Eclipse and analyze the results obtained.\\
In this chapter will be demonstrated the results by Vasco's heuristics from 2016. In each of the three traffic scenarios, it will be shown the network topologies followed by the table with the CAPEX value of the network.\\

\textbf{Low Traffic Scenario:}\\

In this scenario we have to take into account the traffic calculated in \ref{low_scenario}. In a first phase we will show the various existing topologies of the network. The first are the allowed topologies, physical and optical topologies, the second are the logical topology for all ODUs and finally the resulting physical topology.\\

\begin{figure}[H]
\centering
\includegraphics[width=13cm]{sdf/heuristic/transparent_survivability/figures/allowed_physical}
\caption{Allowed physical topology. The allowed physical topology is defined by the duct and sites in the field. It is assumed that each duct supports up to 1 bidirectional transmission system and each site supports up to 1 node.}
\label{allowed_physical_surv_ref_low_heuristic_transparent}
\end{figure}

\begin{figure}[H]
\centering
\includegraphics[width=13cm]{sdf/heuristic/transparent_survivability/figures/allowed_optical}
\caption{Allowed optical topology. The allowed optical topology is defined by the transport mode (transparent transport mode in this case). It is assumed that each connections between demands supports up to 100 lightpaths.}
\label{allowed_optical_surv_ref_low_heuristic_transparent}
\end{figure}

\begin{figure}[H]
\centering
\includegraphics[width=13cm]{sdf/heuristic/transparent_survivability/figures/logical_topology_odu0_low}
\caption{ODU0 logical topology defined by the ODU0 traffic matrix.}
\label{logical_ODU0_surv_ref_low_heuristic_transparent}
\end{figure}

\begin{figure}[H]
\centering
\includegraphics[width=13cm]{sdf/heuristic/transparent_survivability/figures/logical_topology_odu1_low}
\caption{ODU1 logical topology defined by the ODU1 traffic matrix.}
\label{logical_ODU1_surv_ref_low_heuristic_transparent}
\end{figure}

\begin{figure}[H]
\centering
\includegraphics[width=13cm]{sdf/heuristic/transparent_survivability/figures/logical_topology_odu2_low}
\caption{ODU2 logical topology defined by the ODU2 traffic matrix.}
\label{logical_ODU2_surv_ref_low_heuristic_transparent}
\end{figure}

\begin{figure}[H]
\centering
\includegraphics[width=13cm]{sdf/heuristic/transparent_survivability/figures/logical_topology_odu3_low}
\caption{ODU3 logical topology defined by the ODU3 traffic matrix.}
\label{logical_ODU3_surv_ref_low_heuristic_transparent}
\end{figure}

\begin{figure}[H]
\centering
\includegraphics[width=13cm]{sdf/heuristic/transparent_survivability/figures/logical_topology_odu4_low}
\caption{ODU4 logical topology defined by the ODU4 traffic matrix.}
\label{logical_ODU4_surv_ref_low_heuristic_transparent}
\end{figure}

\begin{figure}[H]
\centering
\includegraphics[width=13cm]{sdf/heuristic/transparent_survivability/figures/physical_topology}
\caption{Physical topology after dimensioning.}
\label{physical_topology_surv_ref_low_heuristic_transparent}
\end{figure}

Following all the steps mentioned in the \ref{net2plan_guide}, applying the routing and grooming heuristic algorithms in the Net2Plan software and using all the data referring to this scenario, the obtained result for the Vasco's heuristics can be consulted in the following table \ref{scripttransp_surv_ref_low_heuristic}.\\

\begin{table}[H]
\centering
\begin{tabular}{|| c | c | c | c | c | c | c ||}
 \hline
 \multicolumn{7}{|| c ||}{CAPEX of the Network} \\
 \hline
 \hline
 \multicolumn{3}{|| c |}{ } & Quantity & Unit Price & Cost & Total \\
 \hline
 \multirow{3}{*}{\makecell{Link \\ Cost}} & \multicolumn{2}{ c |}{OLTs} & 16 & 15 000 \euro & 240 000 \euro & \multirow{3}{*}{52 520 000 \euro} \\ \cline{2-6}
 & \multicolumn{2}{ c |}{100 Gbits/s Transceivers} & 52 & 5 000 \euro/Gbit/s & 26 000 000 \euro & \\ \cline{2-6}
 & \multicolumn{2}{ c |}{Amplifiers} & 70 & 4 000 \euro & 280 000 \euro & \\
 \hline
 \multirow{10}{*}{\makecell{Node \\ Cost}} & \multirow{7}{*}{Electrical} & EXCs & 6 & 10 000 \euro & 60 000 \euro & \multirow{10}{*}{3 927 590 \euro} \\ \cline{3-6}
  & & ODU0 Ports & 60 & 10 \euro/port & 600 \euro & \\ \cline{3-6}
 & & ODU1 Ports & 50 & 15 \euro/port & 750 \euro & \\ \cline{3-6}
 & & ODU2 Ports & 16 & 30 \euro/port & 480 \euro & \\ \cline{3-6}
 & & ODU3 Ports & 6 & 60 \euro/port & 360 \euro & \\ \cline{3-6}
 & & ODU4 Ports & 4 & 100 \euro/port & 400 \euro & \\ \cline{3-6}
 & & Transponders & 34 & 100 000 \euro/port & 3 400 000 \euro & \\ \cline{2-6}
 & \multirow{3}{*}{Optical} & OXCs & 6 & 20 000 \euro & 120 000 \euro & \\ \cline{3-6}
 & & Line Ports & 104 & 2 500 \euro/port & 260 000 \euro & \\ \cline{3-6}
 & & Add Ports & 34 & 2 500 \euro/port & 85 000 \euro & \\
 \hline
 \multicolumn{6}{|| c |}{Total Network Cost} & 56 447 590 \euro \\
\hline
\end{tabular}
\caption{Table with detailed description of CAPEX of Vasco's 2016 results.}
\label{scripttransp_surv_ref_low_heuristic}
\end{table}

\vspace{17pt}
All the values calculated in the previous table were obtained through the equations \ref{Capex_Link_heuristic} and \ref{Capex_Node_heuristic} referred to in section \ref{Heuristic_CAPEX}, but for a more detailed analysis we created table \ref{formulas_transp_heuristic} where we can see how all the parameters are calculated individually. \\

\begin{table}[h!]
\centering
\begin{tabular}{|| c | c ||}
 \hline
  & Equation used to calculate the cost \\ \hline
 OLTs & \(\displaystyle 2 \sum_{i=1}^{N}\sum_{j=i+1}^{N} L_{ij} \gamma_0^{OLT} \) \\ \hline
 Transceivers & \(\displaystyle 2 \sum_{i=1}^{N}\sum_{j=i+1}^{N} L_{ij} f_{ij}^{od} \gamma_1^{OLT} \tau \) \\ \hline
 Amplifiers & \(\displaystyle 2 \sum_{i=1}^{N}\sum_{j=i+1}^{N} L_{ij} N^R_{ij} c^R \) \\ \hline
 EXCs & \(\displaystyle \sum_{n=1}^N N_{exc,n} \gamma_{e0} \) \\ \hline
 ODU0 Port & \(\displaystyle \sum_{n=1}^{N} \sum_{d=1}^{N} N_{exc,n} D_{nd,0} \gamma_{e1,0} \) \\ \hline
 ODU1 Port & \(\displaystyle \sum_{n=1}^{N} \sum_{d=1}^{N} N_{exc,n} D_{nd,1} \gamma_{e1,1} \) \\ \hline
 ODU2 Port & \(\displaystyle \sum_{n=1}^{N} \sum_{d=1}^{N} N_{exc,n} D_{nd,2} \gamma_{e1,2} \)\\ \hline
 ODU3 Port & \(\displaystyle \sum_{n=1}^{N} \sum_{d=1}^{N} N_{exc,n} D_{nd,3} \gamma_{e1,3} \) \\ \hline
 ODU4 Port & \(\displaystyle \sum_{n=1}^{N} \sum_{d=1}^{N} N_{exc,n} D_{nd,4} \gamma_{e1,4} \) \\ \hline
 LR Transponders & \(\displaystyle \sum_{n=1}^{N} \sum_{j=1}^{N} N_{exc,n} \lambda_{od} \gamma_{e1,-1} \) \\ \hline
 OXCs & \(\displaystyle \sum_{n=1}^N N_{oxc,n} \gamma_{o0} \) \\ \hline
 Add Port & \(\displaystyle \sum_{n=1}^{N} \sum_{j=1}^{N} N_{oxc,n} \lambda_{od} \gamma_{o1} \) \\ \hline
 Line Port & \(\displaystyle \sum_{n=1}^{N} \sum_{j=1}^{N} N_{oxc,n} f_{ij}^{od} \gamma_{o1} \) \\ \hline
 CAPEX & The final cost is calculated by summing all previous results. \\
 \hline
 \end{tabular}
\caption{Table with description of calculation}
\label{formulas_transp_heuristic}
\end{table}

\textbf{Medium Traffic Scenario:}\\

In this scenario we have to take into account the traffic calculated in \ref{medium_traffic_scenario}. In a first phase we will show the various existing topologies of the network. The first are the allowed topologies, physical and optical topologies, the second are the logical topology for all ODUs and finally the resulting physical topology.\\

\begin{figure}[H]
\centering
\includegraphics[width=13cm]{sdf/heuristic/transparent_survivability/figures/allowed_physical}
\caption{Allowed physical topology. The allowed physical topology is defined by the duct and sites in the field. It is assumed that each duct supports up to 1 bidirectional transmission system and each site supports up to 1 node.}
\label{allowed_physical_surv_ref_medium_heuristic_transparent}
\end{figure}

\begin{figure}[H]
\centering
\includegraphics[width=13cm]{sdf/heuristic/transparent_survivability/figures/allowed_optical}
\caption{Allowed optical topology. The allowed optical topology is defined by the transport mode (transparent transport mode in this case). It is assumed that each connections between demands supports up to 100 lightpaths.}
\label{allowed_optical_surv_ref_medium_heuristic_transparent}
\end{figure}

\begin{figure}[H]
\centering
\includegraphics[width=13cm]{sdf/heuristic/transparent_survivability/figures/logical_topology_odu0_medium}
\caption{ODU0 logical topology defined by the ODU0 traffic matrix.}
\label{logical_ODU0_surv_ref_medium_heuristic_transparent}
\end{figure}

\begin{figure}[H]
\centering
\includegraphics[width=13cm]{sdf/heuristic/transparent_survivability/figures/logical_topology_odu1_medium}
\caption{ODU1 logical topology defined by the ODU1 traffic matrix.}
\label{logical_ODU1_surv_ref_medium_heuristic_transparent}
\end{figure}

\begin{figure}[H]
\centering
\includegraphics[width=13cm]{sdf/heuristic/transparent_survivability/figures/logical_topology_odu2_medium}
\caption{ODU2 logical topology defined by the ODU2 traffic matrix.}
\label{logical_ODU2_surv_ref_medium_heuristic_transparent}
\end{figure}

\begin{figure}[H]
\centering
\includegraphics[width=13cm]{sdf/heuristic/transparent_survivability/figures/logical_topology_odu3_medium}
\caption{ODU3 logical topology defined by the ODU3 traffic matrix.}
\label{logical_ODU3_surv_ref_medium_heuristic_transparent}
\end{figure}

\begin{figure}[H]
\centering
\includegraphics[width=13cm]{sdf/heuristic/transparent_survivability/figures/logical_topology_odu4_medium}
\caption{ODU4 logical topology defined by the ODU4 traffic matrix.}
\label{logical_ODU4_surv_ref_medium_heuristic_transparent}
\end{figure}

\begin{figure}[H]
\centering
\includegraphics[width=13cm]{sdf/heuristic/transparent_survivability/figures/physical_topology}
\caption{Physical topology after dimensioning.}
\label{physical_topology_surv_ref_medium_heuristic_transparent}
\end{figure}

Following all the steps mentioned in the \ref{net2plan_guide}, applying the routing and grooming heuristic algorithms in the Net2Plan software and using all the data referring to this scenario, the obtained result for the Vasco's heuristics can be consulted in the following table \ref{scripttransp_surv_ref_medium_heuristic}. In table \ref{formulas_transp_heuristic} mentioned in previous scenario we can see how all the values were calculated. \\

\begin{table}[H]
\centering
\begin{tabular}{|| c | c | c | c | c | c | c ||}
 \hline
 \multicolumn{7}{|| c ||}{CAPEX of the Network} \\
 \hline
 \hline
 \multicolumn{3}{|| c |}{ } & Quantity & Unit Price & Cost & Total \\
 \hline
 \multirow{3}{*}{\makecell{Link \\ Cost}} & \multicolumn{2}{ c |}{OLTs} & 16 & 15 000 \euro & 240 000 \euro & \multirow{3}{*}{168 520 000 \euro} \\ \cline{2-6}
 & \multicolumn{2}{ c |}{100 Gbits/s Transceivers} & 168 & 5 000 \euro/Gbit/s & 84 000 000 \euro & \\ \cline{2-6}
 & \multicolumn{2}{ c |}{Amplifiers} & 70 & 4 000 \euro & 280 000 \euro & \\
 \hline
 \multirow{10}{*}{\makecell{Node \\ Cost}} & \multirow{7}{*}{Electrical} & EXCs & 6 & 10 000 \euro & 60 000 \euro & \multirow{10}{*}{12 730 900 \euro} \\ \cline{3-6}
 & & ODU0 Ports & 600 & 10 \euro/port & 6 000 \euro & \\ \cline{3-6}
 & & ODU1 Ports & 500 & 15 \euro/port & 7 500 \euro & \\ \cline{3-6}
 & & ODU2 Ports & 160 & 30 \euro/port & 4 800 \euro & \\ \cline{3-6}
 & & ODU3 Ports & 60 & 60 \euro/port & 3 600 \euro & \\ \cline{3-6}
 & & ODU4 Ports & 40 & 100 \euro/port & 4 000 \euro & \\ \cline{3-6}
 & & Transponders & 114 & 100 000 \euro/port & 11 400 000 \euro & \\ \cline{2-6}
 & \multirow{3}{*}{Optical} & OXCs & 6 & 20 000 \euro & 120 000 \euro & \\ \cline{3-6}
 & & Line Ports & 336 & 2 500 \euro/port & 840 000 \euro & \\ \cline{3-6}
 & & Add Ports & 114 & 2 500 \euro/port & 285 000 \euro & \\
 \hline
 \multicolumn{6}{|| c |}{Total Network Cost} & 181 250 900 \euro \\
\hline
\end{tabular}
\caption{Table with detailed description of CAPEX of Vasco's 2016 results.}
\label{scripttransp_surv_ref_medium_heuristic}
\end{table}

\textbf{High Traffic Scenario:}\\

In this scenario we have to take into account the traffic calculated in \ref{high_traffic_scenario}. In a first phase we will show the various existing topologies of the network. The first are the allowed topologies, physical and optical topologies, the second are the logical topology for all ODUs and finally the resulting physical topology.\\

\begin{figure}[H]
\centering
\includegraphics[width=13cm]{sdf/heuristic/transparent_survivability/figures/allowed_physical}
\caption{Allowed physical topology. The allowed physical topology is defined by the duct and sites in the field. It is assumed that each duct supports up to 1 bidirectional transmission system and each site supports up to 1 node.}
\label{allowed_physical_surv_ref_high_heuristic_transparent}
\end{figure}

\begin{figure}[H]
\centering
\includegraphics[width=13cm]{sdf/heuristic/transparent_survivability/figures/allowed_optical}
\caption{Allowed optical topology. The allowed optical topology is defined by the transport mode (transparent transport mode in this case). It is assumed that each connections between demands supports up to 100 lightpaths.}
\label{allowed_optical_surv_ref_high_heuristic_transparent}
\end{figure}

\begin{figure}[H]
\centering
\includegraphics[width=13cm]{sdf/heuristic/transparent_survivability/figures/logical_topology_odu0_high}
\caption{ODU0 logical topology defined by the ODU0 traffic matrix.}
\label{logical_ODU0_surv_ref_high_heuristic_transparent}
\end{figure}

\begin{figure}[H]
\centering
\includegraphics[width=13cm]{sdf/heuristic/transparent_survivability/figures/logical_topology_odu1_high}
\caption{ODU1 logical topology defined by the ODU1 traffic matrix.}
\label{logical_ODU1_surv_ref_high_heuristic_transparent}
\end{figure}

\begin{figure}[H]
\centering
\includegraphics[width=13cm]{sdf/heuristic/transparent_survivability/figures/logical_topology_odu2_high}
\caption{ODU2 logical topology defined by the ODU2 traffic matrix.}
\label{logical_ODU2_surv_ref_high_heuristic_transparent}
\end{figure}

\begin{figure}[H]
\centering
\includegraphics[width=13cm]{sdf/heuristic/transparent_survivability/figures/logical_topology_odu3_high}
\caption{ODU3 logical topology defined by the ODU3 traffic matrix.}
\label{logical_ODU3_surv_ref_high_heuristic_transparent}
\end{figure}

\begin{figure}[H]
\centering
\includegraphics[width=13cm]{sdf/heuristic/transparent_survivability/figures/logical_topology_odu4_high}
\caption{ODU4 logical topology defined by the ODU4 traffic matrix.}
\label{logical_ODU4_surv_ref_high_heuristic_transparent}
\end{figure}

\begin{figure}[H]
\centering
\includegraphics[width=13cm]{sdf/heuristic/transparent_survivability/figures/physical_topology}
\caption{Physical topology after dimensioning.}
\label{physical_topology_surv_ref_high_heuristic_transparent}
\end{figure}

Following all the steps mentioned in the \ref{net2plan_guide}, applying the routing and grooming heuristic algorithms in the Net2Plan software and using all the data referring to this scenario, the obtained result for the Vasco's heuristics can be consulted in the following table \ref{scripttransp_surv_ref_high_heuristic}. In table \ref{formulas_transp_heuristic} mentioned in previous scenario we can see how all the values were calculated. \\

\begin{table}[H]
\centering
\begin{tabular}{|| c | c | c | c | c | c | c ||}
 \hline
 \multicolumn{7}{|| c ||}{CAPEX of the Network} \\
 \hline
 \hline
 \multicolumn{3}{|| c |}{ } & Quantity & Unit Price & Cost & Total \\
 \hline
 \multirow{3}{*}{\makecell{Link \\ Cost}} & \multicolumn{2}{ c |}{OLTs} & 16 & 15 000 \euro & 240 000 \euro & \multirow{3}{*}{314 520 000 \euro} \\ \cline{2-6}
 & \multicolumn{2}{ c |}{100 Gbits/s Transceivers} & 314 & 5 000 \euro/Gbit/s & 157 000 000 \euro & \\ \cline{2-6}
 & \multicolumn{2}{ c |}{Amplifiers} & 70 & 4 000 \euro & 280 000 \euro & \\
 \hline
 \multirow{10}{*}{\makecell{Node \\ Cost}} & \multirow{7}{*}{Electrical} & EXCs & 6 & 10 000 \euro & 60 000 \euro & \multirow{10}{*}{23 736 800 \euro} \\ \cline{3-6}
  & & ODU0 Ports & 1 200 & 10 \euro/port & 12 000 \euro & \\ \cline{3-6}
 & & ODU1 Ports & 1 000 & 15 \euro/port & 15 000 \euro & \\ \cline{3-6}
 & & ODU2 Ports & 320 & 30 \euro/port & 9 600 \euro & \\ \cline{3-6}
 & & ODU3 Ports & 120 & 60 \euro/port & 7 200 \euro & \\ \cline{3-6}
 & & ODU4 Ports & 80 & 100 \euro/port & 8 000 \euro & \\ \cline{3-6}
 & & Transponders & 214 & 100 000 \euro/port & 21 400 000 \euro & \\ \cline{2-6}
 & \multirow{3}{*}{Optical} & OXCs & 6 & 20 000 \euro & 120 000 \euro & \\ \cline{3-6}
 & & Line Ports & 628 & 2 500 \euro/port & 1 570 000 \euro & \\ \cline{3-6}
 & & Add Ports & 214 & 2 500 \euro/port & 535 000 \euro & \\
 \hline
 \multicolumn{6}{|| c |}{Total Network Cost} & 338 256 800 \euro \\
\hline
\end{tabular}
\caption{Table with detailed description of CAPEX of Vasco's 2016 results.}
\label{scripttransp_surv_ref_high_heuristic}
\end{table}

\vspace{13pt}

\subsubsection{Conclusions}

Once we have obtained the results for all the scenarios we will now draw some conclusions about these results. For a better analysis of the results will be created the table \ref{table_comparative_transp_surv_heuristic} with the number of line ports, tributary ports and transceivers because they are important values for the cost of CAPEX, the cost of links, the cost of nodes and finally the cost of CAPEX.\\

\begin{table}[H]
\centering
\begin{tabular}{| c | c | c | c |}
 \hline
   & Low Traffic & Medium Traffic  & High Traffic \\
 \hline\hline
 Traffic (Gbit/s) & 500 & 5 000 & 10 000 \\ \hline
 Bidirectional Links used & 8 & 8 & 8 \\ \hline
 Number of Add ports & 34 & 114 & 214 \\ \hline
 Number of Line ports & 104 & 336 & 628 \\ \hline
 Number of Tributary ports & 136 & 1 360 & 2 720 \\ \hline
 Number of Transceivers & 104 & 336 & 628 \\ \hline
 Link Cost & 52 520 000 \euro & 168 520 000 \euro & 314 520 000 \euro \\ \hline
 Node Cost & 3 927 590 \euro & 12 730 900 \euro & 23 736 800 \euro \\ \hline
 CAPEX & \textbf{56 447 590 \euro} & \textbf{181 250 900 \euro} & \textbf{338 256 800 \euro} \\ \hline
 CAPEX/Gbit/s & \textbf{112 895 \euro/Gbit/s} & \textbf{36 250 \euro/Gbit/s} & \textbf{33 825 \euro/Gbit/s} \\ \hline
\end{tabular}
\caption{Table with different value of CAPEX for this case.}
\label{table_comparative_transp_surv_heuristic}
\end{table}

\noindent
Looking at the previous table we can make some comparisons between the several scenario:

\begin{itemize}
  \item Comparing the low traffic with the others we can see that despite having an increase of factor ten (medium traffic) and factor twenty (high traffic), the same increase does not occur in the final cost (it is lower);
  \subitem This happens because the number of the transceivers is lower than expected which leads by carrying the traffic with less network components and, consequently, the network CAPEX is lower;
  \item Comparing the medium traffic with the high traffic we can see that the increase of the factor is double and in the final cost this factor is very close but still inferior;
  \subitem This happens because the number of the transceivers is also lower but very close to the expected;
  \item Comparing the CAPEX cost per bit we can see that in the low traffic the cost is higher than the medium and high traffic, which in these two cases the value is similar, but still inferior in the higher traffic;
  \subitem This happens because the lower the traffic, the higher CAPEX/bit will be. We can see that in medium and high traffic the results tend to be one closer and lower value.
\end{itemize}

\vspace{13pt}
\subsubsection{Opens Issues}

The creation of this model for any scenario, started with some considerations and some open issues being:

\begin{itemize}
  \item Allow blocking.
  \subitem The presented model assume that the solution is possible or impossible, does not support a partial solution where some demands are not routed (are blocked);
  \item Allow multiple transmission system.
  \subitem The presented model for each link only supports one transmission system.
\end{itemize}

\clearpage

\subsection{Transparent with 1+1 Protection}\label{heuristic_Transp_Protection}
\begin{tcolorbox}	
\begin{tabular}{p{2.75cm} p{0.2cm} p{10.5cm}} 	
\textbf{Student Name}  &:& Pedro Coelho    (March 01, 2018 - )\\
\textbf{Goal}          &:& Implement the heuristic model for the transparent transport mode with 1 plus 1 protection.
\end{tabular}
\end{tcolorbox}

\subsubsection{Model description}

\vspace{11pt}
Contrary to the transparent without survivability transport mode, the transparent with 1+1 protection technique has a backup path, so if there is a network failure it is more likely to not suffer large data losses. The backup paths are always different from the primary ones and they prevent that the information going through optical channels could be lost in these occasions. However, the CAPEX will be significantly higher (more than the double), because that includes a secondary path that will increase several network elements.

After the creation of the matrices and the network topology, it is necessary to apply the routing and grooming algorithms created. In the end, a report algorithm will be applied to obtain the best CAPEX result for the network in question.

We also must take into account the following particularity of this mode of transport:
\begin{itemize}
  \item $N_{OXC,n}$ = 1, \quad $\forall$ n that process traffic
  \item $N_{EXC,n}$ = 1, \quad $\forall$ n that process traffic
\end{itemize}

\vspace{11pt}
The minimization of the network CAPEX is made through the equation \ref{Capex_heuristic} where in this case for the cost of nodes we have in consideration the electric cost \ref{Capex_Node_EXC_heuristic} and the optical cost \ref{Capex_Node_OXC_heuristic}.
In this case the value of $P_{exc,c,n}$ is obtained by equation \ref{EXC_pexc_transparentp_heuristic_protec} and the value of $P_{oxc,n}$ is obtained by equation \ref{OXC_poxc_transparentp_heuristic_protec}.\\

The equation \ref{EXC_pexc_transparentp_heuristic_protec} refers to the number of short-reach ports of the electrical switch with bit-rate $c$ in node $n$, $P_{exc,c,n}$, i.e. the number of tributary ports with bit-rate $c$ in node $n$ which can be calculated as

\begin{equation}
P_{exc,c,n} = 2 \sum_{d=1}^{N} D_{nd,c}
\label{EXC_pexc_transparentp_heuristic_protec}
\end{equation}

\vspace{11pt}
where $D_{nd,c}$ are the client demands between nodes $n$ and $d$ with bit rate $c$.

\vspace{11pt}
In this case there is the following particularity:

\begin{itemize}
  \item When $n$=$d$ the value of client demands is always zero, i.e, $D_{nn,c}=0$
\end{itemize}

\vspace{11pt}
The equation \ref{OXC_poxc_transparentp_heuristic_protec} refers to the number of long-reach ports of the optical switch in node $n$, $P_{oxc,n}$, i.e. the number of line ports and the number of adding ports of node $n$ which can be calculated as

\begin{equation}
P_{oxc,n} = \sum_{j=1}^{N} f_{nj} + \sum_{j=1}^{N} \lambda_{nj}
\label{OXC_poxc_transparentp_heuristic_protec}
\end{equation}

\vspace{11pt}
where $f_{nj}$ refers to the number of line ports and $\lambda_{nj}$ refers to the number of adding ports.

\vspace{17pt}
The function, to be minimized, is the expression \ref{Minimize_Heuristic_CAPEX}.\\

\subsubsection{Result description}

It is already known all the necessary formulas to obtain the CAPEX value for the reference network \ref{Reference_Network_Topology}. As described in the subsection of the network traffic \ref{Reference_Network_Traffic}, it is necessary to obtain three different values of CAPEX for the low (0.5 Tbit/s), medium (5 Tbit/s) and high (10 Tbit/s) traffic. It is used a network software program called Net2Plan which can design the traffic matrices, create all the network topologies, simulate the algorithms into the network implemented in the programming software called Eclipse and analyze the results obtained.

In this chapter will be demonstrated the results by Vasco's heuristics from 2016. In each of the three traffic scenarios, it will be shown the network topologies followed by the table with the CAPEX value of the network.\\

\textbf{Low Traffic Scenario:}\\

In this scenario we have to take into account the traffic calculated in \ref{low_scenario}. In a first phase we will show the various existing topologies of the network. The first are the allowed topologies, physical and optical topology, the second are the logical topology for all ODUs and finally the resulting physical and optical topology.\\

\begin{figure}[H]
\centering
\includegraphics[width=13cm]{sdf/heuristic/transparent_protection/low/allowed_physical_low}
\caption{Allowed physical topology. The allowed physical topology is defined by the duct and sites in the field. It is assumed that each duct supports up to 1 bidirectional transmission system and each site supports up to 1 node.}
\label{allowed_physical_protection_ref_low_heuristic_transparent}
\end{figure}

\begin{figure}[H]
\centering
\includegraphics[width=13cm]{sdf/heuristic/transparent_protection/low/allowed_optical_low}
\caption{Allowed optical topology. The allowed optical topology is defined by the transport mode (transparent transport mode in this case). It is assumed that each connections between demands supports up to 100 lightpaths.}
\label{allowed_optical_protection_ref_low_heuristic_transparent}
\end{figure}

\begin{figure}[H]
\centering
\includegraphics[width=13cm]{sdf/heuristic/transparent_protection/low/logical_topology_odu0_low}
\caption{ODU0 logical topology defined by the ODU0 traffic matrix.}
\label{logical_ODU0_protection_ref_low_heuristic_transparent}
\end{figure}

\begin{figure}[H]
\centering
\includegraphics[width=13cm]{sdf/heuristic/transparent_protection/low/logical_topology_odu1_low}
\caption{ODU1 logical topology defined by the ODU1 traffic matrix.}
\label{logical_ODU1_protection_ref_low_heuristic_transparent}
\end{figure}

\begin{figure}[H]
\centering
\includegraphics[width=13cm]{sdf/heuristic/transparent_protection/low/logical_topology_odu2_low}
\caption{ODU2 logical topology defined by the ODU2 traffic matrix.}
\label{logical_ODU2_protection_ref_low_heuristic_transparent}
\end{figure}

\begin{figure}[H]
\centering
\includegraphics[width=13cm]{sdf/heuristic/transparent_protection/low/logical_topology_odu3_low}
\caption{ODU3 logical topology defined by the ODU3 traffic matrix.}
\label{logical_ODU3_protection_ref_low_heuristic_transparent}
\end{figure}

\begin{figure}[H]
\centering
\includegraphics[width=13cm]{sdf/heuristic/transparent_protection/low/logical_topology_odu4_low}
\caption{ODU4 logical topology defined by the ODU4 traffic matrix.}
\label{logical_ODU4_protection_ref_low_heuristic_transparent}
\end{figure}

\begin{figure}[H]
\centering
\includegraphics[width=13cm]{sdf/heuristic/transparent_protection/low/physical_topology_low}
\caption{Physical topology after dimensioning.}
\label{physical_topology_protection_ref_low_heuristic_transparent}
\end{figure}

Following all the steps mentioned in the \ref{net2plan_guide}, applying the routing and grooming heuristic algorithms in the Net2Plan software and using all the data referring to this scenario, the obtained result for the Vasco's heuristics can be consulted in the following table \ref{scripttransp_protec_ref_low_heuristic}.

\begin{table}[H]
\centering
\begin{tabular}{|| c | c | c | c | c | c | c ||}
 \hline
 \multicolumn{7}{|| c ||}{CAPEX of the Network} \\
 \hline
 \hline
 \multicolumn{3}{|| c |}{ } & Quantity & Unit Price & Cost & Total \\
 \hline
 \multirow{3}{*}{Link Cost} & \multicolumn{2}{ c |}{OLTs} & 16 & 15 000 \euro & 240 000 \euro & \multirow{3}{*}{68 520 000 \euro} \\ \cline{2-6}
 & \multicolumn{2}{ c |}{100 Gb/s Transceivers} & 136 & 5 000 \euro/Gb/s & 68 000 000 \euro & \\ \cline{2-6}
 & \multicolumn{2}{ c |}{Amplifiers} & 70 & 4 000 \euro & 280 000 \euro & \\
 \hline
 \multirow{10}{*}{Node Cost} & \multirow{7}{*}{Electrical} & EXCs & 6 & 10 000 \euro & 60 000 \euro & \multirow{10}{*}{14 547 590 \euro} \\ \cline{3-6}
  & & ODU0 Ports & 60 & 10 \euro & 600 \euro & \\ \cline{3-6}
 & & ODU1 Ports & 50 & 15 \euro & 750 \euro & \\ \cline{3-6}
 & & ODU2 Ports & 16 & 30 \euro & 480 \euro & \\ \cline{3-6}
 & & ODU3 Ports & 6 & 60 \euro & 360 \euro & \\ \cline{3-6}
 & & ODU4 Ports & 4 & 100 \euro & 400 \euro & \\ \cline{3-6}
 & & Line Ports & 136 & 1 000 \euro/Gb/s & 13 600 000 \euro & \\ \cline{3-6}
 & \multirow{3}{*}{Optical} & OXCs & 6 & 20 000 \euro & 120 000 \euro & \\ \cline{3-6}
 & & Line Ports & 272 & 2 500 \euro/porto & 680 000 \euro & \\ \cline{3-6}
 & & Add Ports & 34 & 2 500 \euro/porto & 85 000 \euro & \\
 \hline
 \multicolumn{6}{|| c |}{Total Network Cost} & 83 067 590 \euro \\
\hline
\end{tabular}
\caption{Table with detailed description of CAPEX of Vasco's 2016 results.}
\label{scripttransp_protec_ref_low_heuristic}
\end{table}

\textbf{Medium Traffic Scenario:}\\

In this scenario we have to take into account the traffic calculated in \ref{medium_traffic_scenario}. In a first phase we will show the various existing topologies of the network. The first are the allowed topologies, physical and optical topology, the second are the logical topology for all ODUs and finally the resulting physical and optical topology.\\

\begin{figure}[H]
\centering
\includegraphics[width=13cm]{sdf/heuristic/transparent_protection/medium/allowed_physical_medium}
\caption{Allowed physical topology. The allowed physical topology is defined by the duct and sites in the field. It is assumed that each duct supports up to 1 bidirectional transmission system and each site supports up to 1 node.}
\label{allowed_physical_protection_ref_medium_heuristic_transparent}
\end{figure}

\begin{figure}[H]
\centering
\includegraphics[width=13cm]{sdf/heuristic/transparent_protection/medium/allowed_optical_medium}
\caption{Allowed optical topology. The allowed optical topology is defined by the transport mode (transparent transport mode in this case). It is assumed that each connections between demands supports up to 100 lightpaths.}
\label{allowed_optical_protection_ref_medium_heuristic_transparent}
\end{figure}

\begin{figure}[H]
\centering
\includegraphics[width=13cm]{sdf/heuristic/transparent_protection/medium/logical_topology_odu0_medium}
\caption{ODU0 logical topology defined by the ODU0 traffic matrix.}
\label{logical_ODU0_protection_ref_medium_heuristic_transparent}
\end{figure}

\begin{figure}[H]
\centering
\includegraphics[width=13cm]{sdf/heuristic/transparent_protection/medium/logical_topology_odu1_medium}
\caption{ODU1 logical topology defined by the ODU1 traffic matrix.}
\label{logical_ODU1_protection_ref_medium_heuristic_transparent}
\end{figure}

\begin{figure}[H]
\centering
\includegraphics[width=13cm]{sdf/heuristic/transparent_protection/medium/logical_topology_odu2_medium}
\caption{ODU2 logical topology defined by the ODU2 traffic matrix.}
\label{logical_ODU2_protection_ref_medium_heuristic_transparent}
\end{figure}

\begin{figure}[H]
\centering
\includegraphics[width=13cm]{sdf/heuristic/transparent_protection/medium/logical_topology_odu3_medium}
\caption{ODU3 logical topology defined by the ODU3 traffic matrix.}
\label{logical_ODU3_protection_ref_medium_heuristic_transparent}
\end{figure}

\begin{figure}[H]
\centering
\includegraphics[width=13cm]{sdf/heuristic/transparent_protection/medium/logical_topology_odu4_medium}
\caption{ODU4 logical topology defined by the ODU4 traffic matrix.}
\label{logical_ODU4_protection_ref_medium_heuristic_transparent}
\end{figure}

\begin{figure}[H]
\centering
\includegraphics[width=13cm]{sdf/heuristic/transparent_protection/medium/physical_topology_medium}
\caption{Physical topology after dimensioning.}
\label{physical_topology_protection_ref_medium_heuristic_transparent}
\end{figure}

Following all the steps mentioned in the \ref{net2plan_guide}, applying the routing and grooming heuristic algorithms in the Net2Plan software and using all the data referring to this scenario, the obtained result for the Vasco's heuristics can be consulted in the following table \ref{scripttransp_protec_ref_medium_heuristic}.

\begin{table}[H]
\centering
\begin{tabular}{|| c | c | c | c | c | c | c ||}
 \hline
 \multicolumn{7}{|| c ||}{CAPEX of the Network} \\
 \hline
 \hline
 \multicolumn{3}{|| c |}{ } & Quantity & Unit Price & Cost & Total \\
 \hline
 \multirow{3}{*}{Link Cost} & \multicolumn{2}{ c |}{OLTs} & 16 & 15 000 \euro & 240 000 \euro & \multirow{3}{*}{283 520 000 \euro} \\ \cline{2-6}
 & \multicolumn{2}{ c |}{100 Gb/s Transceivers} & 566 & 5 000 \euro/Gb/s & 283 000 000 \euro & \\ \cline{2-6}
 & \multicolumn{2}{ c |}{Amplifiers} & 70 & 4 000 \euro & 280 000 \euro & \\
 \hline
 \multirow{10}{*}{Node Cost} & \multirow{7}{*}{Electrical} & EXCs & 6 & 10 000 \euro & 60 000 \euro & \multirow{10}{*}{59 990 900 \euro} \\ \cline{3-6}
 & & ODU0 Ports & 600 & 10 \euro & 6 000 \euro & \\ \cline{3-6}
 & & ODU1 Ports & 500 & 15 \euro & 7 500 \euro & \\ \cline{3-6}
 & & ODU2 Ports & 160 & 30 \euro & 4 800 \euro & \\ \cline{3-6}
 & & ODU3 Ports & 60 & 60 \euro & 3 600 \euro & \\ \cline{3-6}
 & & ODU4 Ports & 40 & 100 \euro & 4 000 \euro & \\ \cline{3-6}
 & & Line Ports & 566 & 1 000 \euro/Gb/s & 56 600 000 \euro & \\ \cline{3-6}
 & \multirow{3}{*}{Optical} & OXCs & 6 & 20 000 \euro & 120 000 \euro & \\ \cline{3-6}
 & & Line Ports & 1 132 & 2 500 \euro/porto & 2 830 000 \euro & \\ \cline{3-6}
 & & Add Ports & 142 & 2 500 \euro/porto & 355 000 \euro & \\
 \hline
 \multicolumn{6}{|| c |}{Total Network Cost} & 343 510 900 \euro \\
\hline
\end{tabular}
\caption{Table with detailed description of CAPEX of Vasco's 2016 results.}
\label{scripttransp_protec_ref_medium_heuristic}
\end{table}

\textbf{High Traffic Scenario:}\\

In this scenario we have to take into account the traffic calculated in \ref{high_traffic_scenario}. In a first phase we will show the various existing topologies of the network. The first are the allowed topologies, physical and optical topology, the second are the logical topology for all ODUs and finally the resulting physical and optical topology.\\

\begin{figure}[H]
\centering
\includegraphics[width=13cm]{sdf/heuristic/transparent_protection/high/allowed_physical_high}
\caption{Allowed physical topology. The allowed physical topology is defined by the duct and sites in the field. It is assumed that each duct supports up to 1 bidirectional transmission system and each site supports up to 1 node.}
\label{allowed_physical_protection_ref_high_heuristic_transparent}
\end{figure}

\begin{figure}[H]
\centering
\includegraphics[width=13cm]{sdf/heuristic/transparent_protection/high/allowed_optical_high}
\caption{Allowed optical topology. The allowed optical topology is defined by the transport mode (transparent transport mode in this case). It is assumed that each connections between demands supports up to 100 lightpaths.}
\label{allowed_optical_protection_ref_high_heuristic_transparent}
\end{figure}

\begin{figure}[H]
\centering
\includegraphics[width=13cm]{sdf/heuristic/transparent_protection/high/logical_topology_odu0_high}
\caption{ODU0 logical topology defined by the ODU0 traffic matrix.}
\label{logical_ODU0_protection_ref_high_heuristic_transparent}
\end{figure}

\begin{figure}[H]
\centering
\includegraphics[width=13cm]{sdf/heuristic/transparent_protection/high/logical_topology_odu1_high}
\caption{ODU1 logical topology defined by the ODU1 traffic matrix.}
\label{logical_ODU1_protection_ref_high_heuristic_transparent}
\end{figure}

\begin{figure}[H]
\centering
\includegraphics[width=13cm]{sdf/heuristic/transparent_protection/high/logical_topology_odu2_high}
\caption{ODU2 logical topology defined by the ODU2 traffic matrix.}
\label{logical_ODU2_protection_ref_high_heuristic_transparent}
\end{figure}

\begin{figure}[H]
\centering
\includegraphics[width=13cm]{sdf/heuristic/transparent_protection/high/logical_topology_odu3_high}
\caption{ODU3 logical topology defined by the ODU3 traffic matrix.}
\label{logical_ODU3_protection_ref_high_heuristic_transparent}
\end{figure}

\begin{figure}[H]
\centering
\includegraphics[width=13cm]{sdf/heuristic/transparent_protection/high/logical_topology_odu4_high}
\caption{ODU4 logical topology defined by the ODU4 traffic matrix.}
\label{logical_ODU4_protection_ref_high_heuristic_transparent}
\end{figure}

\begin{figure}[H]
\centering
\includegraphics[width=13cm]{sdf/heuristic/transparent_protection/high/physical_topology_high}
\caption{Physical topology after dimensioning.}
\label{physical_topology_protection_ref_high_heuristic_transparent}
\end{figure}

Following all the steps mentioned in the \ref{net2plan_guide}, applying the routing and grooming heuristic algorithms in the Net2Plan software and using all the data referring to this scenario, the obtained result for the Vasco's heuristics can be consulted in the following table \ref{scripttransp_protec_ref_high_heuristic}.

\begin{table}[H]
\centering
\begin{tabular}{|| c | c | c | c | c | c | c ||}
 \hline
 \multicolumn{7}{|| c ||}{CAPEX of the Network} \\
 \hline
 \hline
 \multicolumn{3}{|| c |}{ } & Quantity & Unit Price & Cost & Total \\
 \hline
 \multirow{3}{*}{Link Cost} & \multicolumn{2}{ c |}{OLTs} & 16 & 15 000 \euro & 240 000 \euro & \multirow{3}{*}{530 520 000 \euro} \\ \cline{2-6}
 & \multicolumn{2}{ c |}{100 Gb/s Transceivers} & 1 060 & 5 000 \euro/Gb/s & 530 000 000 \euro & \\ \cline{2-6}
 & \multicolumn{2}{ c |}{Amplifiers} & 70 & 4 000 \euro & 280 000 \euro & \\
 \hline
 \multirow{10}{*}{Node Cost} & \multirow{7}{*}{Electrical} & EXCs & 6 & 10 000 \euro & 60 000 \euro & \multirow{10}{*}{112 201 800 \euro} \\ \cline{3-6}
  & & ODU0 Ports & 1 200 & 10 \euro & 12 000 \euro & \\ \cline{3-6}
 & & ODU1 Ports & 1 000 & 15 \euro & 15 000 \euro & \\ \cline{3-6}
 & & ODU2 Ports & 320 & 30 \euro & 9 600 \euro & \\ \cline{3-6}
 & & ODU3 Ports & 120 & 60 \euro & 7 200 \euro & \\ \cline{3-6}
 & & ODU4 Ports & 80 & 100 \euro & 8 000 \euro & \\ \cline{3-6}
 & & Line Ports & 1 060 & 1 000 \euro/Gb/s & 106 000 000 \euro & \\ \cline{3-6}
 & \multirow{3}{*}{Optical} & OXCs & 6 & 20 000 \euro & 120 000 \euro & \\ \cline{3-6}
 & & Line Ports & 2 120 & 2 500 \euro/porto & 5 300 000 \euro & \\ \cline{3-6}
 & & Add Ports & 268 & 2 500 \euro/porto & 670 000 \euro & \\
 \hline
 \multicolumn{6}{|| c |}{Total Network Cost} & 642 721 800 \euro \\
\hline
\end{tabular}
\caption{Table with detailed description of CAPEX of Vasco's 2016 results.}
\label{scripttransp_protec_ref_high_heuristic}
\end{table}

\vspace{15pt}

\subsubsection{Conclusions}

Once we have obtained the results for all the scenarios for transparent without survivability and transparent with 1+1 protection we will now draw some conclusions about these results. For a better analysis of the results will be created the table \ref{table_comparative_transparent_protec_heuristic} with the number of line ports, tributary ports and transceivers because they are important values for the cost of CAPEX, the cost of links, the cost of nodes and finally the cost of CAPEX.\\

\begin{table}[H]
\centering
\begin{tabular}{| c | c | c | c |}
 \hline
 & Low Traffic & Medium Traffic & High Traffic \\
 \hline\hline
 Traffic value & 0.5 Tbit/s & 5 Tbit/s & 10 Tbit/s \\ \hline
 Bidirectional Links used & 8 & 8 & 8 \\ \hline
 Number of Line ports & 136 & 566 & 1060 \\ \hline
 Number of Tributary ports & 136 & 1 360 & 2 720 \\ \hline
 Number of Transceivers & 136 & 566 & 1060 \\ \hline
 Link Cost & 68 520 000 \euro & 283 520 000 \euro & 530 520 000 \euro \\ \hline
 Node Cost & 14 547 590 \euro & 59 990 900 \euro & 112 201 800 \euro \\ \hline
 CAPEX without survivability & \textbf{32 247 590 \euro} & \textbf{128 130 900 \euro} & \textbf{238 581 800 \euro} \\ \hline
 CAPEX/bit without survivability & \textbf{64 495 180 \euro} & \textbf{25 626 180 \euro} & \textbf{23 858 180 \euro} \\
 \hline
 CAPEX with 1+1 protection & \textbf{83 067 590 \euro} & \textbf{343 510 900 \euro} & \textbf{642 721 800 \euro} \\ \hline
 CAPEX/bit with 1+1 protection & \textbf{166 135 180 \euro} & \textbf{68 702 180 \euro} & \textbf{64 272 180 \euro} \\
 \hline
\end{tabular}
\caption{Table with different value of CAPEX for this case.}
\label{table_comparative_transp_protec_heuristic}
\end{table}

Looking at the previous table we can make some comparisons between the transparent with 1+1 protection scenario:

\begin{itemize}
  \item Comparing the low traffic with the others we can see that despite having an increase of factor ten (medium traffic) and factor twenty (high traffic), the same increase does not occur in the final cost (it is lower);
  \subitem This happens because the number of the transceivers is lower than expected which leads by carrying the traffic with less network components and, consequently, the network CAPEX is lower;
  \item Comparing the medium traffic with the high traffic we can see that the increase of the factor is double and in the final cost this factor is very close but still inferior;
  \subitem This happens because the number of the transceivers is also lower but very close to the expected;
  \item Comparing the CAPEX cost per bit we can see that in the low traffic the cost is higher than the medium and high traffic, which in these two cases the value is very similar;
  \subitem This happens because the lower the traffic, the higher CAPEX/bit will be. We can see that in medium and high traffic the results tend to be one closer value.
\end{itemize}

We can also make some comparisons between the transparent without survivability and transparent with 1+1 protection scenarios:

\begin{itemize}
  \item We can see that in the transparent with 1+1 protection the CAPEX cost for all the three traffic is more than the double;
    \subitem This happens because in the transparent with 1+1 protection there is a need of having a primary and a backup path, in case of a network failure, and the backup path is typically longer and normally uses more than the double of the capacity of the primary;
  \item The number of the network components and the CAPEX cost are directly proportional to the traffic value. The higher the traffic value, the higher the network CAPEX cost;
  \subitem This happens because if the traffic value is higher, the network components have to be in more quantity to carry all the traffic end-to-end, both in the primary and backup paths;
  \item Comparing the CAPEX cost per bit we can see that has a similar case in both of the two scenarios. In the low traffic the cost is higher than the medium and high traffic, which in these two cases the value is very similar;
  \subitem This happens because the lower the traffic, the higher CAPEX/bit will be. We can see that in medium and high traffic the results tend to be one closer value.
\end{itemize} 
\clearpage

\subsection{Translucent without Survivability}\label{heuristic_Transluc_Survivability}
\begin{tcolorbox}	
\begin{tabular}{p{2.75cm} p{0.2cm} p{10.5cm}} 	
\textbf{Student Name}  &:& Pedro Coelho    (March 01, 2018 - )\\
\textbf{Goal}          &:& Implement the heuristic model for the translucent transport mode without survivability.
\end{tabular}
\end{tcolorbox}

\subsubsection{Model description}

\subsubsection{Result description} 
\clearpage

\subsection{Translucent with 1+1 Protection}\label{heuristic_Transl_Protection}
\begin{tcolorbox}	
\begin{tabular}{p{2.75cm} p{0.2cm} p{10.5cm}} 	
\textbf{Student Name}  &:& Pedro Coelho    (01/03/2018 - )\\
\textbf{Goal}          &:& Implement the heuristic model for the translucent transport mode with 1 plus 1 protection.
\end{tabular}
\end{tcolorbox}

\vspace{11pt}
The translucent networks in the translucent transport mode with 1+1 protection consist also of intermediate optical network architectures between opaque and transparent networks like this transport mode without survivability. In this, there will be created backup paths in order to prevent a network failure and loose all traffic data. The methodology is the same but in this case if, for any reason, there is a link or a path that do not work, then the traffic data can traverse to the same destination path in different links that were created for this purpose. These new created paths are longer than the primary ones and the network CAPEX will be significantly higher. The OEO conversions are also made when the optical signal falls and the maximum optical reach value is used to regenerate the same optical signal.

For cost savings, the translucent networks aims at using the minimum number of regenerators and wavelengths of the network, being the most advantageous transport mode in the optical backbone networks.

We also must take into account the following particularity of this mode of transport:
\begin{itemize}
  \item $N_{OXC,n}$ = 1, \quad $\forall$ n that process traffic
  \item $N_{EXC,n}$ = 1, \quad $\forall$ n that process traffic
\end{itemize}

The minimization of the network CAPEX is made through the equation \ref{Capex_heuristic} where in this case for the cost of nodes we have in consideration the electric cost \ref{Capex_Node_EXC_heuristic} and the optical cost \ref{Capex_Node_OXC_heuristic}.\\

In this case the value of $P_{exc,c,n}$ is obtained by equation \ref{EXC_pexc_translucent_heuristic_protec} for short-reach and by the equation \ref{EXC_pexc2_translucent_heuristic_protec} for long-reach and the value of $P_{oxc,n}$ is obtained by equation \ref{OXC_poxc_translucent_heuristic_protec}.\\

The equation \ref{EXC_pexc_translucent_heuristic_protec} refers to the number of short-reach ports of the electrical switch with bit-rate $c$ in node $n$, $P_{exc,c,n}$, i.e. the number of tributary ports with bit-rate $c$ in node $n$ which can be calculated as

\begin{equation}
P_{exc,c,n} = \sum_{d=1}^{N} D_{nd,c}
\label{EXC_pexc_translucent_heuristic_protec}
\end{equation}

\vspace{11pt}
\noindent
where $D_{nd,c}$ are the client demands between nodes $n$ and $d$ with bit rate $c$.\\

In this case there is the following particularity:
\begin{itemize}
  \item When $n$=$d$ the value of client demands is always zero, i.e, $D_{nn,c}=0$
\end{itemize}

\vspace{11pt}
As previously mentioned, the equation \ref{EXC_pexc2_translucent_heuristic_protec} refers to the number of long-reach ports of the electrical switch with bit-rate -1 in node $n$, $P_{exc,-1,n}$, i.e. the number of add ports of node $n$ which can be calculated as

\begin{equation}
P_{exc,-1,n} = \sum_{j=1}^{N} \lambda_{nj}
\label{EXC_pexc2_translucent_heuristic_protec}
\end{equation}

\vspace{11pt}
\noindent
where $\lambda_{nj}$ is the number of optical channels between node $n$ and node $j$.\\

The equation \ref{OXC_poxc_translucent_heuristic_protec} refers to the number of ports in optical switch in node $n$, $P_{oxc,n}$, i.e. the number of line ports and the number of adding ports of node $n$ which can be calculated as

\begin{equation}
P_{oxc,n} = \sum_{j=1}^{N} f_{nj}^{od} + \sum_{j=1}^{N} \lambda_{nj}
\label{OXC_poxc_translucent_heuristic_protec}
\end{equation}

\vspace{11pt}
\noindent
where $f_{nj}^{od}$ refers to the number of line ports for all demand pairs (od) and $\lambda_{nj}$ refers to the number of add ports.\\

\vspace{11pt}
To implement this heuristic approach there are used algorithms made in Java in a programming software called Eclipse and they are tested in an open-source network program called Net2Plan. In the Net2Plan guide section \ref{net2plan_guide} there is an explanation on how to use and test them in this network planner.

In the next pages it will be described all the steps performed to obtain the final results in the translucent transport mode with 1+1 protection. In the figure below \ref{fluxogram_transl_protec} it is shown a fluxogram with the description of this transport mode approach.

\begin{figure}[H]
\centering
\includegraphics[width=16cm]{sdf/heuristic/translucent_protection/figures/fluxogram_translucent_protec}
\caption{Fluxogram with the steps performed in the translucent with 1+1 protection transport mode approach.}
\label{fluxogram_transl_protec}
\end{figure}

\newpage
\subsubsection{Creation and join the traffic matrices}

\noindent
The first step is to create the traffic matrices based on the reference network \ref{Reference_Network_Topology}. In order to create the 5 traffic matrices in Net2Plan it is necessary the length of all the links and the total traffic used in this network, so later it is needed to define in Net2Plan the length in all end nodes and the total traffic depends on the value of traffic used (low traffic - 0.5 Tbit/s, medium traffic - 5 Tbit/s and high traffic - 10 Tbit/s). As you can see in the figure below, it is defined the path of the 5 ODUs and they will be aggregated in just one single ODU, making it possible to join all the demands in just one file and load it later into the network. This final resulting ODU joins the multiple traffic demands from all the traffic matrices previously created and, of course, the traffic demands will depend on the values used on the creation of the matrices (low, medium and high traffic).

\begin{figure}[h!]
\centering
\includegraphics[width=7cm]{sdf/heuristic/translucent_protection/figures/join_matrices_odus}
\caption{Join the 5 ODU traffic matrices into 1 single file "ODUs". The 5 traffic demands from the traffic matrices previously created are joined into 1 file to load it later on Net2Plan.}
\label{join_matrices_odus_protec_transl}
\end{figure}

\begin{figure}[h!]
\centering
\includegraphics[width=8cm]{sdf/heuristic/translucent_protection/figures/traffic_matrices}
\caption{Load of the join traffic matrices algorithm for the translucent transport mode on Net2Plan. It is defined the 5 paths to load the 5 ODU traffic matrices and the last path is the one where will be saved the file that joins all 5 the traffic demands.}
\label{traffic_matrices_transl_protec_ref}
\end{figure}

\newpage
\subsubsection{Creation of the physical topology}

\vspace{11pt}
The next step is to create the allowed physical topology of the network in Net2Plan. This network consists in 6 nodes and 8 bidirectional links. It is now also possible to define the length in all links. In the figure below it is shown the allowed physical topology in this transport mode.

\begin{figure}[H]
\centering
\includegraphics[width=12cm]{sdf/heuristic/translucent_protection/figures/allowed_physical}
\caption{Allowed physical topology. The allowed physical topology is defined by the duct and sites in the field. It is assumed that each duct supports up to 1 bidirectional transmission system and each site supports up to 1 node.}
\label{allowed_physical_protec_transl}
\end{figure}

\subsubsection{Creation of the logical topology}

\vspace{11pt}
It is now time to create the allowed logical topology. A network topology represents how the links and the nodes of the network interconnect with each other and the logical topology algorithm creates the logical topology on another layer. In the translucent transport mode each node connects to other node if the shortest path distance between them is lower than the maximum optical reach value (in km) that an optical signal can traverse in an optical channel. If the maximum optical reach is higher than the shortest path of a link, then it is not created a logical link of that specific physical link. On the contrary, there are created direct links between all node pairs in the network that follow this rule. These additions of links between end nodes are made in the new upper layer of the network. The respective demands are saved in the new upper layer and those demands from the lower layer are then removed. The lower layer is the physical layer of the network and it is now created a new upper layer which is the logical layer of the network and represents the logical topology of the translucent transport mode.
The allowed physical and optical topologies, the logical topologies for all ODUs and the resulting physical topology is shown in the next section below \ref{result_description_translucent_heuristic_protec} for the three traffic scenarios. It is shown below three figures with the code in Java of the creation of the network logical topology, the load of the logical topology algorithm in Net2Plan and the resulting allowed optical topology for the translucent transport mode with 1+1 protection.

\begin{figure}[H]
\centering
\includegraphics[width=17cm]{sdf/heuristic/translucent_protection/figures/logical_topology_creation_translucent}
\caption{Java code of the logical topology approach for the translucent transport mode. The logical layer is created by adding direct links between all end nodes if their shortest path between them is lower than the maximum optical reach value. The new layer is now the translucent logical topology of the network.}
\label{logical_topology_creation_translucent_protec}
\end{figure}

\begin{figure}[H]
\centering
\includegraphics[width=9cm]{sdf/heuristic/translucent_protection/figures/logical_topology_load_translucent}
\caption{Load of the logical topology algorithm for the translucent transport mode on Net2Plan. It is assumed that the maximum optical reach value is 1000 km, so all direct links between node pairs with different index are created in all optical channels.}
\label{logical_topology_load_translucent_protec}
\end{figure}

\begin{figure}[H]
\centering
\includegraphics[width=12cm]{sdf/heuristic/translucent_protection/figures/allowed_optical}
\caption{Allowed optical topology. It is assumed that each connections between demands supports up to 100 lightpaths. In the translucent transport mode there are several possible logical topologies for a specific network. It depends, for example, on the maximum optical reach value that represents the maximum distance that an optical signal can traverse a link without the need of regenerate.}
\label{allowed_optical_protec_translucent}
\end{figure}

\subsubsection{Creation of routes and aggregation of traffic}

\vspace{11pt}
After a network topology is created, it is now time to set the routing and then grooming algorithms. In the translucent with 1+1 protection transport mode the objective function is to minimize the lightpaths of the network and to find a route for every lightpath demand without considering the wavelength assignment. In this heuristic approach the goal is to minimize the lightpaths blocked and going through the carried lightpaths search for the ones that minimizes the average number of physical hops. Some of the links in the network will suffer from excessive traffic and some other links will occupy only a small part of the bandwidth. In order to do more efficiently, the link's utilizations are almost 1 so there is less non occupied capacity. Firstly, the candidate routes are found by the Dijkstra algorithm for each pair of nodes and calculates the minimum distance for each source and destination nodes and compares with each other. Then, the wavelength assignments are carried out for every lightpaths that have the minimum average of hops for each route and if the path distance is lower than the maximum optical reach value (in this report the maximum optical reach value used is 1000 km).
The algorithm will compare the previous candidate routes that will be saved to a list with the new ones that will be created. If the new routes are different from the previously created ones, if they are the next shortest path routes and if they have the same restrictions as in the translucent without survivability transport mode, then the algorithm will add these routes to the network and they will be the protection segments (backup paths) of the network. The carried lightpaths resulted previously are sequentially processed and the offered traffic demands will be also set into these protection path routes. The routes are saved to a "Set" \ of routes and in each link of end nodes it is set the traffic demands into these routes that will integrate the whole network. The final resulting backup path routes are used to prevent network failures. Despite of the fact the network will be much more secure, the network CAPEX will increase more than the double, due to the creation of primary and backup paths.

\begin{figure}[H]
\centering
\includegraphics[width=10cm]{sdf/heuristic/translucent_protection/figures/grooming_translucent_protec1}
\caption{Creation of routes and aggregation of traffic for the translucent with 1+1 protection transport mode. The candidate routes are searched by the shortest path type method and the average minimum number of physical hops of the routes.}
\label{grooming_translucent_protec1}
\end{figure}

\begin{figure}[H]
\centering
\includegraphics[width=17cm]{sdf/heuristic/translucent_protection/figures/grooming_translucent_protec2}
\caption{Creation of routes and aggregation of traffic for the translucent with 1+1 protection transport mode. The protection segments are added to all the primary paths that were chosen by the shortest path type method.}
\label{grooming_translucent_protec2}
\end{figure}

\begin{figure}[H]
\centering
\includegraphics[width=17cm]{sdf/heuristic/translucent_protection/figures/grooming_translucent_protec3}
\caption{Creation of routes and aggregation of traffic for the translucent with 1+1 protection transport mode. The traffic demands are set into the candidate primary path routes found earlier and also compared with the backup path routes.}
\label{grooming_translucent_protec3}
\end{figure}

\begin{figure}[H]
\centering
\includegraphics[width=15cm]{sdf/heuristic/translucent_protection/figures/grooming_translucent_protec4}
\caption{Creation of routes and aggregation of traffic for the translucent with 1+1 protection transport mode. Minimizing the blocked traffic, the traffic demands are also set into the protection path routes and the wavelength assignment is carried out.}
\label{grooming_translucent_protec4}
\end{figure}

\begin{table}[H]
\centering
\begin{tabular}{|| c | c ||}
 \hline
 Function & Definition \\
 \hline\hline
 netPlan.getDemands(lowerLayer) & Returns the array of demands for the lower layer. \\
 \hline
 d.getRoutes() & Returns all the routes associated to the demand "d". \\
 \hline
 c.getNumberOfHops() & Returns the route number of traversed links. \\
 \hline
 c.getLengthInKm() & \makecell{Returns the route length in km, summing the traversed\\link lengths, as many times as the link is traversed.} \\
 \hline
 c.setCarriedTraffic() & \makecell{Sets the route carried traffic and the occupied capacity\\in the links, setting it up to be the same in all links.} \\
 \hline
 d.getOfferedTraffic() & Returns the offered traffic of the demand "d". \\
 \hline
 save.getSeqLinksRealPath() & Returns the links of routes ordered sequentially. \\
 \hline
 save.addProtectionSegment(segment) & Add "segment" \ as a protection path in the route "save". \\
 \hline
 netPlan.getNodeIds() & Returns the array of the nodes' indexes. \\
 \hline
 netPlan.getNodeFromId(tNodeId) & Returns the node with the index "tNodeId". \\
 \hline
 \makecell{netPlan.getNodePairRoutes\\(in,out,false,lowerLayer)} & \makecell{Returns the routes at "lowerLayer" \ \\from nodes "in" \ and "out".} \\
 \hline
 \makecell{netPlan.getNodePairProtectionSegments\\(in,out,false,lowerLayer)} & \makecell{Returns the protection segments at\\"lowerLayer" \ from nodes "in" \ and "out".} \\
 \hline
 c.getCarriedTraffic() & Returns the route carried traffic at that moment. \\
 \hline
 protect.getCapacity() & Returns the link capacity. \\
 \hline
\end{tabular}
\caption{Table with the description of the main functions in the creation of routes and aggregation of traffic in the grooming algorithm.}
\label{grooming_table_variables_translucent_protec}
\end{table}

\subsubsection{Calculation of the number of wavelengths per link}

\vspace{11pt}
The final step of the routing and grooming algorithms is to calculate the number of wavelengths per link for the whole network. This is the last and an important step because with the number of wavelengths per link in the network, it is possible to calculate other network components. In the translucent transport mode, as in the figure below shows, the algorithm starts with going through all the nodes which have different index between them (end nodes) and if the path distance between them is lower than the maximum optical reach value and in all the links that crosses between these pairs of nodes is reserved a link capacity based on the previous traffic aggregation on routes. The total carried traffic in the link including protection and non-protection segments will be divided by the wavelength capacity and it is now possible to obtain the number of wavelengths per link.

\begin{figure}[H]
\centering
\includegraphics[width=12cm]{sdf/heuristic/translucent_protection/figures/grooming_translucent_protec5}
\caption{Calculation of the number of wavelengths per link for the translucent transport mode. The link capacity is reserved based on the previous traffic aggregation.}
\label{grooming_translucent_protec5}
\end{figure}

\begin{figure}[H]
\centering
\includegraphics[width=9cm]{sdf/heuristic/translucent_protection/figures/grooming_translucent_protec6}
\caption{Load of the grooming algorithm for the translucent with 1+1 protection transport mode. The total number of routes per demand is set to 10, the user can define if the model is with or without protection, the shortest path type is set to "hops" \ and the capacity per wavelength is used 100 optical channels.}
\label{grooming_translucent_protec6}
\end{figure}

\subsubsection{Network cost report}

\vspace{11pt}
In order to obtain the network CAPEX results, the formulas needed to calculate the network elements and that are demonstrated previously in the beginning of this section \ref{heuristic_Transl_Protection} were "translated" \ into Java code in a cost report algorithm. This algorithm can be loaded in Net2Plan and calculates and shows in tables the network CAPEX and also the per-link and per-node information with more details.

\begin{figure}[H]
\centering
\includegraphics[width=10cm]{sdf/heuristic/translucent_protection/figures/cost_report_translucent}
\caption{Load of the cost report algorithm on Net2Plan. The result view is an HTML page with the network optical and electrical components and their costs.}
\label{cost_report_translucent_protec}
\end{figure}

\newpage
\subsubsection{Result description}\label{result_description_translucent_heuristic_protec}

It is already known all the necessary formulas to obtain the CAPEX value for the reference network \ref{Reference_Network_Topology}. As described in the subsection of the network traffic \ref{Reference_Network_Traffic}, it is necessary to obtain three different values of CAPEX for the low (0.5 Tbit/s), medium (5 Tbit/s) and high (10 Tbit/s) traffic. It is used a network software program called Net2Plan which can design the traffic matrices, create all the network topologies, simulate the algorithms into the network implemented in the programming software called Eclipse and analyze the results obtained.\\
In this chapter will be demonstrated the results by Pedro's heuristics. In each of the three traffic scenarios, it will be shown the network topologies followed by the table with the CAPEX value of the network.\\

\noindent
\textbf{Low Traffic Scenario:}\\

In this scenario we have to take into account the traffic calculated in \ref{low_scenario}. In a first phase we will show the various existing topologies of the network. The first are the allowed topologies, physical and optical topologies, the second are the logical topology for all ODUs and finally the resulting physical topology.\\

\begin{figure}[H]
\centering
\includegraphics[width=13cm]{sdf/heuristic/translucent_protection/figures/allowed_physical}
\caption{Allowed physical topology. The allowed physical topology is defined by the duct and sites in the field. It is assumed that each duct supports up to 1 bidirectional transmission system and each site supports up to 1 node.}
\label{allowed_physical_protec_ref_low_heuristic_translucent}
\end{figure}

\begin{figure}[H]
\centering
\includegraphics[width=13cm]{sdf/heuristic/translucent_protection/figures/allowed_optical}
\caption{Allowed optical topology. The allowed optical topology is defined by the transport mode (translucent transport mode in this case). It is assumed that each connections between demands supports up to 100 lightpaths.}
\label{allowed_optical_protec_ref_low_heuristic_translucent}
\end{figure}

\begin{figure}[H]
\centering
\includegraphics[width=13cm]{sdf/heuristic/translucent_protection/figures/logical_topology_odu0_low}
\caption{ODU0 logical topology defined by the ODU0 traffic matrix.}
\label{logical_ODU0_protec_ref_low_heuristic_translucent}
\end{figure}

\begin{figure}[H]
\centering
\includegraphics[width=13cm]{sdf/heuristic/translucent_protection/figures/logical_topology_odu1_low}
\caption{ODU1 logical topology defined by the ODU1 traffic matrix.}
\label{logical_ODU1_protec_ref_low_heuristic_translucent}
\end{figure}

\begin{figure}[H]
\centering
\includegraphics[width=13cm]{sdf/heuristic/translucent_protection/figures/logical_topology_odu2_low}
\caption{ODU2 logical topology defined by the ODU2 traffic matrix.}
\label{logical_ODU2_protec_ref_low_heuristic_translucent}
\end{figure}

\begin{figure}[H]
\centering
\includegraphics[width=13cm]{sdf/heuristic/translucent_protection/figures/logical_topology_odu3_low}
\caption{ODU3 logical topology defined by the ODU3 traffic matrix.}
\label{logical_ODU3_protec_ref_low_heuristic_translucent}
\end{figure}

\begin{figure}[H]
\centering
\includegraphics[width=13cm]{sdf/heuristic/translucent_protection/figures/logical_topology_odu4_low}
\caption{ODU4 logical topology defined by the ODU4 traffic matrix.}
\label{logical_ODU4_protec_ref_low_heuristic_translucent}
\end{figure}

\begin{figure}[H]
\centering
\includegraphics[width=13cm]{sdf/heuristic/translucent_protection/figures/physical_topology}
\caption{Physical topology after dimensioning.}
\label{physical_topology_protec_ref_low_heuristic_translucent}
\end{figure}

Following all the steps mentioned in the \ref{net2plan_guide}, applying the routing and grooming heuristic algorithms in the Net2Plan software and using all the data referring to this scenario, the obtained result for the Pedro's heuristics can be consulted in the following table \ref{scripttransl_protec_ref_low_heuristic}. In table \ref{formulas_transl_heuristic} mentioned in previous model we can see how all the values were calculated. \\

\begin{table}[H]
\centering
\begin{tabular}{|| c | c | c | c | c | c | c ||}
 \hline
 \multicolumn{7}{|| c ||}{CAPEX of the Network} \\
 \hline
 \hline
 \multicolumn{3}{|| c |}{ } & Quantity & Unit Price & Cost & Total \\
 \hline
 \multirow{3}{*}{\makecell{Link \\ Cost}} & \multicolumn{2}{ c |}{OLTs} & 16 & 15 000 \euro & 240 000 \euro & \multirow{3}{*}{23 520 000 \euro} \\ \cline{2-6}
 & \multicolumn{2}{ c |}{100 Gbits/s Transceivers} & 46 & 5 000 \euro/Gbit/s & 23 000 000 \euro & \\ \cline{2-6}
 & \multicolumn{2}{ c |}{Amplifiers} & 70 & 4 000 \euro & 280 000 \euro & \\
 \hline
 \multirow{10}{*}{\makecell{Node \\ Cost}} & \multirow{7}{*}{Electrical} & EXCs & 6 & 10 000 \euro & 60 000 \euro & \multirow{10}{*}{2 142 590 \euro} \\ \cline{3-6}
  & & ODU0 Ports & 60 & 10 \euro/port & 600 \euro & \\ \cline{3-6}
 & & ODU1 Ports & 50 & 15 \euro/port & 750 \euro & \\ \cline{3-6}
 & & ODU2 Ports & 16 & 30 \euro/port & 480 \euro & \\ \cline{3-6}
 & & ODU3 Ports & 6 & 60 \euro/port & 360 \euro & \\ \cline{3-6}
 & & ODU4 Ports & 4 & 100 \euro/port & 400 \euro & \\ \cline{3-6}
 & & Transponders & 18 & 100 000 \euro/port & 1 800 000 \euro & \\ \cline{2-6}
 & \multirow{3}{*}{Optical} & OXCs & 6 & 20 000 \euro & 120 000 \euro & \\ \cline{3-6}
 & & Line Ports & 46 & 2 500 \euro/port & 115 000 \euro & \\ \cline{3-6}
 & & Add Ports & 18 & 2 500 \euro/port & 45 000 \euro & \\
 \hline
 \multicolumn{6}{|| c |}{Total Network Cost} & 25 662 590 \euro \\
\hline
\end{tabular}
\caption{Table with detailed description of CAPEX.}
\label{scripttransl_protec_ref_low_heuristic}
\end{table}

\noindent
\textbf{Medium Traffic Scenario:}\\

In this scenario we have to take into account the traffic calculated in \ref{medium_traffic_scenario}. In a first phase we will show the various existing topologies of the network. The first are the allowed topologies, physical and optical topologies, the second are the logical topology for all ODUs and finally the resulting physical topology.\\

\begin{figure}[H]
\centering
\includegraphics[width=13cm]{sdf/heuristic/translucent_protection/figures/allowed_physical}
\caption{Allowed physical topology. The allowed physical topology is defined by the duct and sites in the field. It is assumed that each duct supports up to 1 bidirectional transmission system and each site supports up to 1 node.}
\label{allowed_physical_protec_ref_medium_heuristic_translucent}
\end{figure}

\begin{figure}[H]
\centering
\includegraphics[width=13cm]{sdf/heuristic/translucent_protection/figures/allowed_optical}
\caption{Allowed optical topology. The allowed optical topology is defined by the transport mode (translucent transport mode in this case). It is assumed that each connections between demands supports up to 100 lightpaths.}
\label{allowed_optical_protec_ref_medium_heuristic_translucent}
\end{figure}

\begin{figure}[H]
\centering
\includegraphics[width=13cm]{sdf/heuristic/translucent_protection/figures/logical_topology_odu0_medium}
\caption{ODU0 logical topology defined by the ODU0 traffic matrix.}
\label{logical_ODU0_protec_ref_medium_heuristic_translucent}
\end{figure}

\begin{figure}[H]
\centering
\includegraphics[width=13cm]{sdf/heuristic/translucent_protection/figures/logical_topology_odu1_medium}
\caption{ODU1 logical topology defined by the ODU1 traffic matrix.}
\label{logical_ODU1_protec_ref_medium_heuristic_translucent}
\end{figure}

\begin{figure}[H]
\centering
\includegraphics[width=13cm]{sdf/heuristic/translucent_protection/figures/logical_topology_odu2_medium}
\caption{ODU2 logical topology defined by the ODU2 traffic matrix.}
\label{logical_ODU2_protec_ref_medium_heuristic_translucent}
\end{figure}

\begin{figure}[H]
\centering
\includegraphics[width=13cm]{sdf/heuristic/translucent_protection/figures/logical_topology_odu3_medium}
\caption{ODU3 logical topology defined by the ODU3 traffic matrix.}
\label{logical_ODU3_protec_ref_medium_heuristic_translucent}
\end{figure}

\begin{figure}[H]
\centering
\includegraphics[width=13cm]{sdf/heuristic/translucent_protection/figures/logical_topology_odu4_medium}
\caption{ODU4 logical topology defined by the ODU4 traffic matrix.}
\label{logical_ODU4_protec_ref_medium_heuristic_translucent}
\end{figure}

\begin{figure}[H]
\centering
\includegraphics[width=13cm]{sdf/heuristic/translucent_protection/figures/physical_topology}
\caption{Physical topology after dimensioning.}
\label{physical_topology_protec_ref_medium_heuristic_translucent}
\end{figure}

Following all the steps mentioned in the \ref{net2plan_guide}, applying the routing and grooming heuristic algorithms in the Net2Plan software and using all the data referring to this scenario, the obtained result for the Pedro's heuristics can be consulted in the following table \ref{scripttransl_protec_ref_medium_heuristic}. In table \ref{formulas_transl_heuristic} mentioned in previous model we can see how all the values were calculated. \\

\begin{table}[H]
\centering
\begin{tabular}{|| c | c | c | c | c | c | c ||}
 \hline
 \multicolumn{7}{|| c ||}{CAPEX of the Network} \\
 \hline
 \hline
 \multicolumn{3}{|| c |}{ } & Quantity & Unit Price & Cost & Total \\
 \hline
 \multirow{3}{*}{\makecell{Link \\ Cost}} & \multicolumn{2}{ c |}{OLTs} & 16 & 15 000 \euro & 240 000 \euro & \multirow{3}{*}{78 520 000 \euro} \\ \cline{2-6}
 & \multicolumn{2}{ c |}{100 Gbits/s Transceivers} & 156 & 5 000 \euro/Gbit/s & 78 000 000 \euro & \\ \cline{2-6}
 & \multicolumn{2}{ c |}{Amplifiers} & 70 & 4 000 \euro & 280 000 \euro & \\
 \hline
 \multirow{10}{*}{\makecell{Node \\ Cost}} & \multirow{7}{*}{Electrical} & EXCs & 6 & 10 000 \euro & 60 000 \euro & \multirow{10}{*}{8 795 900 \euro} \\ \cline{3-6}
  & & ODU0 Ports & 600 & 10 \euro/port & 6 000 \euro & \\ \cline{3-6}
 & & ODU1 Ports & 500 & 15 \euro/port & 7 500 \euro & \\ \cline{3-6}
 & & ODU2 Ports & 160 & 30 \euro/port & 4 800 \euro & \\ \cline{3-6}
 & & ODU3 Ports & 60 & 60 \euro/port & 3 600 \euro & \\ \cline{3-6}
 & & ODU4 Ports & 40 & 100 \euro/port & 4 000 \euro & \\ \cline{3-6}
 & & Transponders & 80 & 100 000 \euro/port & 8 000 000 \euro & \\ \cline{2-6}
 & \multirow{3}{*}{Optical} & OXCs & 6 & 20 000 \euro & 120 000 \euro & \\ \cline{3-6}
 & & Line Ports & 156 & 2 500 \euro/port & 390 000 \euro & \\ \cline{3-6}
 & & Add Ports & 80 & 2 500 \euro/port & 200 000 \euro & \\
 \hline
 \multicolumn{6}{|| c |}{Total Network Cost} & 87 315 900 \euro \\
\hline
\end{tabular}
\caption{Table with detailed description of CAPEX.}
\label{scripttransl_protec_ref_medium_heuristic}
\end{table}

\noindent
\textbf{High Traffic Scenario:}\\

In this scenario we have to take into account the traffic calculated in \ref{high_traffic_scenario}. In a first phase we will show the various existing topologies of the network. The first are the allowed topologies, physical and optical topologies, the second are the logical topology for all ODUs and finally the resulting physical topology.\\

\begin{figure}[H]
\centering
\includegraphics[width=13cm]{sdf/heuristic/translucent_protection/figures/allowed_physical}
\caption{Allowed physical topology. The allowed physical topology is defined by the duct and sites in the field. It is assumed that each duct supports up to 1 bidirectional transmission system and each site supports up to 1 node.}
\label{allowed_physical_protec_ref_high_heuristic_translucent}
\end{figure}

\begin{figure}[H]
\centering
\includegraphics[width=13cm]{sdf/heuristic/translucent_protection/figures/allowed_optical}
\caption{Allowed optical topology. The allowed optical topology is defined by the transport mode (translucent transport mode in this case). It is assumed that each connections between demands supports up to 100 lightpaths.}
\label{allowed_optical_protec_ref_high_heuristic_translucent}
\end{figure}

\begin{figure}[H]
\centering
\includegraphics[width=13cm]{sdf/heuristic/translucent_protection/figures/logical_topology_odu0_high}
\caption{ODU0 logical topology defined by the ODU0 traffic matrix.}
\label{logical_ODU0_protec_ref_high_heuristic_translucent}
\end{figure}

\begin{figure}[H]
\centering
\includegraphics[width=13cm]{sdf/heuristic/translucent_protection/figures/logical_topology_odu1_high}
\caption{ODU1 logical topology defined by the ODU1 traffic matrix.}
\label{logical_ODU1_protec_ref_high_heuristic_translucent}
\end{figure}

\begin{figure}[H]
\centering
\includegraphics[width=13cm]{sdf/heuristic/translucent_protection/figures/logical_topology_odu2_high}
\caption{ODU2 logical topology defined by the ODU2 traffic matrix.}
\label{logical_ODU2_protec_ref_high_heuristic_translucent}
\end{figure}

\begin{figure}[H]
\centering
\includegraphics[width=13cm]{sdf/heuristic/translucent_protection/figures/logical_topology_odu3_high}
\caption{ODU3 logical topology defined by the ODU3 traffic matrix.}
\label{logical_ODU3_protec_ref_high_heuristic_translucent}
\end{figure}

\begin{figure}[H]
\centering
\includegraphics[width=13cm]{sdf/heuristic/translucent_protection/figures/logical_topology_odu4_high}
\caption{ODU4 logical topology defined by the ODU4 traffic matrix.}
\label{logical_ODU4_protec_ref_high_heuristic_translucent}
\end{figure}

\begin{figure}[H]
\centering
\includegraphics[width=13cm]{sdf/heuristic/translucent_protection/figures/physical_topology}
\caption{Physical topology after dimensioning.}
\label{physical_topology_protec_ref_high_heuristic_translucent}
\end{figure}

Following all the steps mentioned in the \ref{net2plan_guide}, applying the routing and grooming heuristic algorithms in the Net2Plan software and using all the data referring to this scenario, the obtained result for the Pedro's heuristics can be consulted in the following table \ref{scripttransl_surv_ref_high_heuristic}. In table \ref{formulas_transl_heuristic} mentioned in previous model we can see how all the values were calculated. \\

\begin{table}[H]
\centering
\begin{tabular}{|| c | c | c | c | c | c | c ||}
 \hline
 \multicolumn{7}{|| c ||}{CAPEX of the Network} \\
 \hline
 \hline
 \multicolumn{3}{|| c |}{ } & Quantity & Unit Price & Cost & Total \\
 \hline
 \multirow{3}{*}{\makecell{Link \\ Cost}} & \multicolumn{2}{ c |}{OLTs} & 16 & 15 000 \euro & 240 000 \euro & \multirow{3}{*}{147 520 000 \euro} \\ \cline{2-6}
 & \multicolumn{2}{ c |}{100 Gbits/s Transceivers} & 294 & 5 000 \euro/Gbit/s & 147 000 000 \euro & \\ \cline{2-6}
 & \multicolumn{2}{ c |}{Amplifiers} & 70 & 4 000 \euro & 280 000 \euro & \\
 \hline
 \multirow{10}{*}{\makecell{Node \\ Cost}} & \multirow{7}{*}{Electrical} & EXCs & 6 & 10 000 \euro & 60 000 \euro & \multirow{10}{*}{16 751 800 \euro} \\ \cline{3-6}
  & & ODU0 Ports & 1 200 & 10 \euro/port & 12 000 \euro & \\ \cline{3-6}
 & & ODU1 Ports & 1 000 & 15 \euro/port & 15 000 \euro & \\ \cline{3-6}
 & & ODU2 Ports & 320 & 30 \euro/port & 9 600 \euro & \\ \cline{3-6}
 & & ODU3 Ports & 120 & 60 \euro/port & 7 200 \euro & \\ \cline{3-6}
 & & ODU4 Ports & 80 & 100 \euro/port & 8 000 \euro & \\ \cline{3-6}
 & & Transponders & 154 & 100 000 \euro/port & 15 400 000 \euro & \\ \cline{2-6}
 & \multirow{3}{*}{Optical} & OXCs & 6 & 20 000 \euro & 120 000 \euro & \\ \cline{3-6}
 & & Line Ports & 294 & 2 500 \euro/port & 735 000 \euro & \\ \cline{3-6}
 & & Add Ports & 154 & 2 500 \euro/port & 385 000 \euro & \\
 \hline
 \multicolumn{6}{|| c |}{Total Network Cost} & 164 271 800 \euro \\
\hline
\end{tabular}
\caption{Table with detailed description of CAPEX.}
\label{scripttransl_protec_ref_high_heuristic}
\end{table}

\vspace{13pt}
\subsubsection{Conclusions}

Once we have obtained the results for all scenarios for the translucent without survivability and translucent with 1+1 protection we will now draw some conclusions about these results. For a better analysis of the results will be created the table \ref{table_comparative_transl_protec_heuristic} with the number of line ports, tributary ports and transceivers because they are important values for the cost of CAPEX, the cost of links, the cost of nodes and finally the cost of CAPEX.\\

\begin{table}[H]
\centering
\begin{tabular}{| c | c | c | c |}
 \hline
 & Low Traffic & Medium Traffic & High Traffic \\
 \hline\hline
 \makecell{CAPEX \\ without survivability} & 11 592 590 \euro & 49 125 900 \euro & 93 921 800 \euro \\ \hline
 \makecell{CAPEX/Gbit/s \\ without survivability} & 23 185 \euro/Gbit/s & 9 825 \euro/Gbit/s & 9 392 \euro/Gbit/s \\ \hline
 Traffic (Gbit/s) & 500 & 5 000 & 10 000 \\ \hline
 Bidirectional Links used & 8 & 8 & 8 \\ \hline
 Number of Add ports & 18 & 80 & 154 \\ \hline
 Number of Line ports & 46 & 156 & 294 \\ \hline
 Number of Tributary ports & 136 & 1 360 & 2 720 \\ \hline
 Number of Transceivers & 46 & 156 & 294 \\ \hline
 Link Cost & 23 520 000 \euro & 78 520 000 \euro & 147 520 000 \euro \\ \hline
 Node Cost & 2 142 590 \euro & 8 795 900 \euro & 16 751 800 \euro \\ \hline
 CAPEX & \textbf{25 662 590 \euro} & \textbf{87 315 900 \euro} & \textbf{164 271 800 \euro} \\ \hline
 CAPEX/Gbit/s & \textbf{51 325 \euro/Gbit/s} & \textbf{17 463 \euro/Gbit/s} & \textbf{16 427 \euro/Gbit/s} \\ \hline
\end{tabular}
\caption{Table with different value of CAPEX for this case.}
\label{table_comparative_transl_protec_heuristic}
\end{table}

\noindent
Looking at the previous table we can make some comparisons between the translucent with 1+1 protection scenario:

\begin{itemize}
  \item Comparing the low traffic with the others we can see that despite having an increase of factor ten (medium traffic) and factor twenty (high traffic), the same increase does not occur in the final cost (it is lower);
  \subitem This happens because the number of the transceivers is lower than expected which leads by carrying the traffic with less network components and, consequently, the network CAPEX is lower;
  \item Comparing the medium traffic with the high traffic we can see that the increase of the factor is double and in the final cost this factor is very close but still inferior;
  \subitem This happens because the number of the transceivers is also lower but very close to the expected;
  \item Comparing the CAPEX cost per bit we can see that in the low traffic the cost is higher than the medium and high traffic, which in these two cases the value is similar, but still inferior in the higher traffic. The difference between the low traffic and medium/high traffics is significantly high;
  \subitem This happens because the lower the traffic, the higher CAPEX/bit will be. We can see that in medium and high traffic the results tend to be one closer and lower value.
\end{itemize}

\noindent
We can also make some comparisons between the translucent without survivability and translucent with 1+1 protection scenarios:

\begin{itemize}
  \item We can see that in the translucent with 1+1 protection transport mode the CAPEX cost for all the three traffic is more than the double;
    \subitem This happens because in the translucent with 1+1 protection transport mode there is a need of having a primary and a backup path, in case of a network failure, and the backup path is typically longer;
  \item Comparing the CAPEX cost per bit we can see that has a similar case in both of the two scenarios. In the low traffic the cost is higher than the medium and high traffic, which in these two cases the value is similar;
  \subitem This happens because the lower the traffic, the higher CAPEX/bit will be. We can see that in medium and high traffic the results tend to be one closer and lower value.
\end{itemize}

\vspace{13pt}
\subsubsection{Opens Issues}

The creation of this model for any scenario, started with some considerations and some open issues being:

\begin{itemize}
  \item Allow maximum optical reach value in optical channels.
  \subitem It is assumed that the maximum optical reach value is 1000 km, so all direct links between node pairs with different index are created in all optical channels;
  \item Allow multiple transmission system.
  \subitem The presented model for each link only supports one transmission system.
\end{itemize} 
%Section OPEX
%Subsection with the different transport mode 