\clearpage

\section{Future directions}
\label{future}

Throughout this dissertation specific situations were analyzed and some open uses were discovered. Future work suggests the following important topics:\\

\textbf{Opaque transport mode}
\begin{itemize}
  \item Allow blocking because the presented model assume that the solution is possible or impossible, does not support a partial solution where some demands are not routed.
  \item Assume a multiple transmission system, that is, for each link there is more than one transmission system.
  \item Allowing multi-path routing, so that not all demands that sharing the same end nodes have to follow the same path.
\end{itemize}

\textbf{Transparent transport mode}
\begin{itemize}
  \item Allow blocking because the presented model assume that the solution is possible or impossible, does not support a partial solution where some demands are not routed.
  \item Assume a multiple transmission system, that is, for each link there is more than one transmission system.
\end{itemize}

\textbf{Translucent transport mode}
\begin{itemize}
  \item Allow blocking because the presented model assume that the solution is possible or impossible, does not support a partial solution where some demands are not routed.
  \item Assume a multiple transmission system, that is, for each link there is more than one transmission system.
  \item Consent to a Maximum Reach.
  \item Define the variable $N_{oxc}$ as not being fixed allowing only certain nodes instead of all.
\end{itemize}

\textbf{Analytical model}
\begin{itemize}
  \item It's necessary to focus on the calculation of the CAPEX for translucent mode.
  \item Include the LR transponders in the node instead of being calculated on the link.
\end{itemize}
