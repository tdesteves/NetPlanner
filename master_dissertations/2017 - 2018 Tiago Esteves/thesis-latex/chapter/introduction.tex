\chapter{Introduction}
\label{introduction}

The amount of traffic, in particular IP traffic, has been increasing very substantially. This increase is due to the growing number of Internet-based applications, the increase in the number of devices connected to the Internet, the expansion of optical fiber to customers' homes, increased bandwidth of mobile access technologies, and increased of video traffic.
At the same time, with the increase in traffic, operators are under heavy pressure to reduce the cost per bit transported. This implies the introduction of new technologies, which on the one hand increase the capacity of transport of the networks and on the other, reduce the costs of operation (OPEX).
This process of technological conversion is operating in a macroeconomic scenario in which operators find it difficult to finance which forces them to have strong investment constraints (CAPEX).
The transport networks have been networks predominantly based on circuit switching, either at the level of the optical channels or at the level of the electrical circuits, and the introduction of packet switching undermines this paradigm.
In this scenario, particularly considering the increase in packet traffic, packet switching solutions for transport networks and mixed solutions have been presented by the device manufacturers where packet and circuit switching coexists on the same equipment.

\newpage
%%%%%%%%%%%%%%%%%%%%%%%%%%%%%%%%%%%%%%%%%%%%%%%%%%%%%%%%%%%%%%%%%%%%%%%%
\section{Motivation and objectives}
\label{objectives}
Falta motivação.

To achieve the main objectives of this dissertation, the following steps must be taken:

\begin{enumerate}
  \item Define one reference network and three different scenarios for performing tests.
  \item Develop ILP models for opaque, transparent and translucent networks without protection and using 1 + 1 protection.
  \item Develop ILP models for opaque, transparent and translucent networks with 1 + 1 protection.
  \item Get analytical solutions for the two previous points.
  \item Compare the analytical results and results based on ILP with the results obtained through heuristics.
\end{enumerate}


%%%%%%%%%%%%%%%%%%%%%%%%%%%%%%%%%%%%%%%%%%%%%%%%%%%%%%%%%%%%%%%%%%%%%%%%%%%%%%%%%%%%%%%
\section{Thesis outline}
\label{outline}

This thesis is organized in 7 chapters. Chapter \ref{chap_reference_network} consists of a state-of-art review about optical transport networks. In this chapter is also where the reference network used throughout the dissertation as well as the different traffics used is defined. The Chapter \ref{chap_capex} begins by determining the CAPEX calculation formula for use in the ILP model and for analytical calculations. The first section refers to ILP models and the other to analytical models. In Chapter \ref{chap_ilp} are several sections each for a particular mode of transport and certain survivability. In section \ref{ILP_Opaque_Survivability} we have opaque without survivability, in section \ref{ILP_Opaque_Protection} opaque with 1+1 protection. Sections \ref{ILP_Transp_Survivability} and \ref{ILP_Transp_Protection} relate to the transparent and lastly sections \ref{ILP_Transluc_Survivability} and \ref{ILP_Transluc_Protection} refer to the translucent. In the referred section it is possible to see the model description, the detailed description of the results and the conclusions of these results. The analytical calculation of all the models referred to in Chapter \ref{chap_ilp} can be found in Chapter \ref{chap_analytical}. In Chapter \ref{chap_comparative} the results obtained throughout this dissertation are compared and the chapter is divided into six sections where each corresponds to a certain mode of transport with their respective survivability. The last step is the conclusions \ref{chap_conclusions} and suggestions for future research directions.

%%%%%%%%%%%%%%%%%%%%%%%%%%%%%%%%%%%%%%%%%%%%%%%%%%%%%%%%%%%%%%%%%%%%%%%%%%%%%%%%%%%%%%%%%%%%%%%%%%%%%%%%%%%%%%%%%%%%%%%%%%%%%
