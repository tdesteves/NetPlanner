
% ------------------------------------------------------------------------
\chapter{Network specification}
\label{chap_reference_network}
The purpose of this chapter is to describe a state-of-art review about optical transport networks and finally describe a reference network that will be used for the various types of dimensioning throughout this dissertation. In addition to the reference network will also be described the various traffic models used in this network in question.\\
The organization of this chapter is done by creating four subsections, the first \ref{network_components} refers to the components of the network, the second \ref{network_topologies} depicts the topologies of the network and in the third \ref{transport_mode} it is possible to describe the different types of mode of transport. At the end, in the \ref{reference_network} is described the physical topology of the network and how to create the traffic matrix for the three existing traffic models (low, medium and high traffic).\\


% ------------------------------------------------------------------------
\chapter{Reference Network Specification}

The purpose of this chapter is to describe a reference network that will be used for the various types of dimensioning throughout this project. In addition to the reference network will also be described the various traffic models used in this network in question.\\
The organization of this chapter is done by creating two sub-chapters, the first to describe the physical topology of the network and a second to create the traffic matrix for the three existing traffic models (low, medium and high traffic).\\


% ------------------------------------------------------------------------
\chapter{Reference Network Specification}

The purpose of this chapter is to describe a reference network that will be used for the various types of dimensioning throughout this project. In addition to the reference network will also be described the various traffic models used in this network in question.\\
The organization of this chapter is done by creating two sub-chapters, the first to describe the physical topology of the network and a second to create the traffic matrix for the three existing traffic models (low, medium and high traffic).\\


% ------------------------------------------------------------------------
\chapter{Reference Network Specification}

The purpose of this chapter is to describe a reference network that will be used for the various types of dimensioning throughout this project. In addition to the reference network will also be described the various traffic models used in this network in question.\\
The organization of this chapter is done by creating two sub-chapters, the first to describe the physical topology of the network and a second to create the traffic matrix for the three existing traffic models (low, medium and high traffic).\\

\input{./sdf/reference_network/reference_network}




%%%%%%%%%%%%%%%%%%%%%%%%%%%%%%%%%%%%%%%%%%%%%%%%%%%%%%%%%%%%%%%%%%%%%%%%%%%%%%%%%%%%%%%%%%%%%%%%%%%%%%%%%%%%%%%%%%%%%%%%%%%%%

% References
\phantomsection
\addcontentsline{toc}{section}{References}
%
\renewcommand{\bibname}{References}
%\bibliographystyle{IEEEtran}
%\bibliography{rmorais}
%
%
% Generated by IEEEtran.bst, version: 1.13 (2008/09/30)
\begin{thebibliography}{10}
\providecommand{\url}[1]{#1}
\csname url@samestyle\endcsname
\providecommand{\newblock}{\relax}
\providecommand{\bibinfo}[2]{#2}
\providecommand{\BIBentrySTDinterwordspacing}{\spaceskip=0pt\relax}
\providecommand{\BIBentryALTinterwordstretchfactor}{4}
\providecommand{\BIBentryALTinterwordspacing}{\spaceskip=\fontdimen2\font plus
\BIBentryALTinterwordstretchfactor\fontdimen3\font minus
  \fontdimen4\font\relax}
\providecommand{\BIBforeignlanguage}[2]{{%
\expandafter\ifx\csname l@#1\endcsname\relax
\typeout{** WARNING: IEEEtran.bst: No hyphenation pattern has been}%
\typeout{** loaded for the language `#1'. Using the pattern for}%
\typeout{** the default language instead.}%
\else
\language=\csname l@#1\endcsname
\fi
#2}}
\providecommand{\BIBdecl}{\relax}
\BIBdecl

\bibitem{book07}
E.~Bouillet, G.~Ellinas, J.-F. Labourdette, and R.~Ramamurthy, \emph{Path Routing in Mesh Optical Networks}, John Wiley \& Sons, 2007.

\bibitem{ramas2010}
R. Ramaswami, K. N.~Sivarajan, and G. H. Sasaki, \emph{Optical Networks: A Practical Perspective}, Morgan Kaufmann, 2010.

\bibitem{tesevasco}
V.~R.~B.~S. Braz, ``Dimensioning and Optimization of Node Architecture in Optical Transport Networks.'' Master's thesis, Universidade de Aveiro, 2016.

\bibitem{teserui}
R.~M.~D. Morais, ``Planning and Dimensioning of Multilayer Optical Transport Networks.'' PhD thesis, Universidade de Aveiro, 2015.

\bibitem{aulas2}
A.~N. Pinto, ``Optical Networks - Topology,'' in \emph{Aulas de redes opticas 2016-2017.}

\bibitem{teselisboa}
A.~M.~P. Fernandes, ``Estratégias de Planeamento para Tráfego Estático e Dinâmico em Redes de Transporte Óticas.'' Master's thesis, Universidade de Lisboa, 2017.

\bibitem{opaque}
A. Autenrieth, A.~K. Tilwankar, C.~M. Machuca, and J.~P. Elbers, ``Power consumption analysis of opaque and transparent optical core networks.'' in \emph{Proc. 13th Int. Conf. Transparent Optical Networks}, pp. 1-5, June 2011.

\bibitem{transparent}
H. Abdalla, H.~A.~F. Crispim, E.~T.~L. Pastor, A.~J.~M. Soares, and L.~A. Bermudez, ``Optical transparent IP/WDM network simulation,'' in \emph{Proc. SBMO/IEEE MTT-S Int. Conf. Microwave and Optoelectronics}, pp. 19-23, July 2005.

\bibitem{zhu}
K. Zhu, H. Zhu, and B. Mukherjee, ``Traffic Grooming in an Optical WDM Mesh Network,'' Springer, 2002.

\bibitem{alcatel}
\BIBentryALTinterwordspacing
Alcatel-Lucent (2010). ``Understanding OTN Optical Transport Network (G.709), March 9, 2010'' [Online]. Available:
  \url{http://www.cvt-dallas.org/March2010.pdf}
\BIBentrySTDinterwordspacing

\end{thebibliography} 