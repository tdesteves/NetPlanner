
\subsection{Opaque transport mode}\label{analytical_Opaque_Mode}

One more time, before executing the equation of the variables we must take into account the particularities of this mode of transport:
\begin{itemize}
  \item $C_{oxc}$ = 0
  \item $\xi$ = 1
  \item $<k>$ = 0 or $<k>$ = $<kp>$ (depending of survivability)
\end{itemize}

The first particularity exists because in this mode of transport there is no optical cost, in the case of the second we are assuming that the grooming coefficient has value 1 and finally in the last particularity we are assuming that the survivability coefficient is zero when it is without survivability or $<kp>$ when it is with 1+1 protection \cite{aulas} where

\begin{equation}
<kp> = \frac{<h'>}{<h>}
\label{coefficient_protec}
\end{equation}

\vspace{13pt}
Finally looking at the equation \ref{analytical_electricalCost} we can see that we already have practically all the values with the exception of two variables. The tributary ports, $P_{TRIB}$, can be calculated through the ODU's matrices referred to in section \ref{Reference_Network_Traffic} and the average number of ports the electrical switch,$<P_{exc}>$, that can be calculated as

\begin{equation}
<P_{exc}> = <d> <h> \left(1 + <k>\right)
\label{Pexc_opaque}
\end{equation}

\noindent
where $<d>$ is the average number of demands, $<h>$ is the average number of hops and $<k>$	is the survivability coefficient. The number of ports of the electrical switch, in this case, is equal to the number of line ports since we already know the number of tributary ports \cite{aulas}.
The variable $<d>$ is calculated through the equation \ref{average_demand}

\begin{equation}
<d> = \frac{D}{N}
\label{average_demand}
\end{equation}
